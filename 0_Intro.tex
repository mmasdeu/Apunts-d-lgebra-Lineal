% !TeX encoding = UTF-8
% !TeX spellcheck = ca_ES-valencia
% !TeX root = MatCADAlgLin.tex
\section*{Introducció}
Encara que el curs serà força autocontingut es requerirà que l'alumne conegui la resolució de sistemes d'equacions lineals, l'aritmètica bàsica de números i polinomis, i que tingui destresa de càlcul amb expressions algebraiques simbòliques.

A tot aquest curs suposarem que treballem sobre un cos commutatiu $\K$ fixat, que podeu pensar és $\Q$, $\R$ o $\C$. Els elements de $K$ els anomenarem nombres o escalars. Les propietats que utilitzarem són:
\begin{itemize}
	\item És commutatiu amb la suma: $a+b=b+a$ $\forall a,b\in \K$.
	\item És commutatiu amb el producte: $ab=ba$ $\forall a,b\in \K$.
	\item La suma té un element neutre que anomenem zero: $0+a=a$ $\forall a\in\K$.
	\item El producte té un elements neutre que anomenem u: $1a=a$ $\forall a\in\K$.
	\item Tot element $a\in\K$ té un invers per la suma que anomenem $-a$: $a+(-a)=0$.
	\item Tot element $a$ diferent de zero té un invers per la multiplicació que anomenem $1/a$ o bé $a^{-1}$: $a a^{-1}=1$.
	\item Hi ha les propietats associatives a la suma i al producte: $(a+b)+c=a+(b+c)$ i $(ab)c=a(bc)$ $\forall a,b,c \in \K$.
	\item Hi ha la propietat distributiva: $a(b+c)=ab+ac$ $\forall a,b,c \in \K$.
\end{itemize}

També suposarem certa familiaritat amb el llenguatge dels conjunts. Si $A$ és un conjunt i $B$ un subconjunt d'$A$, escriurem $B\subset A$. $a\in A$ voldrà dir que $a$ és un element d'$A$. També escriurem $A\setminus B=\{a \in A \mid a \not\in B\}$ i llegirem els $a$ que pertanyen a $a$ i que no pertanyen a $B$ (o bé el complementari de $B$ en $A$). 
 
