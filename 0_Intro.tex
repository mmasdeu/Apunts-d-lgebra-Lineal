% !TeX encoding = UTF-8
% !TeX spellcheck = ca_ES-valencia
% !TeX root = MatCADAlgLin.tex
\section*{}
Aquests apunts corresponen a l'assignatura d'Àlgebra Lineal que s'ha fet al curs 2018/2019 del Grau en Matemàtica Computacional i Analítica de Dades a la UAB. És un curs de 52 hores docents on es barregen les classes teòriques i pràctiques. Per a les classes teòriques (i aquests apunts) s'han utilitzat les referències \cite{Bret} i \cite{NaXa}. Cada capítol acaba amb una selecció d'exercicis proposats, agafant la numeració de \cite{Bret}. A més a més, aquest curs s'ha complementat amb tres sessions de \textbf{Sage} que mostren aplicacions d'aquests resultats.

Encara que aquests apunts intenten ser força autocontinguts, es requereix que l'alumne conegui la resolució de sistemes d'equacions lineals, l'aritmètica bàsica de números i polinomis, i que tingui destresa de càlcul amb expressions algebraiques simbòliques.

A tot aquest curs suposem que treballem sobre un cos commutatiu $\K$ fixat, que podeu pensar és $\Q$, $\R$ o $\C$. Els elements de $\K$ els anomenarem nombres o escalars. Les propietats que utilitzarem són:
\begin{itemize}
	\item És commutatiu amb la suma: $a+b=b+a$ $\forall a,b\in \K$.
	\item És commutatiu amb el producte: $ab=ba$ $\forall a,b\in \K$.
	\item La suma té un element neutre que anomenem zero: $0+a=a$ $\forall a\in\K$.
	\item El producte té un elements neutre que anomenem u: $1a=a$ $\forall a\in\K$.
	\item Tot element $a\in\K$ té un invers per la suma que anomenem $-a$: $a+(-a)=0$.
	\item Tot element $a$ diferent de zero té un invers per la multiplicació que anomenem $1/a$ o bé $a^{-1}$: $a a^{-1}=1$.
	\item Hi ha les propietats associatives a la suma i al producte: $(a+b)+c=a+(b+c)$ i $(ab)c=a(bc)$ $\forall a,b,c \in \K$.
	\item Hi ha la propietat distributiva: $a(b+c)=ab+ac$ $\forall a,b,c \in \K$.
\end{itemize}

També suposem certa familiaritat amb el llenguatge dels conjunts. Si $A$ és un conjunt, escriurem $B\subset A$ per denotar que $B$ és un subconjunt d'$A$. Escriurem $a\in A$ per dir que $a$ és un element d'$A$. També escriurem $A\setminus B=\{a \in A \mid a \not\in B\}$ i llegirem els $a$ que pertanyen a $A$ i que no pertanyen a $B$ (o bé el complementari de $B$ en $A$). 
