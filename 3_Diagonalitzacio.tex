% !TeX encoding = UTF-8
% !TeX spellcheck = ca_ES-valencia
% !TeX root = MatCADAlgLin.tex
El contingut d'aquesta secció és pot trobar a \cite[Temes 6, 7]{Bret} i a \cite[Tema 4]{NaXa}.
\subsection{Motivació}\label{subsec:motiv_diag}
Considerem les matrius següents:
$$
A=\begin{pmatrix}
-1 & 0 & 0 \\ 0 & 2 & 0 \\ 0 & 0 & 0
\end{pmatrix}
\text{ i }
B=\begin{pmatrix}
5 & -15 & -21 \\ -3 & 9 & 13 \\ 3 & -9 & -13
\end{pmatrix}
$$
i suposem que volem calcular $A^5$, $\Rang(A)$, $\Ker(f_A)$ o una base de $\Ima(f_A)$; i exactament els mateixos càlculs per $B$.

Per $A$, aquests càlculs són quasi immediats:
$$
A^5=\begin{pmatrix}
(-1)^5 & 0 & 0 \\ 0 & 2^5 & 0 \\ 0 & 0 & 0
\end{pmatrix} \,
$$
$\Rang(A)=2$, 
$\Ker(f_A)=\langle \left( \begin{smallmatrix} 0 \\ 0 \\ 1 \end{smallmatrix} \right) \rangle$ i  una base de $\Ima(f_A)$ pot ser $\calb=( \left( \begin{smallmatrix} -1 \\ 0 \\ 0 \end{smallmatrix} \right) , \left( \begin{smallmatrix} 0 \\ 2 \\ 0 \end{smallmatrix} \right))$.

Per la matriu $B$, podem fer $B^5$, però ens porta més càlculs, igual que calcular-ne el rang, el nucli de $f_B$ o una base de la imatge de $f_B$.

Observem ara que hi ha una relació entre $A$ i $B$:
$$
B=S A S^{-1}
$$
on 
$$
S=\begin{pmatrix} 1 & 2 & 3 \\ -1 & -1 & 1 \\ 1 & 1 & 0
\end{pmatrix}
$$
i això permet aprofitar els càlculs que hem fet per $A$ per deduir els de $B$:
$$
B^5=S A S^{-1} S A S^{-1} \cdots S A S^{-1}=S A^5 S^{-1} \, ,
$$
tenim la igualtat $\Rang(B)=\Rang(A)=2$, $\Ker(f_B)=S\Ker(f_A)$ i una base de $\Ima(f_B)$ s'aconsegueix amb la base de $\Ima(f_A)$ multiplicada (per l'esquerra) per la matriu $S^{-1}$.

A aquest exemple, el que hem vist és que encara que les matrius $A$ i $B$ corresponen a una mateixa transformació lineal $f$ expressada en dues bases diferents, els càlculs han quedat molt més fàcils amb la matriu $A$ pel simple fet de ser una matriu diagonal. A aquest capítol, l'objectiu principal és: donada una aplicació lineal $f\colon E \to E$ (amb $E$ un espai vectorial de dimensió finita), trobar una base $\calb$ d'$E$ tal que $[f]_\calb$ sigui una matriu diagonal. També veurem que, a vegades, no existeix cap base amb aquesta propietat.

\subsection{Determinants}
Probablement ja coneixem el determinant de matrius $2\times 2$ i $3\times 3$ definits directament com:
\begin{equation}\label{eq:det2}
\begin{vmatrix}
a & b \\ c & d  
\end{vmatrix}= ad -bc
\end{equation}
i
\begin{equation}\label{eq:det3}
\begin{vmatrix}
a_{11} & a_{12} & a_{13}\\
a_{21} & a_{22} & a_{23}\\
a_{31} & a_{32} & a_{33}
\end{vmatrix} = a_{11}a_{22}a_{33}-a_{11}a_{23}a_{32}-a_{12}a_{21}a_{33}+a_{12}a_{23}a_{31}+a_{13}a_{21}a_{32}-a_{13}a_{22}a_{31}.
\end{equation}
Una de les propietats principals és la següent: el determinant val zero si i només si els vectors columna de la matriu són linealment dependents.

Si pensem les columnes de les matrius com a vectors de $\K^n$, podem considerar el determinant com una aplicació:
$$
\begin{array}{rcl}
\det \colon M_n(\K)=(\K^n)^n & \longrightarrow & \K \\
(\vec v_1, \dots , \vec v_n) & \mapsto & \det(\vec v_1, \dots , \vec v_n)
\end{array}$$
i habitualment escriurem:
$$
\det(\vec v_1, \dots, \vec v_n)=\begin{vmatrix}
\mid & \mid & & \mid \\
\vec v_1 & \vec v_2 & \cdots & \vec v_n \\
\mid & \mid & & \mid
\end{vmatrix}
$$
Demanarem que el determinant compleixi les propietats següents:
\begin{enumerate}[\bf {D}1.]
    \item $\det(\vec v_1, \dots, \lambda \vec v_j, \dots,\vec v_n)=\lambda \det(\vec v_1, \dots, \vec v_j, \dots,\vec v_n)$ per a tots $\vec v_j\in \K^n$ i $\lambda \in \K$.
    \item $\det(\vec v_1, \dots, \vec v_{j-1}, \vec v_j+\vec w, \vec v_{j+1} \dots,\vec v_n)= \det(\vec v_1, \dots, \vec v_{j-1},\vec v_j,\vec v_{j+1}, \dots,\vec v_n) + $\\  $\det(\vec v_1, \dots, \vec v_{j-1},\vec w,\vec v_{j+1}, \dots,\vec v_n)$ per a tots $\vec v_k$ i $\vec w\in \K^n$.
    \item $\det(\vec v_1,\dots , \vec v_j, \dots, \vec v_k, \dots ,\vec v_n)=0$ si $\vec v_j=\vec v_k$ amb $j\neq k$.
    \item $\det(\1_n)=\det(\vec e_1, \dots, \vec e_n)=1$, on $\vec e_j$ són els vectors estàndard.
\end{enumerate}
\begin{exercici}
Demostreu que els determinants de matrius $2\times 2$ i $3\times 3$ definits a les Equacions \eqref{eq:det2} i \eqref{eq:det3} respectivament compleixen aquestes propietats i que són les úniques aplicacions de $M_2(\K) \to \K$ i $M_3(\K)\to\K$ respectivament que les compleixen.
\end{exercici}

\begin{observacio}\label{obs:det0}
Observem que \textbf{D1} implica que $\det(\vec v_1, \dots, \vec 0, \dots,\vec v_n)=0$: com que $\vec 0=0 \cdot \vec 0$:
\begin{align*}
\det(\vec v_1, \dots, \vec 0, \dots,\vec v_n) & = \det(\vec v_1, \dots, 0\cdot\vec 0, \dots,\vec v_n) = \\
& = 0 \det(\vec v_1, \dots, \vec 0, \dots,\vec v_n)=0
\end{align*}
\end{observacio}

\begin{lema}\label{lema:det_i_trans_elem}
Si considerem les transformacions elementals per columnes corresponents a les transformacions \textbf{T1}, \textbf{T2} i \textbf{T3} definides a la Subsecció \ref{subsec:trans_el}, el determinant es modifica com:
\begin{itemize}
    \item \textbf{T1}: Si $B$ és la matriu resultant de multiplicar una columna per $\lambda$ a la matriu $A$:
    $$
    \det(B)=\lambda \det(A).
    $$.
    \item \textbf{T2}: Si $B$ és la matriu resultant de sumar a una columna d'$A$ $\mu$ vegades una altra columna d'$A$:
    $$
    \det(B)=\det(A).
    $$
    \item \textbf{T3}: Si $B$ és la matriu resultant d'intercanviar dues columnes diferents d'$A$, llavors:
    $$
    \det(B)=-\det(A).
    $$
\end{itemize}
\end{lema}
\begin{proof}
  \textbf{D1} ens diu com es transforma $\det$ per la transformació elemental \textbf{T1}, obtenint el resultat de l'enunciat.
  
      Estudiem ara el canvi \textbf{T2}: sumem a una columna $\mu$ vegades una altra: 
    \begin{align*}
    \det(\dots, \vec v_j, \dots, \vec v_k+\mu\vec v_j,\dots) & =\det(\dots, \vec v_j, \dots, \vec v_k,\dots) + \det(\dots, \vec v_j, \dots,\mu\vec v_j,\dots)= \\ &
    = \det(\dots, \vec v_j, \dots, \vec v_k,\dots) + \mu\det(\dots, \vec v_j, \dots,\vec v_j,\dots)= \\ &
    = \det(\dots, \vec v_j, \dots, \vec v_k,\dots) + 0
    \end{align*}
    on primer hem aplicat \textbf{D2} (separar la suma de vectors), llavors \textbf{D1} (treure el $\mu$ fora del determinant) i finalment \textbf{D3} per veure que un dels determinants és zero.
  
  Mirem com es modifica el determinant per la transformació \textbf{T3}: vegem que si intercanviem dues columnes $j\neq k$, llavors
    $$
    \det(\vec v_1,\dots , \vec v_j, \dots, \vec v_k, \dots ,\vec v_n)=-\det(\vec v_1,\dots , \vec v_k, \dots, \vec v_j, \dots ,\vec v_n) .
    $$
    Per tal de simplificar la notació escrivim $\det(\dots , \vec v_j, \dots, \vec v_k, \dots)$:
    \begin{align*}
        0  & =  \det(\dots , \vec v_j+\vec v_k, \dots, \vec v_j+\vec v_k, \dots)  = \\
          & =  \det(\dots , \vec v_j+\vec v_k, \dots, \vec v_j, \dots) + \det(\dots , \vec v_j+\vec v_k, \dots, \vec v_k, \dots) = \\
          & =  \det(\dots , \vec v_j, \dots, \vec v_j, \dots) + \det(\dots , \vec v_k, \dots, \vec v_j, \dots) + \\
           &  \quad + \det(\dots , \vec v_j, \dots, \vec v_k, \dots) + \det(\dots , \vec v_k, \dots, \vec v_k, \dots) = \\
           & =  0 + \det(\dots , \vec v_k, \dots, \vec v_j , \dots) + \det(\dots , \vec v_j, \dots, \vec v_k , \dots) + 0
    \end{align*}
\end{proof}

\begin{teorema}
 Si tenim dues aplicacions $\det\colon (\K^n)^n \to \K$ i $\det'\colon (\K^n)^n \to \K$ complint \textbf{D1}, \textbf{D2}, \textbf{D3} i \textbf{D4}, llavors $\det=\det'$.
\end{teorema}
\begin{proof}
    Volem veure que si $\vec v_1, \dots , \vec v_n \in \K^n$, llavors $\det(\vec v_1, \dots , \vec v_n)=\det'(\vec v_1, \dots , \vec v_n)$: si considerem $A$ la matriu formada pels vectors $\vec v_1, \dots, \vec v_n$ per columna, podem aplicar transformacions elementals per columnes fins a tenir, o bé una matriu identitat, o bé una matriu amb l'última columna tot zeros. Pel Lema \ref{lema:det_i_trans_elem}, veiem com es modifica qualsevol aplicació (tant $\det$, com $\det'$) que compleixi els axiomes \textbf{D1}, \textbf{D2}, \textbf{D3} i \textbf{D4}, obtenint que $\det(A)=\lambda\det(\rcef(A))$ (\emph{reduced column echelon form}) i $\det'(A)=\lambda\det'(\rcef(A))$ (amb el mateix $\lambda$), però o bé $\rcef(A)=\1_n$ (i tenim $\det(\rcef(A))=1=\det'(\rcef(A))$ per \textbf{D4}), o bé $\rcef(A)$ té una columna tot zeros i per la Observació \ref{obs:det0}, $\det(A)=0=\det'(A)$.

    Per tant, com que $\det$ i $\det'$ estan determinats pels canvis elementals per columnes i pel seu valor a la identitat o a una matriu amb una columna tot zeros, han de valer el mateix.
\end{proof}

\begin{exemple}
	Calculem el determinant d'$A$, on:$$
	A=\begin{pmatrix}
	1 & 2 & 6 \\ 0 & -1 & -8 \\ 5 & 6 & 0
	\end{pmatrix}
	$$
	fent transformacions elementals:
	\begin{align*}
	\begin{vmatrix}
	1 & 2 & 6 \\ 0 & -1 & -8 \\ 5 & 6 & 0
	\end{vmatrix} & =
	\begin{vmatrix}
	1 & 2 & 6 \\ 0 & -1 & -8 \\ 0 & -4 & -30
	\end{vmatrix}= 
	-\begin{vmatrix}
	1 & 2 & 6 \\ 0 & 1 & 8 \\ 0 & -4 & -30
	\end{vmatrix}= \\ &
	= -\begin{vmatrix}
	1 & 0 & 6 \\ 0 & 1 & 8 \\ 0 & 0 & 2
	\end{vmatrix}=
	-2\begin{vmatrix}
	1 & 0 & 6 \\ 0 & 1 & 8 \\ 0 & 0 & 1
	\end{vmatrix}= -2 \det(\1_3)=-2\,.
	\end{align*}
\end{exemple}

Amb tot això, el que no hem demostrat és que existeixi una aplicació $\det$ que tingui les propietats \textbf{D1}, \textbf{D2}, \textbf{D3} i \textbf{D4}.

Per demostrar la existència, el que farem és construir-la explícitament. 

Considerem $A\in M_n(\K)$ i el producte de $n$ coeficients de la matriu $a_{i_1j_1} \cdots a_{i_nj_n}$ tals que no hi hagi dos coeficients d'una mateixa columna, ni d'una mateixa fila. Dit d'una altra manera, com que hi ha $n$ files i $n$ columnes,considerem $n$ parelles $\{(i_1,j_1),\dots,(i_n,j_n)\}$ que compleixin la igualtat de conjunts:
\begin{equation}\label{eq:patro}
\{i_1, \dots, i_n\}=\{1, \dots, n\} = \{j_1, \dots, j_n\} \,.
\end{equation}
\begin{definicio}\label{def:patro}
Anomenem un \emph{patró}\index{patró} d'$n$ elements a un conjunt $\calp=\{(i_1,j_1),\dots,(i_n,j_n)\}\subset \{1,\dots,n\}^2$ que compleixi que $i_k\neq i_l$ i $j_k\neq j_l$ si $k\neq l$, o, equivalentment, que compleixi l'Equació \eqref{eq:patro}.\\
Donat un patró d'$n$ elements $\calp$ i una matriu $A\in M_n(\K)$, definim l'element $a_\calp$ com el producte:
$$
a_\calp=\prod_{(i,j)\in\calp} a_{ij} .
$$
\end{definicio}
\begin{exemple}
    En el cas de les matrius $3\times 3$, tenim $6$ patrons possibles i ens donen els productes:
    $$
    \begin{array}{ccccc}
\begin{pmatrix} \boxed{a_{11}} & a_{12} & a_{13}\\ a_{21} & \boxed{a_{22}} & a_{23}\\ a_{31} & a_{32} & \boxed{a_{33}} \end{pmatrix} & \Longleftrightarrow & \calp=\{(1,1),(2,2),(3,3)\} &   \Longleftrightarrow & a_\calp=a_{11}a_{22}a_{33} \\
\begin{pmatrix} \boxed{a_{11}} & a_{12} & a_{13}\\ a_{21} & a_{22} & \boxed{a_{23}}\\ a_{31} & \boxed{a_{32}} & a_{33} \end{pmatrix} & \Longleftrightarrow & \calp=\{(1,1),(2,3),(3,2)\} &   \Longleftrightarrow & a_\calp=a_{11}a_{23}a_{32} \\
\begin{pmatrix} a_{11} & \boxed{a_{12}} & a_{13}\\ \boxed{a_{21}} & a_{22} & a_{23}\\ a_{31} & a_{32} & \boxed{a_{33}} \end{pmatrix} & \Longleftrightarrow & \calp=\{(1,2),(2,1),(3,3)\} &   \Longleftrightarrow & a_\calp=a_{12}a_{21}a_{33} \\
\begin{pmatrix} a_{11} & \boxed{a_{12}} & a_{13}\\ a_{21} & a_{22} & \boxed{a_{23}}\\ \boxed{a_{31}} & a_{32} & a_{33} \end{pmatrix} & \Longleftrightarrow & \calp=\{(1,2),(2,3),(3,1)\} &   \Longleftrightarrow & a_\calp=a_{12}a_{23}a_{31} \\
\begin{pmatrix} a_{11} & a_{12} & \boxed{a_{13}}\\ \boxed{a_{21}} & a_{22} & a_{23}\\ a_{31} & \boxed{a_{32}} & a_{33} \end{pmatrix} & \Longleftrightarrow & \calp=\{(1,3),(2,1),(3,2)\} &   \Longleftrightarrow & a_\calp=a_{13}a_{21}a_{32} \\
\begin{pmatrix} a_{11} & a_{12} & \boxed{a_{13}}\\ a_{21} & \boxed{a_{22}} & a_{23}\\ \boxed{a_{31}} & a_{32} & a_{33} \end{pmatrix} & \Longleftrightarrow & \calp=\{(1,3),(2,2),(3,1)\} &   \Longleftrightarrow & a_\calp=a_{13}a_{22}a_{31} 
    \end{array}
    $$
\end{exemple}
I podeu veure que coincideixen amb els sumands de l'Equació \eqref{eq:det3}. Ara tant sols cal decidir si cada element suma o resta, i per això necessitem el \emph{signe d'un patró}:
\begin{definicio}\label{def:signepatro}
Fixat un enter positiu $n$ i un patró $\calp=\{(i_1,j_1),\dots,(i_n,j_n)\}\subset \{1,\dots,n\}^2$, definim $\sign(\calp)$, el \emph{signe de $\calp$}\index{signe d'un patró}, com $(-1)^\epsilon$, on $\epsilon$ és el nombre de parelles $[(i,j),(i',j')]\in\calp$ tals que $i<i'$ i $j>j'$.
\end{definicio}
\begin{exemple}
    En el cas $3\times 3$, els patrons tenen el signe següent:
    $$
    \begin{array}{|c|c|c|c|}
    \hline \text{Patró} & \text{Parelles} & \epsilon & \text{Signe} \\ \hline
    \{(1,1),(2,2),(3,3)\} & & 0 & (-1)^0=1 \\
    \{(1,1),(2,3),(3,2)\} & \{[(2,3),(3,2)]\} & 1 & (-1)^1=-1 \\
    \{(1,2),(2,1),(3,3)\} & \{[(1,2),(2,1)]\} & 1 & (-1)^1=-1 \\
    \{(1,2),(2,3),(3,1)\} & \{[(1,2),(3,1)],[(2,3),(3,1)]\} & 2 & (-1)^2=1 \\
    \{(1,3),(2,1),(3,2)\} & \{[(1,3),(2,1)],[(1,3),(3,2)\} & 2 & (-1)^2=1 \\
    \{(1,3),(2,2),(3,1)\} & \{[(1,3),(2,2)],[(1,3),(3,1)],[(2,2),(3,1)]\} & 3 & (-1)^3=-1 \\ \hline
    \end{array}
    $$
\end{exemple}
Els patrons tenen les propietats següents:
\begin{lema}\label{lema:permutacions}
Considerem $A\in M_n(\K)$, amb $n$ fixada.
\begin{enumerate}[\rm (a)]
    \item Cada patró $\calp$ es correspon amb una aplicació bijectiva $\sigma\colon \{1,2,\dots,n\} \to \{1,2,\dots,n\}$ (s'anomena \emph{permutació}\index{permutació}).
    \item Hi ha $n!$ (factorial d'$n$) patrons diferents a $A$.
    \item Definim el \emph{signe d'una permutació}\index{permutació!signe} $\sigma\colon \{1,\dots,n\}\to\{1,\dots,n\}$ com
    $$
    \sign(\sigma)=\prod_{i<j} \frac{\sigma(j)-\sigma(i)}{j-i} .
    $$
    Si $\sigma$ és la permutació corresponent a $\calp$, llavors $\sign(\sigma)=\sign(\calp)$.
    \item Si $\Id\colon \{1,\dots,n\}\to\{1,\dots,n\}$ és l'aplicació $\Id(i)=i$ (identitat), llavors $\sign(\Id)=1$.
    \item Si $\sigma$ i $\tau$ són dues permutacions de $\{1,\dots, n\}$, llavors 
    $$\sign(\sigma\circ\tau)=\sign(\sigma)\sign(\tau)$$.
    \item $\sign(\sigma)=\sign(\sigma^{-1})$.
    \item Si $\tau_{k\ell}$ correspon a la permutació:
    $$
    \tau_{k\ell}(m)=\left\{\begin{array}{ll} m & \text{si $m\not\in\{k,\ell\}$}\\ \ell & \text{si $m=k$} \\ k & \text{si $m=\ell$} \end{array} \right. 
    $$
    amb $k\neq \ell$, llavors $\sign(\tau_{k\ell})=-1$.
\end{enumerate}
\end{lema}
\begin{proof}
    Els patró $\calp$ té $n$ parelles $(i,j)$ on, si mirem tant sols la primera coordenada, hi ha exactament un $1$, un $2$, \ldots i un $n$. Definim $\sigma(i)=j$ si $(i,j)\in\calp$. Aquesta aplicació és injectiva i exhaustiva perquè a la segona coordenada de les parelles de $\calp$ també hi ha un únic $1$, un únic $2$, \ldots i un únic $n$.\\
    Si tenim una aplicació bijectiva $\sigma\colon \{1,2,\dots,n\} \to \{1,2,\dots,n\}$, definim els patró corresponent com $\calp=\{(1,\sigma(1)), \ldots, (n,\sigma(n))\}$.
    
    Per demostrar (b) comptem quantes aplicacions bijectives $\sigma$ hi ha de $\{1,2,\dots, n\}$ en ell mateix: $\sigma(1)$ pot ser qualsevol element de $\{1,2,\dots, n\}$, per tant en podem escollir $n$; 
    $\sigma(2)$ pot ser qualsevol element de $\{1,2,\dots, n\}$ excepte $\sigma(1)$, per tant en podem escollir $n-1$;
    iterant aquest procediment, tindrem $n(n-1)(n-2) \cdots 2\cdot 1$ aplicacions bijectives, i aquesta és la definició de factorial d'$n$.
    
    A l'expressió de (c) veiem que tant al denominador hi ha el producte $(2-1)(3-1)\cdots (n-1)(3-2) \cdots$, mentre que al numerador hi ha els mateixos factors, on es canvien de signe els que compleixen $\sigma(i)>\sigma(j)$ amb $i<j$, pel que el quocient serà $\pm 1$, i el signe ve determinat pel nombre de vegades que passa $\sigma(i)>\sigma(j)$ amb $i<j$, que és la definició de signe d'un patró.
    
    Calculem el signe de la identitat per demostrar (d):
    $$
    \sign(\Id)=\prod_{i<j} \frac{\Id(j)-\Id(i)}{j-i}=\prod_{i<j} \frac{j-i}{j-i}=1 .
    $$
    
    La demostració d'(e) ve de considerar:
    \begin{align*}
    \sign(\sigma\circ\tau) & =\prod_{i<j} \frac{\sigma(\tau(j))-\sigma(\tau(i))}{j-i}  =
    \prod_{i<j} \frac{\sigma(\tau(j))-\sigma(\tau(i))}{\tau(j)-\tau(i)}\frac{\tau(j)-\tau(i)}{j-i}=\\
     & =\prod_{i<j} \frac{\sigma(\tau(j))-\sigma(\tau(i))}{\tau(j)-\tau(i)}\prod_{i<j}\frac{\tau(j)-\tau(i)}{j-i}= \sign(\sigma)\sign(\tau),
    \end{align*}
    on observem que:
    $$
    \prod_{i<j}\frac{\sigma(j)-\sigma(i)}{j-i}=\prod_{i<j} \frac{\sigma(\tau(j))-\sigma(\tau(i))}{\tau(j)-\tau(i)}
    $$
    ja que hi ha els mateixos factors al numerador i denominador, i si un factor del numerador ha canviat de signe, el corresponent factor del denominador també.
    
    La demostració d'(f) es dedueix de que $1=\sign(\Id)=\sign(\sigma\circ \sigma^{-1})=\sign(\sigma)\sign(\sigma^{-1})$, i que el signe de qualsevol permutació és $\pm1$.

    Finalment, la demostració de (g) és molt semblant a la de (d): primer considerem $k<\ell$ (com que $\tau_{k\ell}=\tau_{\ell}$, si cal, les intercanviem). Llavors el signe de $\tau_{k\ell}$ serà com el de $\Id$, però hi haurà un factor diferent a quan $(i,j)=(k,\ell)$, que sortirà $\frac{k-\ell}{\ell-k}=-1$, per tant $\sign(\tau_{k\ell})=-1$. 
\end{proof}

Ara ja podem definir el determinant d'una matriu:
\begin{definicio}\label{def:determinant}
Sigui $A\in M_n(\K)$ una matriu quadrada. Considerem $P_n$ el conjunt de tots els patrons d'$n$ elements ($P_n$ té $n!$ elements) i per cada $\calp\in P_n$, considerem $a_\calp$ com a la Definició \ref{def:patro}. Definim el \emph{determinant d'$A$}\index{determinant} com:
\begin{equation} \label{eq:def_det}
    \det(A)=\sum_{\calp \in P_n} \sign(\calp) a_\calp \,.
\end{equation}
\end{definicio}

\begin{teorema}
El determinant de la Definició \ref{def:determinant} compleix les propietats \textbf{D1}, \textbf{D2}, \textbf{D3} i \textbf{D4}.
\end{teorema}
\begin{proof}
    Fixem $A\in M_n(\K)$ i $a_{ij}$ els seus coeficients.
    
    \textbf{D1}: si fixem una columna $j$, cada sumand de l'Equació \eqref{eq:def_det} té exactament un coeficient $a_{ij}$. Si el substituïm per $\lambda a_{ij}$, tots els sumands de l'Equació \eqref{eq:def_det} queden multiplicats per $\lambda$, pel que es compleix \textbf{D1}.
    
    \textbf{D2}: si la columna $j$ es pot escriure com la suma de dues columnes: $a_{ij}=a'_{ij}+a''_{ij}$, com que a l'Equació \eqref{eq:def_det} cada sumand té exactament un d'aquests coeficients, podem separar la suma, obtenint \textbf{D2}.
    
    \textbf{D3}: Si tenim la columna $j$ i la columna $k$ d'$A$ que són iguals (amb $j\neq k$), tenim que cada sumand de l'Equació \eqref{eq:def_det} apareix dues vegades, una pel patró $\calp_1$ i $\calp_2$. Hem de veure que apareix amb signe diferent, i llavors la suma serà zero. Si $\sigma_1$ és la permutació que correspon al patró $\calp_1$ i $\sigma_2$ la que correspon al patró $\calp_2$, llavors $\sigma_1=\tau_{jk}\circ\sigma_2$ i, pel Lema \ref{lema:permutacions}, tenen signe diferent. 
    
    \textbf{D4}: l'únic patró $\calp$ tal que $(\1_n)_\calp\neq 0$ és el patró $\{(1,1),(2,2),\dots, (n,n)\}$, i és un producte d'uns amb signe positiu.
\end{proof}

Ara ja podem demostrar una de les propietats principals del determinant:

Vegem ara més propietats dels determinants:
\begin{proposicio}
	Si $A, B\in M_{n\times n}(\K)$, llavors:
	\begin{enumerate}[\rm (a)]
		\item $A$ és invertible si i només si $\det(A)\neq 0$.
		\item $\det(A)\neq 0$ si i només si les files (i les columnes) d'$A$ són linealment independents.
		\item $\det(A)=\det(A^T)$, on $A^T$ és la transposada d'$A$.
		\item $\det(AB)=\det(A)\det(B)$.
	\end{enumerate}
\end{proposicio}
\begin{proof}
    Per demostrar (a), recordem que una matriu és invertible si i només si podem aconseguir la identitat mitjançant transformacions elementals. En aquest cas, cada transformació modifica el determinant canviant-li el signe o multiplicant per un $\lambda\neq 0$. Llavors:
	
	Si $A$ és invertible, $\det(A)=\lambda \det(\1_n)=\lambda \neq 0$.
	
	Si $A$ no és invertible, $\det(A)=\lambda \det(B)$, on $B$ és una matriu amb l'última fila tot zeros, per tant $\det(B)=0$ (cada sumand del determinant a l'Equació \eqref{eq:def_det} té un coeficient de l'última fila), d'on es dedueix que $\det(A)=0$.
	
	Per demostrar (b), el raonament de files (o columnes) linealment independents és el mateix, tenint en compte que les files (o columnes) són linealment independents si i només si la matriu $A$ és equivalent a la identitat.
	
	Per demostrar (c), observem que si $B=A^T$, per l'Equació \eqref{eq:def_det}, $\det(A)$ i $\det(B)$ tenen els mateixos sumands, pel que cal veure que els signes dels patrons $\calp$ i $\calp'$ corresponents a coeficients iguals $a_\calp$ i $b_{\calp'}$, són els mateixos: observem que si $a_\calp$ és el coeficient corresponent a $\calp$, que és un patró que correspon a una permutació $\sigma$ (veure Lema \ref{lema:permutacions}), llavors, si $\calp'$ és el patró corresponent a $\sigma^{-1}$, tenim que $a_\calp=b_{\calp'}$, i amb el mateix signe, ja que pel Lema \ref{lema:permutacions}, $\sign(\calp)=\sign(\sigma)=\sign(\sigma^{-1})=\sign(\calp')$.

    Falta demostrar la fórmula del producte de determinants $\det(AB)=\det(A)\det(B)$: comencem amb la matriu $A$ i li fem les transformacions elementals necessàries per obtenir $\rref(A)$. Aquestes transformacions elementals aniran modificant el determinant, i obtindrem $\det(A)=\lambda\det(\rref(A))$, on:
    \begin{itemize}
        \item Aquestes transformacions elementals es corresponen amb multiplicar per una matriu invertible $P$: $\rref(A)=PA$.
        \item $\lambda$ només depèn de les transformacions elementals que hem fet, per tant, si agafem $C\in M_n(\K)$ i hi fem les mateixes transformacions que hem fet a $A$ i obtenim $C'=PC$, tindrem $\det(C)=\lambda \det(C')$.
        \item Si $A$ és invertible, llavors $\rref(A)= PA=\1_n$ i $\det(A)=\lambda\det(\1_n)=\lambda$.
    \end{itemize}
    Considerem primer el cas en que $A$ no és invertible: llavors $\det(A)=0$ i les $n$ columnes d'$A$ no són linealment independents. Com que les columnes d'$AB$ són combinació lineal de les de $A$ i també en té $n$, tenim que les columnes de $AB$ tampoc són linealment independents i per tant $\det(AB)=0$ i en particular $\det(AB)=\det(A)\det(B)$.
    
    Suposem ara que $A$ és invertible, per tant $\rref(A)=PA=\1_n$. Considerem ara $C=AB$ i fem les mateixes transformacions elementals que hem fet a $A$ per obtenir $\rref(A)$. Amb els raonaments que hem vist, tenim:
    $$
    \det(AB)=\det(C)=\lambda \det(C')=\lambda \det(PAB)=\lambda \det(\1_n B)=\det(A)\det(B).
    $$
\end{proof}


 Ara veurem una manera inductiva de calcular el determinant. També es podia haver definit com veurem ara, i s'hauria de demostrar que compleix les propietats \textbf{D1}, \textbf{D2}, \textbf{D3} i \textbf{D4}. Per això, necessitem una notació que utilitzarem a aquesta secció (es pot confondre amb una de les notacions utilitzades a les seccions anteriors, fixeu-vos amb la $c$ del superíndex).
\begin{notacio}
	Si $A\in M_{n}(\K)$, notem per $A^c_{ij}\in M_{(n-1)}(\K)$ la matriu que resulta d'eliminar la fila $i$ i la columna $j$ d'$A$. 
\end{notacio}
%El determinant és una aplicació que a cada $A\in M_{n}(\K)$ li assigna un escalar $\det(A)$ (o bé $|A|)\in \K$. El podem calcular de manera recurrent com:
%\begin{enumerate}
%	\item En el cas de matrius $1\times 1$ $A=(a)$, tenim $\det(A)=a$.
%	\item Si tenim una matriu $A \in M_{n\times n}(A)$, calculem el determinant com:
%	$$
%	\det(A)=\sum_{j=1}^n (-1)^{j+1} a_{1j} \det(A^c_{1j}) \,.
%	$$
%\end{enumerate} 
%\begin{exemple}
%	$$
%	\det\begin{pmatrix}
%	a & b \\ c & d
%	\end{pmatrix} =
%	\begin{vmatrix}
%	a & b \\ c & d
%	\end{vmatrix}=ad-bc \,.
%	$$
%\end{exemple}
\begin{proposicio}\label{prop:defdet}
	Si $A\in M_{n\times n}(\K)$, amb $n\geq2$, podem desenvolupar el determinant per qualsevol fila o columna segons les fórmules següents:
	$$
	\begin{array}{ll}
	\det(A)=\sum_{j=1}^n (-1)^{i+j} a_{ij} \det(A^c_{ij}) & \text{(si desenvolupem per la fila $i$)},\\[3mm]
	\det(A)=\sum_{i=1}^n (-1)^{i+j} a_{ij} \det(A^c_{ij}) & \text{(si desenvolupem per la columna $j$)}.
	\end{array}
	$$
\end{proposicio}
\begin{proof}
	Per a demostrar això, mirem com són tots els sumands i els comparem amb els de l'Equació \eqref{eq:def_det}: l'hem calculat de manera recursiva, i cada cop que fem una iteració anem esborrant la fila i columna corresponent a aquell coeficient. Per tant, hi haurà un coeficient de la primera fila, un altre de la segona, \ldots, de tal manera que cada cop agafem una columna diferent, i per tant tindrem el sumand:
	\begin{equation}\label{eq:sumanddet}
	(-1)^\epsilon a_{1j_1} a_{2j_2} \cdots a_{nj_n}
	\end{equation}
	amb $j_k\neq j_l$ si $k\neq l$, i el signe ve determinat per la paritat de $\epsilon$, que es pot veure que és la mateixa que la de la permutació, obtenint el mateix que a l'Equació \eqref{eq:def_det}
	
	Aquesta expressió no depèn de per quina fila o columna desenvolupem, pel que el resultat final serà el mateix.
\end{proof}

\begin{exemple}
	Calculem el determinant d'$A$, on:$$
	A=\begin{pmatrix}
	1 & 2 & 6 \\ 0 & -1 & -8 \\ 5 & 6 & 0
	\end{pmatrix}
	$$
	desenvolupant per la primera fila:
	\begin{align*}
	\begin{vmatrix}
	1 & 2 & 6 \\ 0 & -1 & -8 \\ 5 & 6 & 0
	\end{vmatrix} & = 1 \begin{vmatrix} -1 & -8 \\ 6 & 0 \end{vmatrix} 
	-2 \begin{vmatrix} 0 & -8 \\ 5 & 0  \end{vmatrix} +
	6 \begin{vmatrix} 0 & -1 \\ 5 & 6  \end{vmatrix} = \\
	 & = (0-(-48))-2(40)+6(5)=48-80+30=-2.
	\end{align*}
\end{exemple}

\begin{exercici}
Si $\calp$ és un patró amb $n$ elements i $A_\calp\in M_n(\K)$ és una matriu que té tots els coeficients zero, excepte els elements $a_{ij}=1$, per a $(i,j)\in\calp$, llavors $\det(A_\calp)=\sign(\calp)$. 
\end{exercici}
\begin{exercici}
Si $A\in M_n(\K)$ és una matriu triangular superior (o inferior), llavors $\det(A)$ és el producte d'elements de la diagonal.
\end{exercici}

%\textbf{[Albert: afegim relació del determinant i el volum?]}

\subsection{Polinomi característic. Valors i vectors propis}
A la motivació d'aquest capítol (\S\ref{subsec:motiv_diag}) hem vist una aplicació lineal $f\colon \R^3\to \R^3$ que escrita en una base $\calb$ o en una altra $\calc$ tenia les expressions següents:
$$
[f]_\calb=\begin{pmatrix}
-1 & 0 & 0 \\ 0 & 2 & 0 \\ 0 & 0 & 0
\end{pmatrix}
\text{ i }
[f]_\calc=\begin{pmatrix}
5 & -15 & -21 \\ -3 & 9 & 13 \\ 3 & -9 & -13
\end{pmatrix}
$$
Veiem que la primera és diagonal, mentre que la segona no. Això vol dir que l'aplicació lineal envia les rectes generades per cada vector de la base $\calb$ a elles mateixes, mentre que les rectes generades pels vectors de la base $\calc$ es converteixen en altres rectes. Si fixem uns eixos de coordenades amb la base $\calb$, podrem veure com $f$ modifica els vectors d'$\R^3$ molt més fàcilment que si fixem els eixos a la base $\calc$. Per tant, estem dient que per la mateixa aplicació lineal $f\colon E \to E$, aprofitant que els coeficients de $[f]_\calb$ depenen de la base $\calb$, buscarem una base $\calb$ tal que la matriu $[f]_\calb$ sigui el més ``\emph{senzilla}'' possible.
\begin{definicio}
Sigui $A \in M_n(\K)$ i $f_A\colon \K^n\to\K^n$ l'aplicació lineal induïda. Diem que \emph{$A$ és diagonalitzable}\index{matriu!diagonalitzable} si es compleix una de les condicions equivalents següents:
\begin{enumerate}[\rm (a)]
    \item $A$ és similar a una matriu diagonal.
    \item Existeix una base $\calb$ de $\K^n$ tal que $[f_A]_\calb$ és diagonal.
    \item Existeix una matriu invertible $S\in M_n(\K)$ tal que el producte $S^{-1}AS$ és diagonal.
\end{enumerate}
\end{definicio}
\begin{exemple}
La matriu $B=\left(\begin{smallmatrix}
5 & -15 & -21 \\ -3 & 9 & 13 \\ 3 & -9 & -13
\end{smallmatrix}\right)$ és diagonalitzable (veure apartat \ref{subsec:motiv_diag}).
\end{exemple}
\begin{observacio}\label{obs:diagonalitzables}
Si una matriu $A$ és diagonalitzable i $\calb=[\vec v_1, \dots, \vec v_n]$ és una base en que $[f_A]_\calb$ és diagonal amb coeficients a la diagonal $\lambda_1, \dots, \lambda_n$, llavors es compleix que $A \vec v_j=\lambda_j \vec v_j$. Per tant, els vectors $\vec v_j$ de la base $\calb$ compleixen que $A\vec v_j$ és un múltiple de $\vec v_j$.
\end{observacio}
Utilitzem aquest fet per veure que no totes les matrius són diagonalitzables: 
\begin{exemple}
La matriu $A=\left(\begin{smallmatrix} 0 & -1 \\ 1 & 0\end{smallmatrix}\right)\in M_2(\R)$, que correspon a una rotació d'angle $\pi/2$, no és diagonalitzable: suposem que sí, i que $\vec v$ és un vector d'una base $\calb$ tal que $[f_A]_\calb$ és diagonal, llavors es complirà que $A\vec v=\lambda \vec v$ per a cert $\lambda\in \R$, però $\vec v$ i $A\vec v$ són perpendiculars per a tot $\vec v \in \R^2$, pel que no pot ser. Una altra manera de veure-ho és que si $\vec v=\left(\begin{smallmatrix} x \\ y \end{smallmatrix}\right)$ s'hauria de complir:
$$
\begin{pmatrix}
0 & -1 \\ 1 & 0 
\end{pmatrix}
\begin{pmatrix}
x \\ y 
\end{pmatrix} =
\begin{pmatrix}
\lambda x \\ \lambda y 
\end{pmatrix}
$$
I queda el sistema homogeni:
\begin{align*}
    -\lambda x - y =0 \\
    x - \lambda y=0
\end{align*}
Com que estem treballant a $\R$, aquest sistema té rang $2$ per qualsevol $\lambda\in\R$, i per tant l'única solució és $\left(\begin{smallmatrix} x \\ y \end{smallmatrix}\right)=\left(\begin{smallmatrix} 0 \\ 0 \end{smallmatrix}\right)$, que no pot formar part de cap base.
\end{exemple}

Veiem, doncs, que per saber si una matriu $A\in M_n(\K)$ és diagonalitzable, hem de veure si existeix una base $\calb=[\vec v_1, \dots, \vec v_n]$ i escalars $\lambda_j\in\K$ tals que $A\vec v_j=\lambda_j \vec v_j$. Posem nom a aquests escalars i vectors:
\begin{definicio}\label{def:vapivep}
Donada una matriu $A\in M_n(\K)$, diem que un vector no nul $\vec v\in \K^n$ és un \emph{vector propi}\index{vector!propi}\index{propi!vector} de \emph{valor propi}\index{valor propi}\index{propi!valor} $\lambda \in \K$ si $A\vec v=\lambda \vec v$.\\
Els elements del conjunt format pels $\lambda\in \K$ tals que existeix un vector $\vec v\in\K^n$ no nul tal que $A\vec v=\lambda \vec v$ s'anomenen \emph{valors propis d'$A$}.
\end{definicio}
\begin{exemple}
Analitzem què és un vector propi de valor propi $0$ d'una matriu $A\in M_n(\K)$: serà $\vec v\neq \vec 0$ tal que $A\vec v=\vec 0$, per tant, serà un vector de $\Ker(f_A)$ (les solucions del sistema homogeni amb matriu associada $A$). Deduïm que $0$ és un valor propi d'$A$ si i només si $\Ker(f_A)\neq \{\vec 0\}$, si i només si $\det(A)=0$.
\end{exemple}
\begin{exemple}\label{exem:nuclivap0}
Quins són els valors propis i els vectors propis de $\1_n$? Per a tot $\vec v\in\K^n$ es compleix que $\1_n\vec v=\vec v$, per tant, tot vector no nul és vector propi de valor propi $1$.
\end{exemple}
\begin{exemple}
Considerem la matriu de la reflexió a $\R^2$ respecte la recta $r$ que passa per l'origen amb vector director $\left(\begin{smallmatrix}1\\1\end{smallmatrix}\right)$. Segons la Proposició \ref{prop:reflexio}, correspon a la matriu:
$$
\refl{}=\begin{pmatrix}
0 & 1 \\ 1 & 0
\end{pmatrix}
$$
Per definició de la reflexió, els vectors $\vec v$ de la recta $r$ compliran $A\vec v=\vec v$, per tant són valors propis de vector propi $1$: $A\left(\begin{smallmatrix}1\\1\end{smallmatrix}\right)=\left(\begin{smallmatrix}1\\1\end{smallmatrix}\right)$.

En canvi, els vectors $\vec w$ perpendiculars a $r$ (els generats per $\left(\begin{smallmatrix}1\\-1\end{smallmatrix}\right)$), compleixen que $A\vec w=-\vec w$, per tant, són vectors propis de valor propi $-1$.

Això vol dir que, per a $\calb=[\left(\begin{smallmatrix}1\\1\end{smallmatrix}\right),\left(\begin{smallmatrix}1\\-1\end{smallmatrix}\right)]$, 
$$
[\refl]_\calb=\begin{pmatrix}
1 & 0 \\ 0 & -1
\end{pmatrix}.
$$
\end{exemple}
Mirant l'Exemple \ref{exem:nuclivap0}, podem deduir quan un valor $\lambda\in\K$ és un valor propi d'una matriu:
\begin{teorema}\label{teo:vapsA}
Fixada una matriu $A\in M_n(\K)$, $\lambda\in\K$ els un valor propi de $A$ si, i només si, $\det(A-\lambda\1_n)=0$.
\end{teorema}
\begin{proof}
$\lambda$ és un valor propi d'$A$ si i només si existeix $\vec v\neq \vec 0$ tal que $A\vec v=\lambda \vec v$, i aquesta igualtat és equivalent a que $(A-\lambda \1_n)\vec v=\vec 0$. Per tant, és equivalent a que el sistema homogeni donat per la matriu $A-\lambda\1_n$ tingui solució diferent de $\vec 0$, i això és equivalent a que $\det(A-\lambda\1_n)=0$.
\end{proof}

Això ens porta a la definició següent:
\begin{definicio}
Si $A\in M_n(\K)$, $\det(A-x \1_n)$ és un polinomi $p_A(x)$ de grau $n$ en la variable $x$ que s'anomena \emph{polinomi característic d'$A$}\index{polinomi característic}.
\end{definicio}

El que hem vist al Teorema \ref{teo:vapsA} és que $\lambda$ és un valor propi d'$A$ si i només si $p_A(\lambda)=0$, on $p_A(x)$ és el polinomi característic d'$A$ en la variable $x$.

\begin{exemple}
Calculem els valors propis de la matriu $B=\left(\begin{smallmatrix}
5 & -15 & -21 \\ -3 & 9 & 13 \\ 3 & -9 & -13
\end{smallmatrix}\right)$: hem de fer el determinant:
$$
0=
\begin{vmatrix}
5-x & -15 & -21 \\ -3 & 9-x & 13 \\ 3 & -9 & -13-x
\end{vmatrix} = -x^3 -x^2+2x=-x(x+1)(x-2)
$$
i per tant els valors propis són $\{0,-1,2\}$ (tal i com hem vist a l'apartat \ref{subsec:motiv_diag}).
\end{exemple}
\begin{observacio}
El Teorema \ref{teo:vapsA} redueix el problema de trobar els valors propis d'una matriu a un de trobar les arrels d'un polinomi de grau $n$. Per a $n\leq 4$ existeixen fórmules explícites per trobar expressions d'aquestes arrels, mentre que per a $n\geq 5$ es pot demostrar que no n'hi ha (de fòrmula explícita general).
\end{observacio}

\begin{exercici}
Calculeu el polinomi característic d'una matriu $2\times 2$ general $\left(\begin{smallmatrix}a&b\\c&d\end{smallmatrix} \right)$.
\end{exercici}

Hi ha casos, però en que és fàcil calcular les arrels del polinomi característic:
\begin{lema}\label{lema:vap_triangsup}
Si $A\in M_n(\K)$ és una matriu triangular superior (o inferior), els valors propis d'$A$ són els valors de la diagonal d'$A$ (els elements $a_{ii}$).
\end{lema}
\begin{proof}
Podem calcular l'expressió $\det(A-x\1_n)$ desenvolupant per l'última fila i anar tirant enrera, i tindrem:
$$
\det(A-x \1_n)=(a_{11}-x)(a_{22}-x)\cdots (a_{nn}-x).
$$
I aquesta expressió s'anul·la pels valors $x=a_{ii}$.
\end{proof}

\begin{exemple}\label{exem:mat2101}
Estudiem la diagonalització de la matriu
$$
A=\begin{pmatrix}
2 &  1 \\ 0 & 1
\end{pmatrix}.
$$
Pel Lema \ref{lema:vap_triangsup}, com que $A$ és triangular superior, els únics possibles valors propis són els elements $2$ i $1$.

Un vector propi de valor propi $2$ és un vector $\vec v=\left(\begin{smallmatrix} x \\ y \end{smallmatrix}\right)$ tal que:
$$
\begin{pmatrix}
2 &  1 \\ 0 & 1
\end{pmatrix}
\begin{pmatrix} x \\ y \end{pmatrix} =
\begin{pmatrix} 2x \\ 2y \end{pmatrix}
$$
I per tant queda només l'equació $y  = 0$. Per tant la solució és $\vec v= x\left(\begin{smallmatrix} 1 \\ 0 \end{smallmatrix}\right)$.

Un vector propi de valor propi $1$ és un vector $\vec v=\left(\begin{smallmatrix} x \\ y \end{smallmatrix}\right)$ tal que:
$$
\begin{pmatrix}
2 &  1 \\ 0 & 1
\end{pmatrix}
\begin{pmatrix} x \\ y \end{pmatrix} =
\begin{pmatrix} x \\ y \end{pmatrix}
$$
i queda l'equació $x+y=0$, per tant la solució és $\vec v=x\left(\begin{smallmatrix} 1 \\ -1 \end{smallmatrix}\right)$.

Per tant, a la base $\calb=[\left(\begin{smallmatrix} 1 \\ 0 \end{smallmatrix}\right),\left(\begin{smallmatrix} 1 \\ -1 \end{smallmatrix}\right)]$, tenim:
$$
[f_A]_\calb=\begin{pmatrix}
2 & 0 \\ 0 & 1
\end{pmatrix}.
$$
\end{exemple}
\begin{exemple}\label{exem:mat1101}
Estudiem la diagonalització de la matriu
$$
A=\begin{pmatrix}
1 &  1 \\ 0 & 1
\end{pmatrix}.
$$
Pel Lema \ref{lema:vap_triangsup}, com que $A$ és triangular superior, l'únic valor propi possible és l'$1$.

Si fos diagonalitzable, $A$ hauria de ser similar a $\1_2$, la matriu identitat $2\times 2$, per tant, hauria d'existir $S$ una matriu invertible tal que $A=S\cdot\1_2\cdot S^{-1}$, però $S\cdot \1_2 \cdot S^{-1}=\1_2$, pel que $A$ no és diagonalitzable.
\end{exemple}

El fet de que els valors propis d'$A$ siguin les arrels del polinomi característic d'$A$ dóna una limitació del nombre de valors propis diferents que pot tenir $A$:
\begin{lema}
Si $A\in M_n(\K)$, el nombre de valors propis d'$A$ és com a molt $n$.
\end{lema}
\begin{proof}
Els valors propis d'$A$ són els zeros de $p_A(x)$, que és un polinomi de grau $n$. Utilitzem ara que un polinomi de grau $n$ sobre un cos $\K$ pot tenir com a molt $n$ arrels diferents.
\end{proof}

També es poden calcular alguns coeficients del polinomi característic:
\begin{lema}
Si $A\in M_n(\K)$,
$$
p_A(x)=(-1)^{n} x^n + (-1)^{n-1}\Tr(A) x^{n-1} + \cdots + \det(A) ,
$$
on $\Tr(A)$ s'anomena la traça d'$A$\index{matriu!traça} i és la suma dels elements de la diagonal; i els termes que no estan escrits corresponen a $x$, $x^2$, \ldots i $x^{n-2}$.
\end{lema}
\begin{proof}
Quan calculem $p_A(x)=\det(A-x\1_n)$ a partir de l'Equació \eqref{eq:def_det}, l'únic sumand que conté $n$ o $n-1$ factors que contenen una $x$ és el corresponent al patró $(1,1),(2,2),\dots, (n,n)$ (si fem fem el producte d'un patró a $A-x\1_n$ que té un element de fora de la diagonal, com a mínim en té dos fora de la diagonal, pel que el grau en $x$ és menor o igual a $n-2$), i per tant, tenim que:
$$
p_A(x)=(a_{11}-x)(a_{22}-x)\cdots(a_{nn}-x)+\text{polinomi de grau $n-2$ en $x$} ,
$$
i d'aquí veiem que el coeficient de $x^n$ és $(-1)^n$ i el de $x_{n-1}$ és $(-1)^{n-1}(a_{11}+a_{22}+\cdots + a_{nn})$.

Per veure que el terme independent de $p_A(x)$ és el determinant d'$A$, cal utilitzar que el terme independent de $p_A(x)$ és $p_A(0)$, i per tant és $\det(A-0\cdot\1_n)=\det(A)$.
\end{proof}

Demostrem ara que el polinomi característic és el mateix per matrius similars:
\begin{proposicio}
    Si $A$ i $B\in M_n(\K)$ són matrius similars, llavors $p_A(x)=p_B(x)$.
\end{proposicio}
\begin{proof}
Si $A$ i $B$ són matrius similars, existeix $S\in M_n(\K)$ una matriu invertible tal que $S^{-1}AS=B$. Llavors:
\begin{align*}
p_B(x) & =\det(B-x\1_n)=\det(S^{-1}AS-x\1_n)=\det(S^{-1}AS-S^{-1}(x\1)_nS)=\\ & = \det(S^{-1}(A-x\1_n)S)= \det(S^{-1})\det(A-x\1_n)\det(S)=\\
 & = \det(S)^{-1}\det(S)\det(A-x\1_n) = p_A(x) \, .
    \end{align*}
On a l'última igualtat hem utilitzat que $\det(S^{-1})=\det(S)^{-1}$.
\end{proof}
\begin{exercici}
Demostreu que el recíproc no és cert: trobeu dues matrius $A$ i $B$ amb $p_A(x)=p_B(x)$ però tals que $A$ i $B$ no siguin similars.
\end{exercici}

\subsection{Vectors propis associats a un valor propi}
L'objectiu d'aquesta secció és estudiar els vectors propis d'un valor propi donat. Aquestes tenen un nom:
\begin{definicio}
Fixat $\lambda\in\K$, un valor propi d'una matriu $A\in M_n(\K)$, el subespai de vectors propis de valor propi $\lambda$ (més el vector $\vec 0$) s'anomena \emph{subespai propi d'$A$ associat a $\lambda$} i el denotem per $E_\lambda$.
\end{definicio}
En altres paraules:
$$
E_\lambda=\Ker(f_{A-\lambda\1_n}) ,
$$
quedant demostrat que és un subespai vectorial.

Vegem que els subespais propis només s'intersequen al vector $\vec 0$:
\begin{lema}\label{lema:veps_vap_dif_LI}
Fixem $A\in M_n(\K)$ i $\lambda_1\neq\lambda_2$ dos valors propis diferents d'$A$. Llavors
$$ E_{\lambda_1}\cap E_{\lambda_2}=\{\vec 0\}. $$

Més en general, vectors propis de valor propi diferent són linealment independents.
\end{lema}
\begin{proof}
Sigui $\vec v\in E_{\lambda_1}\cap E_{\lambda_2}$, llavors $A\vec v=\lambda_1\vec v$ i $A\vec v=\lambda_2\vec v$, amb $\lambda_1\neq\lambda_2$, i això només pot passar si $\vec v=\vec 0$.

Vegem ara el cas general: siguin $\{\vec v_1, \dots, \vec v_k\}$ vectors propis de valors propis $\{\lambda_1, \dots, \lambda_k\}$ respectivament, amb tots els $\lambda_i$ diferents. Considerem una subconjunt de vectors de $\{\vec v_1, \dots, \vec v_k\}$ que siguin linealment independents i que sigui maximal. Aquest subconjunt tindrà $\ell$ vectors i podem considerar que són els primers (si és necessari, els reordenem). Per tant  $\{\vec v_1, \dots, \vec v_\ell\}\subset\{\vec v_1, \dots, \vec v_k\}$ amb els $\ell$ primers linealment independents. Volem veure $\ell=k$. Suposem que no, llavors $\ell<k$ i existeixen uns únics $\mu_1,\dots, \mu_\ell \in \K$ tals que:
\begin{equation}\label{eq:v_ell+1}
\vec v_{\ell+1}=\mu_1\vec v_1+ \cdots+ \mu_\ell\vec v_\ell.
\end{equation}
També podem considerar que hi ha dues $\mu_i\neq0$: com que  $\vec v_{\ell+1}\neq\vec 0$, com a mínim n'hi ha una. Si només n'hi hagués una, llavors $E_{\lambda_i}\cap E_{\lambda_{\ell+1}}\neq\{\vec 0\}$, contradient el primer apartat d'aquest lema. Suposem doncs (si cal, reordenem), $\mu_1\neq0\neq\mu_2$.

Resumint, la situació és la següent: si $\ell<k$, podem considerar $\vec v_{\ell+1}$ és un vector propi de valor propi $\lambda_{\ell+1}$ que es pot posar com $\vec v_{\ell+1}=\mu_1\vec v_1+ \cdots+ \mu_ell\vec v_\ell$ amb $\mu_1\neq0\neq\mu_2$ i $\vec v_1$ i $\vec v_2$ són vectors propis de valor propi $\lambda_1$ i $\lambda_2$ respectivament, amb $\lambda_1\neq\lambda_2$. Apliquem $A$ a l'Equació \eqref{eq:v_ell+1}:
\begin{align*}
\lambda_{\ell+1}\vec v_{\ell+1} & =A\vec v_{\ell+1}=A(\mu_1\vec v_1+ \cdots+ \mu_ell\vec v_\ell)= \\ & = \lambda_1\mu_1\vec v_1+\lambda_2\mu_2\vec v_2+\dots+\lambda_\ell\mu_\ell\vec v_\ell
\end{align*}
I també:
\begin{align*}
\lambda_{\ell+1}\vec v_{\ell+1} & = \lambda_{\ell+1}\mu_1\vec v_1+\lambda_{\ell+1}\mu_2\vec v_2+\dots+\lambda_{\ell+1}\mu_\ell\vec v_\ell
\end{align*}
Per tant, com que $\{\vec v_1, \dots, \vec v_\ell\}$ són linealment independents, tenim:
$$
\lambda_{\ell+1} \mu_1=\lambda_1\mu_1 \text{ i } \lambda_{\ell+1}\mu_2=\lambda_2\mu_2
$$
amb $\mu_1\neq0\neq\mu_2$, per tant $\lambda_1=\lambda_{\ell+1}=\lambda_2$, contradient que $\lambda_1\neq\lambda_2$.

Per tant, la contradicció ve de suposar $\ell<k$, el que implica $\ell=k$ i els $k$ vectors són linealment independents.
\end{proof}

\begin{exemple}
Si agafem la matriu
$$
A=\begin{pmatrix}2 & 1 \\ 0 & 1\end{pmatrix},
$$
als càlculs de l'Exemple \ref{exem:mat2101} hem vist que els subespais propis són:
$$
E_2=\langle\begin{pmatrix}
1 \\ 0 
\end{pmatrix}\rangle
\text{ i }
E_1=\langle\begin{pmatrix}
1 \\ -1 
\end{pmatrix}\rangle.
$$
A més, com que tenim dos vectors propis linealment independents en un espai de dimensió $2$, tenim una base de vectors propis i la matriu $A$ diagonalitza.
\end{exemple}

\begin{exemple}
Si agafem ara la matriu
$$
A=\begin{pmatrix}1 & 1 \\ 0 & 1\end{pmatrix},
$$
als càlculs de l'Exemple \ref{exem:mat1101} hem vist que l'únic subespai propi és:
$$
E_1=\langle\begin{pmatrix}
1 \\ 0 
\end{pmatrix}\rangle.
$$
En aquest cas, el subespai format pels vectors propis té dimensió $1$, pel que no tenim una base de vectors propis i la matriu $A$ no diagonalitza.
\end{exemple}
Més en general, per saber si una matriu diagonalitza, cal estudiar el polinomi característic i els subespais propis. Per això, ens convé la definició següent:
\begin{definicio}
Considerem $A\in M_n(\K)$ i $\lambda$ un valor propi d'$A$.
\begin{itemize}
    \item Definim la \emph{multiplicitat algebraica de $\lambda$ com a valor propi d'$A$}\index{multiplicitat algebraica} com el valor $m\geq 1$ tal que
    $$
    p_A(x)=(x-\lambda)^m q(x)
    $$
    amb $q(x)$ un polinomi de grau $n-m$ tal que $q(\lambda)\neq 0$. Escriurem $\multalg_A(\lambda)$.
    \item Definim la \emph{multiplicitat geomètrica de $\lambda$ com a valor propi d'$A$}\index{multiplicitat geomètrica} com la dimensió de $E_\lambda$. Escriurem $\multgeom_A(\lambda)$.
\end{itemize}
\end{definicio}
\begin{observacio}
A la definició de $\multalg_A(\lambda)$ utilitzem que si $\lambda$ és un valor propi, llavors, $p(\lambda)=0$. Si ara fem la divisió de polinomis $p(x)$ dividit per $(x-\lambda)$, existeixen polinomis $q_1(x)$ (quocient) i $r_1(x)$ (la resta, que com que ha de ser de grau menor a $1$, ha de ser una constant, i per tant escrivim $r_1$) tals que:
$$p(x)=(x-\lambda)q_1(x)+r.$$
Si avaluem a $\lambda$ ens queda:
$$
0=p(\lambda)=(\lambda-\lambda)q_1(\lambda)+r_1=r_1 ,
$$
i per tant $r_1=0$, obtenint que:
$$
p(x)=(x-\lambda)q_1(x).
$$
Aquest procediment es pot fer un altre cop si $q_1(\lambda)=0$, i tindríem que $p(x)=(x-\lambda)^2q_2(x)$.

Iterem el procediment fins que $q_m(\lambda)\neq0$, no podem continuar el procediment, i definim la multiplicitat algebraica d'aquesta manera.
\end{observacio}
Mirem ara un cas particular:
\begin{proposicio}\label{prop:n_vaps_diferents}
    Si $A \in M_n(\K)$ té $n$ valor propis diferents, llavors:
    \begin{enumerate}[\rm (a)]
        \item Per a cada $\lambda$ valor propi d'$A$, $\multalg_A(\lambda)=\multgeom_A(\lambda)=1$.
        \item La matriu $A$ diagonalitza.
    \end{enumerate}
\end{proposicio}
\begin{proof}
Com que tenim $n$ valors propis diferents, com a mínim tenim $n$ vectors propis de valor propi diferents, i pel Lema \ref{lema:veps_vap_dif_LI}, seran linealment independents a $\K^n$, per tant una base. D'aquí obtenim que $A$ diagonalitza.

Les $n$ multiplicitats $\multalg_A(\lambda)$ i $\multgeom_A(\lambda)$ són com a mínim $1$ i, sumades, com a molt $n$ (la suma de les $\multalg$ és, com a molt, el grau de $p_A(x)$; la suma de les $\multgeom$ és, com a molt, el nombre màxim d'una família de vectors linealment independents a $\K^n$), per tant, han de ser $1$. 
\end{proof}

En general, la situació no serà tant bona i el que tenim és:
\begin{lema}
Si $A\in M_n(\K)$ i $\lambda$ és un valor propi d'$A$, llavors:
\begin{enumerate}[\rm (a)]
    \item Si $\lambda$ és un valor propi d'$A$, llavors $\multgeom(\lambda)\leq \multalg(\lambda)$.
    \item $A$ diagonalitza si i només si 
    \begin{itemize}
      \item $p_A(x)$ es pot escriure com a producte de polinomis de grau $1$
      \begin{equation}\label{eq:p_A_factors_lineals}
      p_A(x)=(-1)^n (x-\lambda_1)^{\multalg_A(\lambda_1)}\cdots(x-\lambda_k)^{\multalg_A(\lambda_k)} 
      \end{equation}
      amb $\lambda_i$ valors propis diferents i
      \item $\multgeom_A(\lambda_i)=\multalg_A(\lambda_i)$ per a tot $i=1,\dots, k$.
    \end{itemize}
\end{enumerate}
\end{lema}
\begin{proof}
Vegem primer (a): sigui $\lambda$ un valor propi d'$A$ i considerem $m=\multgeom(\lambda)$, i per tant $\vec v_1, \dots, \vec v_m$ vectors linealment independents de $\Ker(A-\lambda\1_n)$. Ampliem la família $\vec v_1, \dots, \vec v_m$ fins a tenir una base $\calb$ de $\K^n$. En aquesta base, la matriu $[f_A]_\calb$ tindrà a les primeres $m$ columnes tots els coeficients zero, excepte la diagonal, que valdrà $\lambda$, per tant, $p_A(x)=p_{[f_A]_\calb}=(-1)^n(x-\lambda)^m q(x)$ i per tant $m=\multgeom(A)\leq\multalg(A)$.

Vegem ara (b): suposem primer que $A$ diagonalitza, llavors, la suma de les multiplicitats geomètriques d'$A$ serà $n$:
$$
\multgeom_A(\lambda_1)+\cdots+\multgeom_A(\lambda_k)=n
$$
amb $\lambda_i$ valors propis diferents. Però com que $\multgeom(\lambda)\leq\multalg(\lambda)$, 
$$
n\leq \multalg_A(\lambda_1)+\cdots+\multalg_A(\lambda_n) \leq n
$$
on la segona desigualtat és perquè $p_A(x)$ té grau $n$, per tant, les multiplicitats geomètriques i algebraiques són iguals i el polinomi $p_A(x)$ es pot escriure com a producte de polinomis de grau $1$.

El recíproc es fa desfent els arguments previs: si $p_A(x)$ factoritza com a producte de factors de grau $1$ com a l'Equació \eqref{eq:p_A_factors_lineals}, per a que diagonalitzi, s'ha de complir que trobem una base de vectors propis. Com que per cada $\lambda$ valor propi, per hipòtesis, $\multgeom(\lambda)_A=\multalg_A(\lambda)$, tindrem $n$ vector propis linealment independents, i per tant una base en que $A$ diagonalitza.
\end{proof}
\begin{exemple}
Volem estudiar si la matriu $A$ següent diagonalitza i, si ho fa, en quina base:
$$
A=\left(\begin{array}{rrrr}
55 & 91 & -29 & 50 \\
-27 & -47 & 15 & -24 \\
-33 & -67 & 21 & -26 \\
-33 & -57 & 18 & -29
\end{array}\right).
$$
Calculem primer el polinomi característic:
$$
p_A(x)=\left|\begin{array}{cccc}
55-x & 91 & -29 & 50 \\
-27 & -47-x & 15 & -24 \\
-33 & -67 & 21-x & -26 \\
-33 & -57 & 18 & -29-x
\end{array}\right|=x^4-3x^2+2x=x(x-1)^2(x+2).
$$
Llavors hem de calcular $E_0=\Ker(f_A)$, $E_1=\Ker(f_{A-\1_4})$ i $E_{-2}=\Ker(f_{A+2\cdot\1_4})$. Sense calcular-los explícitament, per comprovar si diagonalitza només cal calcular $\dim(E_1)$. Si $\dim(E_1)=2$, llavors diagonalitza, però si $\dim(E_1)=1$, no diagonalitza. En aquest cas surt $\dim(E_1)=2$ i per tant diagonalitza. Fem els càlculs:
$$
E_0=\langle\begin{pmatrix}4\\-3\\-7\\-3\end{pmatrix}\rangle \text{, }
E_1=\langle\begin{pmatrix}4\\2\\-4\\-3\end{pmatrix},\begin{pmatrix}11\\-5\\-9\\-8\end{pmatrix}\rangle \text{ i }
E_{-2}=\langle\begin{pmatrix}15\\-9\\-16\\-10\end{pmatrix}\rangle
$$
Pel que podem confirmar que diagonalitza (hi ha 4 vectors propis linealment independents).
Llavors es pot comprovar que:
$$D=S^{-1}AS$$ on
$$
D=\left(\begin{array}{rrrr}
0 & 0 & 0 & 0 \\
0 & 1 & 0 & 0 \\
0 & 0 & 1 & 0 \\
0 & 0 & 0 & -2
\end{array}\right)
\text{ i }
S=\left(\begin{array}{rrrr}
4 & 4 & 11 & 15  \\
-3 & 2 & -5 & -9   \\
-7 & -4 & -9 & -16  \\
-3 & -3 & -8 & -10 
\end{array}\right).
$$
\end{exemple}
\begin{exemple}
Considerem ara la matriu:
$$
A=\left(\begin{array}{rrrr}
107 & 151 & -49 & 106 \\
-53 & -77 & 25 & -52 \\
-85 & -127 & 41 & -82 \\
-72 & -102 & 33 & -71
\end{array}\right)
$$
Podem fer els càlculs d'abans, obtenint el mateix polinomi característic:
$$
p_A(x)=\left|\begin{array}{cccc}
107-x & 151 & -49 & 106 \\
-53 & -77-x & 25 & -52 \\
-85 & -127 & 41-x & -82 \\
-72 & -102 & 33 & -71-x
\end{array}\right|=x^4-3x^2+2x=x(x-1)^2(x+2).
$$
En aquest cas, però, tenim que (després de fer els càlculs) $\dim(E_1)=1<2=\multalg_A(1)$, per tant no diagonalitza.
\end{exemple}

\begin{exemple}[La successió de Fibonacci]\index{Fibonacci}
Es defineix la successió de Fibonacci com:
\[
a_1=a_2=1 \text{ i } a_{n}=a_{n-1}+a_{n-2} \text{ per a $n\geq 2$}.
\]
Podem calcular-ne els seus primers termes:
$$
1, 1, 2, 3, 5, 8, 13, 21, 34, \dots
$$
aquesta definició fa que per calcula el terme $n$, haguem de calcular tots els anteriors. L'objectiu és trobar una fórmula que $f(n)$ tal que $a_n=f(n)$.

Considerem els vectors $\smat{a_{n}\\a_{n-1}}$, i veiem que es compleix la relació següent:
\[
\begin{pmatrix}
a_{n}\\a_{n-1}
\end{pmatrix}=
\begin{pmatrix}
1 & 1\\1 & 0
\end{pmatrix}
\begin{pmatrix}
a_{n-1}\\a_{n-2}
\end{pmatrix}=
\begin{pmatrix}
1 & 1\\1 & 0
\end{pmatrix}^2
\begin{pmatrix}
a_{n-2}\\a_{n-3}
\end{pmatrix}= \cdots =
\begin{pmatrix}
1 & 1\\1 & 0
\end{pmatrix}^{n-2}
\begin{pmatrix}
a_{2}\\a_{1}
\end{pmatrix}=
\begin{pmatrix}
1 & 1\\1 & 0
\end{pmatrix}^{n-2}
\begin{pmatrix}
1\\ 1
\end{pmatrix}
\]
\end{exemple}
Per tant, ens convé calcular:
\[ A^{n-2}=
\begin{pmatrix}
1 & 1\\1 & 0
\end{pmatrix}^{n-2}
\]
Si aconseguim escriure $A=SDS^{-1}$ ($D$ matriu diagonal, per tant, la diagonal de $D$ està formada pels valors propis de $A$), tenim que $A^{n-2}=SD^{n-2}S^{-1}$, i, si $D=\smat{\lambda_1 & 0 \\ 0 & \lambda_2}$, llavors:
\[
D^{n-2}=\begin{pmatrix}
\lambda_1^{n-2} & 0 \\ 0 & \lambda_2^{n-2}
\end{pmatrix}
\]
Calculem els valor propis d'$A$. Calculem el polinomi característic:
\[
p(x)=\det(A-x\1_2)=\begin{vmatrix}
1-x & 1 \\ 1 & -x
\end{vmatrix}= x^2-x-1
\]
I els valors propis són les solucions de l'equació $p(x)=0$, per tant:
\[
\lambda_i=\frac{1\pm \sqrt{1+4}}{2}=\frac{1\pm\sqrt{5}}{2}
\]
Per tant, com que tenim dos valors propis diferents, ja sabem que diagonalitza i tenim:
\[
D=
\begin{pmatrix} \lambda_1 & 0 \\ 0  & \lambda_2
\end{pmatrix}=
\begin{pmatrix}
\frac{1+\sqrt{5}}{2} & 0 \\ 0 & \frac{1-\sqrt{5}}{2}
\end{pmatrix}
\]
Per calcular els vectors propis corresponents a cada valor propi, hem de resoldre els sistemes homogenis:
\[
\left.
\begin{array}{rrr}
    (1-\lambda_i) x + & y & =0 \\ x  - & \lambda_i y  &= 0
\end{array} \right\} \Longleftrightarrow \begin{pmatrix}
x \\ y 
\end{pmatrix} \in \langle \begin{pmatrix}
\lambda_i \\ 1
\end{pmatrix} \rangle
\]
%\[
%\left.
%\begin{array}{rrr}
%    \frac{1-\sqrt{5}}{2} x + & y & =0 \\ x  - & \frac{1+\sqrt{5}}{2}y  &= 0
%\end{array} \right\} \Longleftrightarrow \begin{pmatrix}
%x \\ y 
%\end{pmatrix} \in \langle \begin{pmatrix}
%\frac{1+\sqrt{5}}{2} \\ 1
%\end{pmatrix} \rangle \text{ (per a $\lambda_1=\frac{1+\sqrt{5}}{2}$)}
%\]
%\[
%\left.
%\begin{array}{rrr}
%    \frac{1+\sqrt{5}}{2} x + & y & =0 \\ x  - & \frac{1-\sqrt{5}}{2}y  &= 0
%\end{array} \right\} \Longleftrightarrow \begin{pmatrix}
%x \\ y 
%\end{pmatrix} \in \langle \begin{pmatrix}
%\frac{1-\sqrt{5}}{2} \\ 1
%\end{pmatrix} \rangle \text{ (per a $\lambda_2=\frac{1-\sqrt{5}}{2}$)}
%\]
Llavors, tenim que:
\[
S=
\begin{pmatrix}
\lambda_1 & \lambda_2 \\ 1 & 1
\end{pmatrix}
%\begin{pmatrix}
%\frac{1+\sqrt{5}}{2} & \frac{1-\sqrt{5}}{2} \\ 1 & 1
%\end{pmatrix}
\text{ i }
S^{-1}=
\frac{1}{\lambda_1-\lambda_2}\begin{pmatrix}
1 & -\lambda_2 \\
-1 & \lambda_1
\end{pmatrix}
%\frac{1}{\sqrt{5}}\begin{pmatrix}
%1 & \frac{-1+\sqrt{5}}{2} \\
%-1 & \frac{1+\sqrt{5}}{2}
%\end{pmatrix}
\]
I per tant:
\[
A^{n-2}=SD^{n-2}S^{-1}=
\begin{pmatrix}
\lambda_1 & \lambda_2 \\ 1 & 1
\end{pmatrix}
\begin{pmatrix}
\lambda_1^{n-2} & 0 \\
0 & \lambda_2^{n-2}
\end{pmatrix}
\frac{1}{\lambda_1-\lambda_2}\begin{pmatrix}
1 & -\lambda_2 \\
-1 & \lambda_1
\end{pmatrix}
%\begin{pmatrix}
%\frac{1+\sqrt{5}}{2} & \frac{1-\sqrt{5}}{2} \\ 1 & 1
%\end{pmatrix}
%\begin{pmatrix}
%\left(\frac{1+\sqrt{5}}{2}\right)^{n-2} & 0 \\
%0 & \left(\frac{1-\sqrt{5}}{2}\right)^{n-2}
%\end{pmatrix}
%\frac{1}{\sqrt{5}}\begin{pmatrix}
%1 & \frac{-1+\sqrt{5}}{2} \\
%-1 & \frac{1+\sqrt{5}}{2}
%\end{pmatrix}
\]
I si fem els càlculs:
\[
\begin{pmatrix}
a_n\\a_{n-1}
\end{pmatrix}=
A^{n-2} \begin{pmatrix} 1 \\ 1 \end{pmatrix}=
\frac{1}{\lambda_1-\lambda_2} \begin{pmatrix}
\lambda_1^{n-1}-\lambda_2^{n-1} &
\lambda_1^{n-2}-\lambda_2^{n-2} \\
\lambda_1^{n-2}-\lambda_2^{n-2} &
\lambda_1^{n-3}-\lambda_2^{n-3} 
\end{pmatrix}\begin{pmatrix} 1 \\ 1 \end{pmatrix}
%\frac{1}{\sqrt{5}} \begin{pmatrix}
%\left(\frac{1+\sqrt{5}}{2}\right)^{n-1}-\left(\frac{1-\sqrt{5}}{2}\right)^{n-1} &
%\left(\frac{1+\sqrt{5}}{2}\right)^{n-2}-\left(\frac{1-\sqrt{5}}{2}\right)^{n-2} \\
%\left(\frac{1+\sqrt{5}}{2}\right)^{n-2}-\left(\frac{1-\sqrt{5}}{2}\right)^{n-2} &
%\left(\frac{1+\sqrt{5}}{2}\right)^{n-3}-\left(\frac{1-\sqrt{5}}{2}\right)^{n-3} 
%\end{pmatrix}
\]
Llavors, després de simplificar:
\[
a_n= \frac{1}{\lambda_1-\lambda_2}(\lambda_1^n-\lambda_2^n)=\frac{1}{\sqrt{5}}\left(
\left(\frac{1+\sqrt{5}}{2}\right)^n-\left(\frac{1-\sqrt{5}}{2}\right)^n
\right).
%=\frac{1}{\sqrt{5}} \left(\left(\frac{1+\sqrt{5}}{2}\right)^{n-1}-\left(\frac{1-\sqrt{5}}{2}\right)^{n-1} +
%\left(\frac{1+\sqrt{5}}{2}\right)^{n-2}-\left(\frac{1-\sqrt{5}}{2}\right)^{n-2}\right) 
\]
%\subsection{Polinomi mínim}
%\textbf{[Albert: fem aquest apartat o no?]}

\subsection{Matrius sobre \texorpdfstring{$\R$}{R}}
En aquest apartat, suposarem donada una matriu $A\in M_n(\R)$. Una conseqüència del teorema fonamental de l'àlgebra és que $p_A(x)$ descomposa en producte de factors de grau $1$ i $2$. Si hi ha algun factor de grau $2$, aleshores $A$ no diagonalitza sobre $\R$, tal i com hem vist.

\begin{exemple}
Considerem la matriu $R_{a,b}=\begin{pmatrix}a&-b\\b&a\end{pmatrix}$, on $a, b\in\R$ i $b\neq 0$. El seu polinomi característic és
\[
p_A(x) = x^2-2ax +a^2+b^2 = (x-(a+ib))(x-(a-ib)).
\]
Escrivim $\lambda^+= a+bi$, i $\lambda^- = a-bi$. Podem calcular
\[
E_{\lambda^+} = \langle \begin{pmatrix}i\\1\end{pmatrix}\rangle, \quad E_{\lambda^+} = \langle \begin{pmatrix}-i\\1\end{pmatrix}\rangle.
\]
Per tant, obtenim
\[
\begin{pmatrix}
i & -i\\1 & 1
\end{pmatrix}^{-1}
\begin{pmatrix}
a&-b\\b&a
\end{pmatrix}
\begin{pmatrix}
i & -i\\1 & 1
\end{pmatrix}
=
\begin{pmatrix}
a+bi&0\\0&a-bi
\end{pmatrix}.
\]
\end{exemple}
\begin{exemple}
Suposem que $A\in M_2(\R)$ té valors propis $a\pm bi$, amb $b\neq 0$. Com que els valors propis són diferents, la matriu $A$ és diagonalitzable i, per tant, és similar a
\[
D = \begin{pmatrix}
a+bi&0\\0&a-bi
\end{pmatrix}.
\]
Com que la matriu $D$ és també similar a
\[
R_{a,b} = \begin{pmatrix}
a&-b\\b&a
\end{pmatrix},
\]
la matriu $A$ és similar a $R_{a,b}$. De fet, tenim:
\[
D = S^{-1} A S,
\]
on $S$ té per columnes $\vec u$ i $\bar \vec u$, on $\vec u = \vec v  + i \vec w$, amb $\vec v,\vec w\in \R^2$. Substituïnt $D = T^{-1} R_{a,b} T$ amb $T = \begin{pmatrix}
i & -i\\1 & 1
\end{pmatrix}$, tenim
\[
T^{-1} R_{a,b} T = = S^{-1} A S\Longrightarrow R_{a,b} = TS^{-1} A ST^{-1}.
\]
Calculem directament que
\[
ST^{-1} = \begin{pmatrix}
\vert&\vert\\
\vec w&\vec v\\
\vert&\vert
\end{pmatrix}.
\]
\end{exemple}

\begin{llista-exercicis}
\item[Secció 6.2:] 4, 18.
\item[Secció 6.3:] 2, 6.
\item[Secció 7.1:] 10, 18, 34, 36.
\item[Secció 7.2:] 2, 10, 20.
\item[Secció 7.3:] 14, 22, 24.
\item[Secció 7.5:] 14, 24, 26.
\end{llista-exercicis}