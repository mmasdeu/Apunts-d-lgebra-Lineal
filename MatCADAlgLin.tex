% !TeX spellcheck = ca_ES-valencia
% !TeX encoding = UTF-8
\documentclass[a4paper,12pt,twoside]{article}
\usepackage[utf8]{inputenc}
\usepackage[catalan]{babel}
\usepackage{enumerate}
\usepackage{amsmath,amsthm,amssymb}
\usepackage{fancyhdr}
\usepackage{makeidx}
\usepackage{mathtools}
\usepackage{hyperref}
\usepackage[all]{xy}

\newcommand{\bbdef}[1]{\expandafter\newcommand% 
	\csname#1\endcsname{\mathbb{#1}}}
\bbdef{C} \bbdef{F} \bbdef{R} \bbdef{Z} \bbdef{Q} \bbdef{K} \bbdef{N}
%\bbdef{1}

%%% SCRIPT COMMANDS:  \cala=\mathcal{A}, ... \calz=\mathcal{Z}
\newcounter{let} \setcounter{let}{0}
\loop\stepcounter{let}
\expandafter\edef\csname cal\alph{let}\endcsname%
{\noexpand\mathcal{\Alph{let}}}
\ifnum\thelet<26\repeat


\renewcommand{\1}{\mathbf{1}}
\newcommand{\0}{\mathbf{0}}

\newenvironment{amatrix}[1]{%
  \left(\begin{array}{@{}*{#1}{r}|r@{}}
}{%
  \end{array}\right)
}

\newenvironment{llista-exercicis}{%
\subsection*{Exercicis recomanats}
Els exercicis que segueixen són útils per practicar el 
material presentat. La numeració és la de~\cite{Bret}.
\begin{description}}{\end{description}}

\newcommand{\smat}[1]{\left(\begin{smallmatrix}#1\end{smallmatrix}\right)}

\DeclareMathOperator{\Rang}{Rang}
\DeclareMathOperator{\rref}{rref}
\DeclareMathOperator{\rcef}{rcef}
\DeclareMathOperator{\Ker}{Ker}
\DeclareMathOperator{\Ima}{Im}
\DeclareMathOperator{\Id}{Id}
\DeclareMathOperator{\Map}{Map}
\DeclareMathOperator{\sign}{Sign}
\DeclareMathOperator{\refl}{Refl}
\DeclareMathOperator{\Tr}{Tr}
\DeclareMathOperator{\multalg}{MultAlg}
\DeclareMathOperator{\multgeom}{MultGeom}
\DeclareMathOperator{\proj}{Proj}

%margin
\usepackage[left=2cm, right=2cm, top=2cm, bottom=2cm, head=16pt]{geometry}
%\addtolength{\headheight}{5pt}
%\addtolength{\headsep}{2pt}
%\addtolength{\textwidth}{44mm}
%\addtolength{\topmargin}{-25mm}
%\addtolength{\textheight}{40mm}
%\addtolength{\evensidemargin}{-33mm}
%\addtolength{\oddsidemargin}{-13mm}

\pagestyle{fancy}
\lhead[\fancyplain{}{\bfseries \thepage}]%
{\fancyplain{}{\bfseries Departament de Matemàtiques, Universitat Autònoma de Barcelona}}
\rhead[\fancyplain{}{\bfseries Apunts d'Àlgebra Lineal}]%
{\fancyplain{}{\bfseries \thepage}}
\cfoot{\relax}


\setcounter{tocdepth}{2}


\newtheorem{teorema}{Teorema}[section]
\newtheorem{proposicio}[teorema]{Proposició}
\newtheorem{lema}[teorema]{Lema}
\newtheorem{corollari}[teorema]{Coro{\lgem}ari}

\theoremstyle{definition}
\newtheorem{definicio}[teorema]{Definició}
\newtheorem{exemple}[teorema]{Exemple}
\newtheorem{exercici}{Exercici}[section]

\theoremstyle{remark}
\newtheorem{notacio}[teorema]{Notació}
\newtheorem{observacio}[teorema]{Observació}

\makeindex

\title{Apunts d'Àlgebra Lineal}
\author{MatCAD - Universitat Autònoma de Barcelona}
%\date{}
\begin{document}
\maketitle
\tableofcontents
\newpage
% !TeX encoding = UTF-8
% !TeX spellcheck = ca_ES-valencia
% !TeX root = MatCADAlgLin.tex
\section*{Introducció}
Encara que el curs serà força autocontingut es requerirà que l'alumne conegui la resolució de sistemes d'equacions lineals, l'aritmètica bàsica de números i polinomis, i que tingui destresa de càlcul amb expressions algebraiques simbòliques.

A tot aquest curs suposarem que treballem sobre un cos commutatiu $\K$ fixat, que podeu pensar és $\Q$, $\R$ o $\C$. Els elements de $K$ els anomenarem nombres o escalars. Les propietats que utilitzarem són:
\begin{itemize}
	\item És commutatiu amb la suma: $a+b=b+a$ $\forall a,b\in \K$.
	\item És commutatiu amb el producte: $ab=ba$ $\forall a,b\in \K$.
	\item La suma té un element neutre que anomenem zero: $0+a=a$ $\forall a\in\K$.
	\item El producte té un elements neutre que anomenem u: $1a=a$ $\forall a\in\K$.
	\item Tot element $a\in\K$ té un invers per la suma que anomenem $-a$: $a+(-a)=0$.
	\item Tot element $a$ diferent de zero té un invers per la multiplicació que anomenem $1/a$ o bé $a^{-1}$: $a a^{-1}=1$.
	\item Hi ha les propietats associatives a la suma i al producte: $(a+b)+c=a+(b+c)$ i $(ab)c=a(bc)$ $\forall a,b,c \in \K$.
	\item Hi ha la propietat distributiva: $a(b+c)=ab+ac$ $\forall a,b,c \in \K$.
\end{itemize}

També suposarem certa familiaritat amb el llenguatge dels conjunts. Si $A$ és un conjunt i $B$ un subconjunt d'$A$, escriurem $B\subset A$. $a\in A$ voldrà dir que $a$ és un element d'$A$. També escriurem $A\setminus B=\{a \in A \mid a \not\in B\}$ i llegirem els $a$ que pertanyen a $a$ i que no pertanyen a $B$ (o bé el complementari de $B$ en $A$). 
 
 
% !TeX encoding = UTF-8
% !TeX spellcheck = ca_ES-valencia
% !TeX root = MatCADAlgLin.tex
El contingut d'aquesta secció el podem trobar a \cite[Tema 1]{Bret} i a \cite[Tema 2]{NaXa}.
\begin{definicio}
	Si $m$ i $n$ són dos nombres naturals, una \emph{matriu $m\times n$ amb entrades a $\K$}\index{matriu} és una taula rectangular d'elements de $\K$ amb $m$ files i $n$ columnes. Denotem $M_{m\times n}(\K)$ al conjunt de matrius que tenen $m$ files i $n$ columnes i els seus elements són de $\K$.
\end{definicio}
\begin{notacio}
	Denotarem amb lletres majúscules el nom de les matrius i amb la mateixa lletra i subíndexs cadascun dels coeficients: si $A$ és una matriu, anomenarem $a_{ij}$ al nombre de la fila $i$, columna $j$. En el producte de matrius, a vegades utilitzarem la notació $(AB)_{ij}$ per a fer referència al coeficient de la posició $(i,j)$ després de fer el producte.
\end{notacio}
\begin{exemple}
	Si $A=\big(\begin{smallmatrix}
	1 & 3 & 0 \\ 0 & -1 & 1
	\end{smallmatrix}\big) \in M_{2\times3}(\Q)$, llavors $a_{11}=1$, $a_{12}=3$, $a_{13}=0$, $a_{21}=0$, $a_{22}=-1$ i $a_{23}=1$.
\end{exemple}
A continuació fixem algunes notacions i definicions de casos particulars:
\begin{itemize}
	\item Una \emph{matriu quadrada}\index{matriu!quadrada} és una matriu amb el nombre de columnes igual al nombre de files. Denotarem per $M_n(\K)=M_{n\times n}(\K)$.
	\item Un element està a la \emph{diagonal d'una matriu quadrada}\index{element!diagonal} si la posició que ocupa té el mateix nombre de fila que de columna: si la matriu és $A$, els elements de la diagonal són els $a_{ii}$.  
	\item Una \emph{matriu diagonal}\index{matriu!diagonal} és una matriu quadrada on els únics elements no nuls estan a la diagonal: $A$, matriu quadrada, és diagonal si $a_{ij}=0$ $\forall i\neq j$. 
	\item La \emph{matriu identitat $n\times n$}\index{matriu!identitat} és una matriu diagonal on tots els elements de la diagonal valen $1$ (i per tant els altres valen $0$). La denotem per $\1_n$ la matriu identitat $n\times n$.
	\item En general, escriurem els vectors\index{vector} per columnes: un \emph{vector de $\K^n$} és una matriu amb $n$ files i $1$ columna.
	\item Direm que una matriu quadrada $A$ és \emph{triangular superior}\index{matriu!triangular inferior} si tots els coeficients per sota de la diagonal valen $0$, o sigui, $a_ {ij}=0$ si $i>j$.
	\item Direm que una matriu quadrada $A$ és \emph{triangular inferior}\index{matriu!triangular superior} si tots els coeficients per sobre de la diagonal valen $0$, o sigui, $a_ {ij}=0$ si $i<j$.
	\item Donada una matriu $A \in M_{m\times n}(\K)$, definim la \emph{transposada d'$A$}\index{matriu!transposada} i la denotem per $A^T$ com la matriu de $M_{n\times m}(\K)$ que té per columnes les files d'$A$, o sigui, que a la posició $(i,j)$ té el coeficient $a_{ji}$. Tenim la propietat:
	$$
	(A^T)^T=A \,.
	$$
	\item Diem que una matriu $A$ és \emph{simètrica}\index{matriu!simètrica} si $A=A^T$ (en particular, ha de ser quadrada). Per exemple, la matriu identitat $\1_n$ és simètrica.
\end{itemize}

\begin{exemple}\index{sistema d'equacions}
	Considerem un sistema d'equacions amb $m$ equacions i $n$ incògnites:
	\begin{align*}
	a_{11}x_1+a_{12}x_2+ \cdots + a_{1n}x_n &= b_1 \\
	a_{21}x_1+a_{22}x_2+ \cdots + a_{2n}x_n &= b_2 \\
	&\vdots \\
	a_{m1}x_1+a_{m2}x_2+ \cdots + a_{mn}x_n &= b_m
	\end{align*}
	
	D'aquí podem treure la matriu associada\index{sistema d'equacions!matriu associada} als coeficients del sistema:
	\[A=
	\begin{pmatrix*}[l]
	a_{11} & a_{12} & \cdots & a_{1n} \\
	a_{21} & a_{22} & \cdots & a_{2n} \\
	\vdots & \vdots & \ddots & \vdots \\
	a_{m1} & a_{m2} & \cdots & a_{mn} 
	\end{pmatrix*}
	\]
	el vector de termes independents\index{sistema d'equacions!termes independents}:
	\[
	B=
	\begin{pmatrix}
	b_1 \\ b_2 \\ \vdots \\ b_m
	\end{pmatrix}
	\]
	o bé escriure-ho tot en una sola matriu (matriu ampliada)\index{sistema d'equacions!matriu ampliada}, on habitualment separem els termes independents:
	\[
	\begin{amatrix}{4}
	a_{11} & a_{12} & \cdots & a_{1n} & b_1 \\
	a_{21} & a_{22} & \cdots & a_{2n} & b_2 \\
	\vdots & \vdots & \ddots & \vdots & \vdots \\
	a_{m1} & a_{m2} & \cdots & a_{mn} & b_m 
	\end{amatrix}
	\]
\end{exemple}
\subsection{Operacions amb matrius. Matriu invertible}\index{matriu!operacions}\label{subsec:opmat}
Considerem $\lambda \in \K$ i $A,B \in M_{m\times n}(\K)$. Les primeres operacions que podem fer són les que corresponen a veure $M_{m\times n} (\K)$ com $\K$-espai vectorial (més endavant veurem què vol dir):
\begin{itemize}
	\item Definim la matriu $\lambda A$ com la matriu que té per coeficients $\lambda a_{ij}$.\index{matriu!producte per escalar}
	\item Definim la matriu $A+B$ com la matriu que té per coeficients $a_{ij}+b_{ij}$.\index{matriu!suma}
\end{itemize}
\begin{exemple}
	\[
	\begin{pmatrix*}[r] -1 & 1 & 0 \\ 1 & 2 & 1 \end{pmatrix*} +
	\begin{pmatrix*}[r] 0 & -1 & 2 \\ -1 & 3 & 2 \end{pmatrix*} =
	\begin{pmatrix*}[r] -1 & 0 & 2 \\ 0 & 5 & 3 \end{pmatrix*}
	\]
\end{exemple}
\begin{exemple}
	$$
	2 \begin{pmatrix} -1 & 1 & 0 \\ 1 & 2 & 1 \end{pmatrix} =
	\begin{pmatrix} -2 & 2 & 0 \\ 2 & 4 & 2 \end{pmatrix} 
	$$
\end{exemple}
Aquestes definicions, més les propietats dels elements de $\K$ impliquen:
\begin{enumerate}
	\item Existeix una matriu $\0_{mn}$ complint $A+\0_{mn}=A$, $\forall A \in M_{mn}(\K)$ ($\0_{mn}$ té tots els coeficients zero).
	\item $0 A = \0_{mn}$, $\forall A \in M_{m\times n}(\K)$.
	\item $A+B=B+A$, $\forall A,B \in M_{m\times n}(\K)$.
	\item $1A=A$, $\forall A \in M_{m\times n}(\K)$.
	\item $\lambda (\mu A)= (\lambda \mu) A$, $\forall \lambda,\mu \in \K$ i $\forall A \in M_{m\times n}(\K)$
\end{enumerate}

Abans de la definició del producte de matrius és convenient utilitzar el llenguatge següent:
\begin{definicio}\label{def:dependlineal}
	Diem que \emph{un vector $\vec{w}$ és combinació lineal de $\{\vec{v}_1,\dots,\vec{v}_n\}$}\index{vector!combinació lineal} si existeixen escalars $\lambda_1, \dots , \lambda_n$ tals que $w=\lambda_1\vec{v}_1+\cdots+\lambda_n\vec{v}_n$.
	
	Si un dels vectors de $\{\vec{v}_1,\dots,\vec{v}_n\}$ es pot escriure com a combinació lineal dels altres, diem que \emph{la família de vectors $\{\vec{v}_1,\dots,\vec{v}_n\}$ és linealment dependent}\index{vector!dependència lineal}.
	
	En cas contrari, diem que \emph{la família de vectors $\{\vec{v}_1,\dots,\vec{v}_n\}$ és linealment independent}\index{vector!independència lineal}.
\end{definicio}

Parlarem de files (o columnes) linealment dependents o independents d'una matriu $A$ pensades com a vectors amb tantes components com files (o columnes) tingui $A$.

\begin{exercici}
	Demostreu que $\{\vec{v}_1,\dots,\vec{v}_n\}$ són linealment independents si l'única manera d'escriure el vector zero $\vec{0}=\lambda_1\vec{v}_1+\cdots+\lambda_n\vec{v}_n$ és amb $\lambda_1=\cdots\lambda_n=0$.
\end{exercici}

A més, a més, podem definir:
\begin{definicio}\index{matriu!producte de matrius}
	Si $A \in M_{m\times n}(\K)$, $B \in M_{n\times r}(\K)$ (o sigui, el nombre de columnes de $A$ és igual al nombre de files de $B$) podem definir el \emph{producte $AB$} com la matriu $C\in M_{m\times r}$ que té per coeficients:
	\[
	c_{ij}=\sum_{k=1}^{n} a_{ik}b_{kj} \,.
	\]
	La matriu $C$ és pot pensar que té per columnes combinacions lineals de columnes de $A$ ($B$ ens diu quines són aquestes combinacions lineals): la columna $j$ de la matriu $C$ és:
	\[
	\begin{pmatrix}
	c_{1j}\\c_{2j}\\ \vdots \\ c_{mj} 
	\end{pmatrix} =
	b_{1j}	\begin{pmatrix}
	a_{11}\\a_{21}\\ \vdots \\ a_{m1} 
	\end{pmatrix} +
	b_{2j}	\begin{pmatrix}
	a_{12}\\a_{22}\\ \vdots \\ a_{m2} 
	\end{pmatrix} + \cdots +
	b_{nj}	\begin{pmatrix}
	a_{1n}\\a_{2n}\\ \vdots \\ a_{mn} 
	\end{pmatrix}
	\]
	Anàlogament, la matriu $C$ es pot pensar que té per files combinacions lineals de files de $B$ ($A$ ens diu quines són aquestes combinacions lineals): la fila $i$ de la matriu $C$ és (separem amb comes per a que quedi més clar):
	\[
	(c_{i1},c_{i2},\cdots,c_{ir})=
	a_{i1} (b_{11},b_{12},\dots,b_{1r})+
	a_{i2} (b_{21},b_{22},\dots,b_{2r})+ \cdots +
	a_{in} (b_{n1},b_{n2},\dots,b_{nr})
	\]
\end{definicio}
\begin{exemple}\label{exempl:prodmat}
	$$
	\begin{pmatrix*}[r] -1 & 1 & 0 \\ 1 & 2 & 1 \end{pmatrix*}
	\begin{pmatrix*}[r] 1 & 2 \\ 3 & 4 \\ 5 & 6 \end{pmatrix*}
	= 	\begin{pmatrix*}[r] 2 & 2 \\ 12 & 16 \end{pmatrix*}
	$$
	$$
	\begin{pmatrix} 1 & 2 \\ 3 & 4 \\ 5 & 6 \end{pmatrix}
	\begin{pmatrix} -1 & 1 & 0 \\ 1 & 2 & 1 \end{pmatrix}
	= 	\begin{pmatrix} 1 & 5 & 2 \\ 1 & 11 & 4 \\ 1 & 17 & 6 \end{pmatrix}
	$$
\end{exemple}

\begin{exemple}\index{sistema d'equacions}
	Podem escriure el sistema d'equacions
	\begin{align*}
	a_{11}x_1+a_{12}x_2+ \cdots + a_{1n}x_n &= b_1 \\
	a_{21}x_1+a_{22}x_2+ \cdots + a_{2n}x_n &= b_2 \\
	&\vdots \\
	a_{m1}x_1+a_{m2}x_2+ \cdots + a_{mn}x_n &= b_m
	\end{align*}
	com $AX=B$, on
	\[
	A=
	\begin{pmatrix}
	a_{11} & a_{12} & \cdots & a_{1n} \\
	a_{21} & a_{22} & \cdots & a_{2n} \\
	\vdots & \vdots & \ddots & \vdots \\
	a_{m1} & a_{m2} & \cdots & a_{mn} 
	\end{pmatrix},\quad
	X=
	\begin{pmatrix}
	x_1 \\ x_2 \\ \vdots \\ x_n
	\end{pmatrix} \text{ i}\quad
	B=
	\begin{pmatrix}
	b_1 \\ b_2 \\ \vdots \\ b_m
	\end{pmatrix}.
	\]
\end{exemple}

\begin{proposicio}
	El producte de matrius té les propietats següents:
	\begin{enumerate}[\rm (a)]
		\item Element neutre: si $A \in M_{m\times n}(\K)$, $\1_m A = A \1_n = A$.
		\item Propietat associativa: si $A \in M_{m\times n}(\K)$, $B \in M_{n\times r}(\K)$, $C \in M_{r\times s}(\K)$, llavors
		\[(AB)C=A(BC).\]
		\item Distributiva respecte el producte: si $A \in M_{m\times n}(\K)$, $B, C \in M_{n\times r}(\K)$ i $D \in M_{r\times s}(\K)$, llavors $A(B+C)=AB+AC$ i $(B+C)D=BD+CD$.
	\end{enumerate}
\end{proposicio}
\begin{proof}
	Escriure les fórmules amb els coeficients i surt. Fem la propietat associativa com exemple: volem comparar el coeficient a la posició $(i,j)$ d'$(AB)C$ amb el d'$A(BC)$, que denotem $((AB)C)_{ij}$ i $(A(BC))_{ij}$ respectivament:
	$$
	((AB)C)_{ij}=\sum_{k=1}^r(AB)_{ik}c_{kj}=\sum_{k=1}^r(\sum_{l=1}^na_{il}b_{lk})c_{kj}=\sum_{k=1}^r\sum_{l=1}^na_{il}b_{lk}c_{kj}
	$$
	Mentre que:
	$$
	(A(BC))_{ij}=\sum_{l=1}^na_{il}(BC)_{lj}=\sum_{l=1}^na_{il}(\sum_{k=1}^rb_{lk}c_{kj})=\sum_{l=1}^n\sum_{k=1}^ra_{il}b_{lk}c_{kj}
	$$
	I els dos resultats són el mateix ja que podem commutar els sumatoris.
\end{proof}
\begin{observacio}
	El producte de matrius, en general, no és commutatiu (veure l'Exemple \ref{exempl:prodmat}).
\end{observacio}
\begin{proposicio}
	Si $A\in M_{m\times n}(\K)$ i $B\in M_{n\times r}(\K)$, llavors tenim la relació següent entre productes i transposades
	$$
	(AB)^T=B^T A^T
	$$
\end{proposicio}
\begin{proof}
	Escrivim les fórmules dels coeficients:
	$$
	((AB)^T)_{ij}=(AB)_{ji}=\sum_{k=1}^n a_{jk}b_{ki}
	$$
	mentre que
	$$
	(B^TA^T)_{ij}=\sum_{k=1}^n (B^T)_{ik}(A^T)_{kj}=\sum_{k=1}^n b_{ki}a_{jk}
	$$
	i són iguals per la propietat commutativa del producte a $\K$.
\end{proof}
\begin{definicio}\index{producte escalar}
	Si considerem $\vec{v}$ i $\vec{w}$ vectors de $\K^n$, que escrivim com una columna cadascun, definim el \emph{producte escalar $\vec{v}\cdot\vec{w}$} com:
	$$
	\vec{v}\cdot \vec{w}=\vec{v}^T \vec{w}
	$$
	Per tant, si les coordenades són:
	$$
	\vec{v}=\begin{pmatrix}
	v_1 \\ v_2 \\ \vdots \\ v_n 
	\end{pmatrix}, 
	\vec{w}=\begin{pmatrix}
	w_1 \\ w_2 \\ \vdots \\ w_n 
	\end{pmatrix}
	\text{ llavors }
	\vec{v}\cdot\vec{w}=\sum_{i=1}^n v_iw_i \,.
	$$
\end{definicio}

Volem definir la inversa d'una matriu quadrada, però com que el producte no és commutatiu, hauríem de parlar d'inversa per l'esquerra o per la dreta.
\begin{definicio}\index{matriu!inversa}
	Diem que una \emph{matriu quadrada $A \in M_{n}(\K)$ és invertible} si existeix una matriu $B \in M_n(\K)$ tal que $AB=BA=\1_n$.\\
	%Diem que una \emph{matriu quadrada $A \in M_{n}(\K)$ és invertible per l'esquerra} si existeix una matriu $C \in M_n(\K)$ tal que $CA=\1_n$.
\end{definicio}
\begin{teorema}\label{teo:invuniq}
	Si $A$ és una matriu quadrada $n\times n$ i $B$ és una matriu també $n\times n$. Llavors $AB=\1_n$ si i només si $BA=\1_n$. En particular, si això passa, $A$ és invertible i la inversa és única. Denotem la matriu $B$ com $A^{-1}$.
\end{teorema}
\begin{proof}
	La primera part la demostrarem quan tinguem el concepte de rang d'una matriu (Observació \ref{obs:invuniq}). Vegem ara que si $B$ és tal que $AB=\1_n$, i $C$ tal que $CA=\1_n$, llavors $B=C$:
	$$
	C = C \1_n = C(AB)=(CA)B=\1_n B=B \,.
	$$
	Això implica que la inversa és única: Suposem $B'$ tal que $AB'=\1_n$, llavors, pel raonament d'abans, $B'=C=B$.
\end{proof}
\begin{proposicio}
	\begin{enumerate}[\rm (a)]
		\item Si $A \in M_{n\times n}(\K)$ és invertible, llavors $A^T$ també ho és i $(A^T)^{-1}=(A^{-1})^T$.
		\item Si $A,B \in M_{n\times n}(\K)$ són matrius invertibles, llavors el producte $AB$ també ho és i $(AB)^{-1}=B^{-1}A^{-1}$.
	\end{enumerate}
\end{proposicio}
\begin{proof}
	Demostrem primer (a): com que ens proposen una inversa, tant sols cal comprovar que ho és:
	$$
	(A^T)(A^{-1})^T=(A^{-1}A)^T=\1_n^T=\1_n \,.
	$$
	per tant, $(A^{-1})^T$ és inversa d'$A^T$ (com que $A^T$ és quadrada, pel Teorema \ref{teo:invuniq}, tant sols cal comprovar-ho per un dels costats).
	
	Demostrem ara (b): com que $A$ i $B$ són invertibles, existeixen les matrius inverses $A^{-1}$ i $B^{-1}\in M_{n\times n}(\K)$ respectivament. Llavors:
	$$
	(AB)(B^{-1}A^{-1})=A(BB^{-1})A^{-1}=A\1_n A^{-1}=A A^{-1}=\1_n
	$$
	Igual que (a), això ja demostra que $B^{-1}A^{-1}$ és inversa d'$AB$.
\end{proof}
\begin{observacio}
	D'aquí es dedueix que si $A_1, \dots, A_r \in M_{n\times n}(\K)$ són invertibles, el seu producte també ho és i $(A_1\cdots A_r)^{-1}=A_r^{-1}\cdots A_1^{-1}$.
\end{observacio}
\subsection{Transformacions elementals en matrius}\label{subsec:trans_el}\index{matriu!transformacions elementals}
Considerem una matriu organitzada per files (encara que tot el que farem aquí també es pot fer per columnes) i definim les \emph{transformacions elementals}:
\begin{center}
	\fbox{
		\begin{minipage}{0.95\textwidth}
			\begin{enumerate}[\bf T1.]
				\item Multiplicar una de les files per $\lambda\neq 0$.
				\item Sumar a una de les files $\mu$ vegades una altra fila.
				\item Intercanviar dues files.
			\end{enumerate}
		\end{minipage}
	}
\end{center}
%\begin{definicio}
%	Donada una matriu $A\in M_{m\times n}(K)$ fixada. Direm que $B$ és el resultat d'aplicar una transformació elemental a $A$ si els coeficients de $B$ són els mateixos que els de $A$ excepte:
%	\begin{enumerate}[\bf T1.]
%		\item Una de les files de $B$ té els corresponents valors de la mateixa fila d'$A$ multiplicats per $\lambda\neq 0$.
%		\item Una de les files de $B$ té els valors de la mateixa fila d'$A$ més $\mu$ vegades una altra fila d'$A$.
%		\item $B$ és el resultat d'intercanviar dues files d'$A$.
%	\end{enumerate}
%\end{definicio}
\begin{observacio}
	Si considerem la matriu d'un sistema d'equacions i apliquem qualsevol de les transformacions elementals, no modifiquem les solucions.
\end{observacio}
\begin{observacio}
	Les tres transformacions elementals es poden desfer mitjançant una transformació elemental:
	\begin{enumerate}[\bf 1.]
		\item Multiplicar una de les files per $1/\lambda\neq 0$.
		\item Sumar a una de les files $-\mu$ vegades una altra fila.
		\item Intercanviar dues files.
	\end{enumerate} 
\end{observacio}
Aquestes transformacions elementals es poden fer (i desfer) multiplicant per una matriu invertible $P$ per l'esquerra. Suposem que $A$ és una matriu amb $m$ files:
\begin{enumerate}[\bf T1.]
	\item si apliquem aquest canvi a la fila $i$ d'$\1_m$ tindrem una matriu $P$ on hem modificat el $1$ de la fila $i$ per $\lambda\neq 0$. Llavors $PA$ té els mateixos valors que $A$, però amb la fila $i$ multiplicada per $\lambda$. En aquest cas, $P^{-1}$ és una matriu identitat amb un $1/\lambda$ a la posició $(i,i)$.
	\item si sumem a la fila $i$ de la matriu $\1_n$ $\mu$ vegades la fila $k\neq i$ tindrem una matriu $P$ tal que $PA$ té a la fila $i$ la fila $i$ d'$A$ més $\mu$ vegades la fila $k$ d'$A$. En aquest cas, $P^{-1}$ és una matriu amb $1$ a la diagonal, $0$ fora, excepte la posició $(k,i)$, que val $-\mu$.
	\item si intercanviem les files $i$ i $k$ de la matriu $\1_n$ ($i\neq k$) obtenim una matriu $P$ tal que $PA$ és el resultat d'intercanviar les files $i$ i $k$ d'$A$. En aquest cas, $P^{-1}=P$.
\end{enumerate}
Amb aquests raonaments hem demostrat:
\begin{proposicio}
	Si considerem la matriu identitat $\1_m$ i li apliquem transformacions elementals per files, la matriu $P$ que obtenim és invertible. Si apliquem exactament les mateixes transformacions elementals a una altra matriu $A\in M_{m\times n}(\K)$, la matriu que resulta és exactament $PA$.
\end{proposicio}
\begin{definicio}\index{matriu!matrius equivalents}
	Diem que \emph{dues matrius $A$ i $B$ són equivalents per files} si es pot passar d'$A$ a $B$ mitjançant transformacions elementals per files. Escriurem $A \sim B$.
\end{definicio}
\begin{proposicio} \label{prop:relequiv} Si $A$, $B$ i $C$ són matrius de dimensions iguals, aleshores es té:
	\begin{itemize}
		\item $A \sim A$ (reflexiva),
		\item $A \sim B$ si i només si $B \sim A$ (simètrica),
		\item $A \sim B$ i $B \sim C$ implica $A \sim C$ (transitiva).
	\end{itemize}
\end{proposicio}
\begin{proof}
	La primera és \textbf{T2} per a qualsevol fila i $\mu=0$. 
	
	La segona és que la inversa d'una transformació elemental és una transformació elemental (i les composem en ordre invers): si $P_1$, \dots $P_r$ són les transformacions elementals que apliquem a $A$ per obtenir $B$, resulta que:
	$$
	P_r \cdots P_1 A=B
	$$
	llavors
	$$
	A= P_1^{-1} \cdots P_r^{-1} B
	$$
	Però si $P_i$ és una transformació elemental, $P_i^{-1}$ també i per tant $B\sim A$.
	
	La tercera és per definició de $\sim$, ja que passem de $A$ a $C$ fent primer els canvis elementals que transformen $A$ en $B$ i després els que transformen $B$ en $C$.
\end{proof}

\begin{observacio}
	Si tenim un conjunt $S$ i una relació $\sim$ entre els seus elements complint les propietats de la Proposició \ref{prop:relequiv} (reflexiva, simètrica i transitiva) diem que és una \emph{relació d'equivalència}\index{relació equivalència}.
\end{observacio}

\begin{definicio}
	\index{matriu!matriu esglaonada}
	Diem que una matriu $A$ està \emph{en forma esglaonada (per files)} si compleix que:
	\begin{enumerate}
		\item El primer element no nul de cada fila val $1$, i l'anomenem \emph{pivot}\index{matriu!pivot}.
		\item Si una fila conté un pivot a la columna $j$, les files superiors també tenen un pivot a una columna $j'<j$.
		\item Totes les entrades per sota d'un pivot valen $0$.
	\end{enumerate}
	
	\index{matriu!matriu redu{\"\i}da}
	Diem que una matriu $A$ està \emph{en forma reduïda (per files)} si està esglaonada i a més compleix que:
	\begin{enumerate}
		\item[3'] Si una columna té un pivot, aquest és l'únic element no nul de la seva columna.
	\end{enumerate}
\end{definicio}
\begin{exemple}
	De les matrius següents:
	\[
	A=\begin{pmatrix}
	1 & 2 & 0 \\ 0 & 0 & 1
	\end{pmatrix},
	B=\begin{pmatrix}
	1 & 0 & 1 \\ 0 & 0 & 1
	\end{pmatrix} \text{ i }
	C=\begin{pmatrix}
	0 & 1 & 0 \\ 1 & 0 & 1
	\end{pmatrix} ,
	D = \begin{pmatrix}
	1&2&3\\
	0&1&5
	\end{pmatrix}
	\]
	la matriu $A$ està reduïda per files, mentre que $B$ i $C$ no ho són. La matriu $D$ està esglaonada, però no reduïda.
\end{exemple}
Considerem ara l'algorisme següent (\textbf{Mètode de Gauss} o \textbf{Mètode de Gauss-Jordan})\index{mètode de Gauss}\index{mètode de Gauss-Jordan}, que aplica canvis elementals per files a una matriu $A$ fins que obtenim una matriu en forma reduïda per files. A aquest mètode també li diem \textbf{triangular la matriu per files}.
\begin{center}
	\fbox{
		\begin{minipage}{0.97\textwidth}
			Suposem que tenim una matriu $A\in M_{m\times n}(\K)$, on la submatriu formada per les $i_1$ primeres files ja té forma reduïda i si el pivot de la fila $i_1$ és a la posició $j_1$, els coeficients $a_{ij}=0$ si $i>i_1$ i $j<j_1$.
			
			\begin{enumerate}[\bf {G}1]
				\item Busquem la primera columna que no sigui tota zero a la submatriu formada per les files $i_1+1$ fins la $m$ (suposem columna $j_2$), i triem una fila a aquesta submatriu on hi hagi un element diferent de zero a aquesta columna. Si cal, apliquem el canvi \textbf{T3} per a que la fila $i_2:=i_1+1$ tingui un valor diferent de zero a aquesta columna. Podem suposar, doncs, que $a_{i_2j_2}\neq0$ i que $a_{ij}=0$ si $i>i_2$ i $j<j_2$.
				\item Volem que a la posició $(i_2,j_2)$ hi hagi un pivot ($=1$), per tant multipliquem tota la fila $i_2$ per $1/a_{i_2j_2}$ (canvi \textbf{T1}).
				\item Per a totes les files $i\neq i_2$ tals que $a_{ij_2}\neq 0$, hi restem la fila $i_2$ multiplicada per $a_{ij_2}$ (canvi \textbf{T2}), de tal manera que farem que a la posició $(i,j_2)$ hi hagi un zero per a tota $i\neq i_2$.
			\end{enumerate}
			Ara hem aconseguit que la submatriu formada per les files $1$, \dots ,$i_1+1$ sigui en forma reduïda.
			
			Iterem aquest procediment fins que $i_1=m$.
		\end{minipage}
	}
\end{center}
Com que la matriu té un nombre finit de files, aquest algorisme sempre acaba. A més, acabem de demostrar que:
\begin{teorema}
	Tota matriu $A\in M_{m\times n}(\K)$ és equivalent a una matriu reduïda.
\end{teorema}

\begin{exemple}
	Considerem la matriu:
	$$
	A=\begin{pmatrix}
	0 & 1 & 3 & 0 \\  1 & -2 & -5 & 4\\ 2 & -3 & -7 & 7
	\end{pmatrix}
	$$
	Apliquem el canvi \textbf{T3}, canviant la primera fila per la segona, obtenint:
	$$
	\begin{pmatrix}
	1 & -2 & -5 & 4\\0 & 1 & 3 & 0 \\   2 & -3 & -7 & 7
	\end{pmatrix}
	$$
	Apliquem el canvi \textbf{T2}, restant a la tercera fila $2$ cops la primera:
	$$
	\begin{pmatrix}
	1 & -2 & -5 & 4\\0 & 1 & 3 & 0 \\   0 & 1 & 3 & -1
	\end{pmatrix}
	$$
	Com que a la segona fila, la primera posició ja és $1$, podem utilitzar-la directament de pivot i sumar-la $2$ cops a la primera fila, i restar-la a la tercera:
	$$
	\begin{pmatrix}
	1 & 0 & 1 & 4\\0 & 1 & 3 & 0 \\   0 & 0 & 0 & -1
	\end{pmatrix}
	$$
	Les dues primeres files ja estan en forma reduïda, pel que podem considerar la tercera fila, i multiplicar-la per $-1$ per a que l'única posició no nu{\lgem}a sigui un pivot:
	$$
	\begin{pmatrix}
	1 & 0 & 1 & 4\\0 & 1 & 3 & 0 \\   0 & 0 & 0 & 1
	\end{pmatrix}
	$$
	Finalment restem la tercera fila a la primera multiplicada per $4$:
	$$
	\begin{pmatrix}
	1 & 0 & 1 & 0\\0 & 1 & 3 & 0 \\   0 & 0 & 0 & 1
	\end{pmatrix}
	$$
\end{exemple}

\subsection{Criteri d'invertibilitat. Rang d'una matriu}
Considerem primer el resultat següent:
\begin{teorema}\label{teo:critinv}\index{matriu!inversa}
	Donada una matriu quadrada $A \in M_{n\times n}(\K)$, les condicions següents són equivalents:
	\begin{enumerate}[\rm (a)]
		\item $A$ és equivalent a $\1_n$.
		\item $A$ és invertible.
	\end{enumerate}
\end{teorema}
\begin{proof}
	Vegem primer (a) implica (b): si $A$ és equivalent a $\1_n$, existeix $P$ tal que $PA=\1_n$. Com que $P$ és invertible, existeix $P^{-1}$ i podem fer:
	$$PA=\1_n \Rightarrow P ^{-1}PA=P^{-1} \Rightarrow A=P^{-1} \Rightarrow AP = P^{-1}P=\1_n$$
	per tant $P$ és la inversa d'$A$.
	
	Vegem (b) implica (a): suposem que $A$ no és equivalent a $\1_n$. Llavors, forçosament, quan l'esglaonem hi haurà una fila sense pivot, i per tant, tot zeros. Llavors obtenim $PA=A'$, amb $P$ invertible i $A'$ una matriu que té l'última fila tot zeros. Si $A$ fos invertible, voldria dir que existeix $Q$ tal que $AQ=\1_n$, llavors, però llavors:
	$$
	AQ=\1_n \Rightarrow PAQ=P \Rightarrow A'Q=P \Rightarrow A'(QP^{-1})=\1_n
	$$
	Observem ara que si $A'$ té l'última fila tot zeros $A'(QP^{-1})$ també, i per tant no pot ser $\1_n$.
\end{proof}
\begin{observacio}\label{obs:invuniq}
	Aquests resultats donen la demostració de la primera part del Teorema \ref{teo:invuniq}.
\end{observacio}
\begin{observacio}\label{obs:noinv}
	Si considerem $A \in M_{n\times n}(\K)$ que es pot subdividir en 4 submatrius:
	$$
	A=\left(\begin{array}{c|c}
	B & C \\ \hline \0 & D
	\end{array}\right)
	$$ 
	on la matriu $\0$ està formada per zeros i toca la diagonal (conté un coeficient amb coordenades $(i,i)$), llavors $A$ no és invertible.
\end{observacio}
\begin{proof}
	Primer observem que per a que una matriu sigui invertible, l'esglaonament ha de fer que a totes les seves columnes (i files) hi hagi un pivot (es dedueix del Teorema \ref{teo:critinv}).
	
	Si fos invertible podríem fer els canvis elementals per files i obtenir $\1_n$. Com que la submatriu $\0$ toca la diagonal, llavors $B$ té més columnes que files i $D$ més files que columnes. Quan esgraonem la matriu $A$, el pivot de les primeres columnes ha de ser a $B$ (sota hi ha zeros) i per tant no hi pot haver més pivots que files a $B$ (a les columnes de $B$), per tant hi ha columnes de $A$ sense pivot, i per tant no pot ser invertible.
\end{proof}

Podem utilitzar el concepte de matriu reduïda equivalent per a definir el rang d'una matriu $A$. Cal tenir en compte que necessitarem demostrar que la definició no depèn de quina matriu reduïda equivalent a $A$ considerem.

%Fem primer l'observació següent, que ens diu que les files diferent de zero d'una %matriu en forma reduïda són \emph{linealment independents}:
%\begin{observacio}\label{obs:linindep}\index{independència lineal}
%Si $A_1$, \dots , $A_{i_1}$ són les files diferents de zero d'una matriu $A$ reduïda %per files, i considerem $\lambda_1$, \dots , $\lambda_{i_1} \in \K$ tals que %$\lambda_1A_1+\cdots+\lambda_{i_1}A_{i_1}$ té tots els coeficients zero, llavors %$\lambda_1 = \cdots = \lambda_{i_1}=0$ (diem que les files $A_1$, \ldots, $A_{i_1}$ són %\emph{linealment independents}). 
%\end{observacio}

%\begin{proof}
%Com que $A$ està en forma reduïda, cada fila $i$ diferent de zero conté un pivot (=1), i d'aquí resulta que s'ha de complir $\lambda_i=0$. Això es pot aplicar a cada $i$.
%\end{proof}

\begin{proposicio}
	Si $A$ i $B$ són dues matrius en forma reduïda que són equivalents, llavors $A=B$. %tenen el mateix nombre de files diferent de zero.
\end{proposicio}
\begin{proof}
	Suposem que $A$ i $B$ són diferents. Considerem submatrius $A'$ i $B'$ formades per la primera columna on difereixin, així com per totes les columnes a l'esquerra d'aquesta que continguin pivots. Observem que $A'$ i $B'$ segueixen essent equivalents, mitjançant les mateixes transformacions elementals associades a $A$ i $B$. Podem interpretar $A'$ i $B'$ com matrius augmentades de sistemes lineals equivalents, posem en $r$ incògnites. Si aquests són compatibles, aleshores necessàriament els termes independents han de coincidir i, per tant obtenim una contradicció. Si aquests dos sistemes són incompatibles, això significa que l'última columna conté zeros en les primeres files (tantes com pivots) i necessàriament tindrem que aquesta última columna tindrà un pivot a l'entrada $r$-èssima. Per tant, obtenim una nova contradicció.
\end{proof}
%\begin{proof}
%	Vegem primer que tenen el mateix nombre de files diferent de zero: suposem que no, 
%que una de les matrius té més files diferents de zero que l'altra. Podem suposar que 
%$A$ té $i_1$ files diferent de zero, que $B$ té $i_2$ files diferent de zero i que 
%$i_1>i_2$.
%
%A més, com que les dues matrius són equivalents, existeix una matriu invertible $P$ (la
%composició de les transformacions elementals) tal que $PA=B$.
%
%Quan fem $PA=B$, estem dient que les files de $B$ són combinacions lineals de les d'$A$ %(que denotem per $A_1$, \dots $A_m$). Tenim que:
%$$
%0=B_{i_2+1}=p_{i_2+1,1}A_1+p_{i_2+1,2}A_2 + \cdots p_{i_2+1,i_1}A_{i_1}
%$$
%i per l'Observació \ref{obs:linindep}, 
%$p_{i_2+1,1}=p_{i_2+1,2}=\cdots=p_{i_2+1,i_1}=0$. Podem fer el mateix raonament per 
%totes les files entre $i_2+1$ i $n$, per tant tenim que la matriu $P$ és de la forma de
%l'Observació \ref{obs:noinv}, i per tant no és invertible, arribant a contradicció.

%Com que la contradicció ver de suposar que $A$ i $B$ tenen diferent nombre de files 
%diferent de zero, vol dir que han de tenir el mateix nombre de files no nu{\lgem}es.

%Falta veure que les files no nu{\lgem}es són iguals: com que són equivalents, existeix una 
%matriu invertible $P$ tal que $PA=B$, per tant, les files de $B$ són combinació lineal %de les d'$A$ (i les d'$A$ també són combinació de les de $B$). Això vol dir, en %particular, que els pivots estan a les mateixes columnes (si no, posant les matrius una %sobre l'altre $\big(\begin{smallmatrix}A \\ B \end{smallmatrix})\big)$, l'esglaonament %tindria més files diferent de zero).

%Mirem la primera fila d'$A$ i $B$: la posició del pivot bé donada per la primera %columna no nu{\lgem}a, i, com que són equivalents, ha de ser la mateixa, per tant (només %escric com a combinació lineal de les files no nu{\lgem}es, i que per tant tenen un pivot):
%$$
%(b_{11},b_{12},\dots,b_{1n})=(a_{11},a_{12},\dots,a_{1n})+\lambda_2(a_{21},a_{22},\dots%, a_{2n})+\cdots+\lambda_{i_1}(a_{i_11},a_{i_12},\dots,a_{i_1n})
%$$
%Utilitzem ara que els pivots estan a les mateixes columnes, que per tant tenen el %corresponent $b_{ij=0}$ per a veure que $\lambda_2=\dots=\lambda_{i_1}=0$.

%Aquest argument es pot utilitzar a totes les files no nu{\lgem}es: totes tenen un pivot a %la mateixa posició i zero on hi ha el pivot de les altres files.
%\end{proof}
\begin{notacio}
	D'aquí és dedueix que donada una matriu $A\in M_{n\times n}(\K)$, existeix una sola matriu reduïda equivalent a $A$, i l'anomenem $\rref(A)$ (\emph{reduced row-echelon form}) \index{rref}.
\end{notacio}
Ara ja podem definir el rang d'una matriu:
\begin{definicio}\index{matriu!rang}
	Donada una matriu $A$, definim el \emph{rang d'$A$} com el nombre de files diferents de zero (igual al nombre de pivots) de $\rref(A)$.
\end{definicio}
\begin{corollari}
	Si fem transformacions elementals a una matriu $A\in M_{m\times n}(\K)$, el rang de la matriu resultant és el mateix. Això també ho podem enunciar dient que si $P\in M_{n\times n}(\K)$ és una matriu invertible i $A\in M_{m\times n}(\K)$, llavors $\Rang(A)=\Rang(PA)$.
\end{corollari}
\begin{corollari}
	Una matriu quadrada $A\in M_{n\times n}(\K)$ és invertible si i només si té rang $n$.
\end{corollari}
\begin{proof}
	Considerem el Teorema \ref{teo:critinv}, i per tant és invertible si i només si és equivalent a $\1_n$, per tant, si i només si té rang $n$.
\end{proof}
El Mètode de Gauss ens dóna una manera de calcular la inversa d'una matriu: suposem que ens donen una matriu $A \in M_{n\times n}(\K)$. Considerem la matriu formada per una matriu identitat $n\times n$ al costat de la matriu $A$:
$$
\left(\begin{array}{c|c}A & \1_n \end{array}\right) \in M_{n\times 2n}(\K)
$$
Si apliquem canvis per files a aquesta matriu (cada fila té $2n$ coeficients) obtindrem matrius que podem escriure com:
$$
\left(\begin{array}{c|c}A'& P \end{array}\right) \in M_{n\times 2n}(\K)
$$
amb $A',P\in M_{n\times n}(\K)$ i que compleixen que $PA=A'$.

Com a cas particular tenim que, si $A$ és invertible, podem arribar a la situació
\begin{equation}\label{eq:inversa}
\left(\begin{array}{c|c}\1_n & P \end{array}\right) \in M_{n\times 2n}(\K)
\end{equation}
i tindrem que $P=A^{-1}$.

Si la matriu $A$ no fos invertible, el Teorema~\ref{teo:critinv} ens diu que no seria possible arribar a la situació de l'Equació \eqref{eq:inversa}.
\begin{exemple}
	Considerem la matriu:
	$$
	A=\begin{pmatrix*}[r]
	1 & 2 & 6 \\ 0 & -1 & -8 \\ 5 & 6 & 0
	\end{pmatrix*}
	$$
	Escrivim la matriu amb una còpia de la matriu identitat a la dreta:
	$$
	\left(\begin{array}{rrr|rrr}
	1 & 2 & 6 & 1 & 0 & 0 \\
	0 & -1 & -8& 0 & 1 & 0 \\
	5 & 6 & 0 & 0 & 0 & 1
	\end{array}\right)
	$$
	Esglaonem la part matriu resultant segon el mètode de Gauss:
	$$
	\left(\begin{array}{rrr|rrr}
	1 & 2 & 6 & 1 & 0 & 0\\
	0 & -1 & -8 & 0 & 1 & 0\\
	0 & -4 & -30 & -5 & 0 & 1
	\end{array}\right)
	\rightsquigarrow
	\left(\begin{array}{rrr|rrr}
	1 & 2 & 6 & 1 & 0 & 0\\
	0 & 1 & 8 & 0 & -1 & 0\\
	0 & -4 & -30 & -5 & 0 & 1
	\end{array}\right)
	$$
	$$
	\left(\begin{array}{rrr|rrr}
	1 & 0 & -10  & 1 & 2 & 0\\
	0 &1 & 8 & 0 & -1 & 0\\
	0 & 0 & 2 & -5 & -4 & 1
	\end{array}\right)
	\rightsquigarrow
	\left(\begin{array}{rrr|rrr}
	1 & 0 & 6 & 1 & 0 & 0\\
	0 & 1 & 8 & 0 & -1 & 0\\
	0 & 0 & 1 & -5/2 & -2 & 1/2
	\end{array}\right)
	$$
	$$
	\rightsquigarrow\left(\begin{array}{rrr|rrr}
	1 & 0 & 0 & -24 & -18 & 5\\
	0 & 1 & 0 & 20 & 15 & -4\\
	0 & 0 & 1 & -5/2 & -2 & 1/2
	\end{array}\right)
	$$
	Per tant:
	$$
	A^{-1}=\begin{pmatrix*}[r]
	-24 & -18 & 5 \\
	20 & 15 & -4 \\
	-5/2 & -2 & 1/2
	\end{pmatrix*}
	$$
	
\end{exemple}
\begin{exercici}
	Tot el que hem fet per files té el seu anàleg per columnes. Una manera senzilla d'adaptar totes les definicions i resultats és dir que la transposada compleix la definició per files. Per exemple:
	\begin{quote}
		La matriu $A$ és \emph{en forma reduïda per columnes} si $A^T$ és en forma reduïda per files.
	\end{quote}
	\begin{enumerate}[(a)]
		\item Enuncieu l'anàleg per columnes de cada resultat que s'ha vist a aquest capítol per files.
		\item Demostreu que $\Rang(A)=\Rang(A^T)$.
	\end{enumerate}
\end{exercici}
\begin{exercici}
	Considerem $A\in M_{m\times n}(\K)$ i hi afegim una fila, obtenint $A'\in M_{(m+1)\times n}(\K)$. Demostreu $\Rang(A)=\Rang(A')$ si i només si la fila que hem afegit és combinació lineal de les files d'$A$.
\end{exercici}
\begin{exercici}
	Demostreu que el rang d'una matriu $A$ és el nombre màxim de files (o columnes) linealment independents que conté $A$.
\end{exercici}
\subsection{Resolució de sistemes d'equacions lineals}
Recordem la notació d'un sistema d'equacions:
\begin{align*}
a_{11}x_1+a_{12}x_2+ \cdots + a_{1n}x_n &= b_1 \\
a_{21}x_1+a_{22}x_2+ \cdots + a_{2n}x_n &= b_2 \\
&\vdots \\
a_{m1}x_1+a_{m2}x_2+ \cdots + a_{mn}x_n &= b_m
\end{align*}
Que també escrivim com $AX=B$, on
\[
A=
\begin{pmatrix*}[r]
a_{11} & a_{12} & \cdots & a_{1n} \\
a_{21} & a_{22} & \cdots & a_{2n} \\
\vdots & \vdots & \ddots & \vdots \\
a_{m1} & a_{m2} & \cdots & a_{mn} 
\end{pmatrix*}\text{, }
X=
\begin{pmatrix}
x_1 \\ x_2 \\ \vdots \\ x_n
\end{pmatrix} \text{ i }
B=
\begin{pmatrix}
b_1 \\ b_2 \\ \vdots \\ b_m
\end{pmatrix}
\]
Volem esbrinar si el sistema té solució o no, i, en cas de tenir-ne, saber quantes en té i calcular-les.

Una primera interpretació és considerar que tenim una solució $X$ i fer el càlcul següent:
\[
\begin{pmatrix}
b_1 \\ b_2 \\ \vdots \\ b_m
\end{pmatrix} = B = AX =
x_1 \begin{pmatrix}
a_{11} \\ a_{21} \\ \vdots \\ a_{m1}
\end{pmatrix} +
x_2 \begin{pmatrix}
a_{12} \\ a_{22} \\ \vdots \\ a_{m2}
\end{pmatrix} + \cdots +
x_n \begin{pmatrix}
a_{1n} \\ a_{2n} \\ \vdots \\ a_{mn}
\end{pmatrix}
\]
Per tant, una solució ens dóna $B$ com a combinació lineal de les columnes d'$A$, i al revés: si podem escriure $B$ com a combinació lineal de les columnes d'$A$, tenim una solució.

Argumentant amb el rang per columnes, ja tenim un criteri per saber si un sistema d'equacions lineals té solució o no:
\begin{proposicio}\index{sistema d'equacions!existència de solució}
	El sistema d'equacions:
	\begin{align*}
	a_{11}x_1+a_{12}x_2+ \cdots + a_{1n}x_n &= b_1 \\
	a_{21}x_1+a_{22}x_2+ \cdots + a_{2n}x_n &= b_2 \\
	&\vdots \\
	a_{m1}x_1+a_{m2}x_2+ \cdots + a_{mn}x_n &= b_m
	\end{align*}
	té solució si i només si:
	\[\Rang \begin{pmatrix*}[r]
	a_{11} & a_{12} & \cdots & a_{1n} \\
	a_{21} & a_{22} & \cdots & a_{2n} \\
	\vdots & \vdots & \ddots & \vdots \\
	a_{m1} & a_{m2} & \cdots & a_{mn} 
	\end{pmatrix*} =
	\Rang \begin{pmatrix*}[r]
	a_{11} & a_{12} & \cdots & a_{1n} & b_1 \\
	a_{21} & a_{22} & \cdots & a_{2n} & b_2 \\
	\vdots & \vdots & \ddots & \vdots & \vdots \\
	a_{m1} & a_{m2} & \cdots & a_{mn} & b_m 
	\end{pmatrix*}
	\]
\end{proposicio}
\begin{proof}
	Denotem per $A$ la matriu associada al sistema i per $\overline{A}$ la matriu ampliada.
	Esglaonem per files la matriu $\overline{A}$ fins a tenir una matriu reduïda $P\overline{A}$ i fem exactament les mateixes transformacions a la matriu $A$, obtenint $PA$.
	Com que les primeres $n$ columnes d'$\overline{A}$ són precisament les d'$A$, tenim que $PA$ també és una matriu en forma reduïda, i tants sols difereixen de que $P\overline{A}$ té una columna més. 
	
	Si $\Rang(A)=\Rang(\overline{A})$, vol dir que a l'última columna de $P\overline{A}$ no hi ha cap pivot, i per tant es pot escriure l'última columna de $P\overline{A}$ com a combinació lineal de les $n$ primeres, obtenint una solució del sistema.
	
	Si $\Rang(A)\neq\Rang(\overline{A})$, per força $\Rang(\overline{A})=\Rang(A)+1$ i l'última columna té un pivot. La fila on hi ha el pivot de l'última columna correspon a l'equació $0=1$, que no té solució.
\end{proof}
El resultat anterior també es pot enunciar dient que si esglaonem per files la matriu ampliada del sistema, no queda cap fila on l'únic element no nul sigui a la columna de termes independents.

Com que les transformacions elementals per files a la matriu ampliada no modifiquen les solucions del sistema, suposem que hem esglaonat la matriu inicial del sistema. Llavors, obtenint una matriu reduïda $\overline{A}'$:
\begin{enumerate}[(a)]
	\item El sistema té solució si i només si l'última columna d'$\overline{A}'$ no té cap pivot. Si el sistema té solució, diem que el \emph{sistema és compatible}, i quan no té solució, diem que és un \emph{sistema incompatible}.
	\item Suposem que el sistema té solució, llavors cada pivot ens permet aïllar la variable corresponent a la seva columna.
	\item Continuem suposant que el sistema té solució: les columnes (de la matriu sense ampliar) que no tenen pivot definiran el que anomenem \emph{paràmetres lliures}\index{sistema d'equacions!paràmetres lliures}. N'hi haurà tants com $k:=n-\Rang(A)$, on $n$ és el número de incògnites. En aquest cas direm que \emph{la solució té dimensió $k$}.\\
	En el cas particular que $k=0$ diem que té solució única i diem que és un \emph{sistema compatible determinat}.\\
	Si $k>0$ diem que és un \emph{sistema compatible indeterminat amb $k$ paràmetres lliures}.
\end{enumerate}
\begin{exemple}\label{exem:SCI}
	Considerem el sistema d'equacions:
	
	\begin{align*}
	x - y + 2z + 3t &= 21 \\
	-x+2y+z+5t &= 26\\
	3x+y-2z+t &= -9\\
	3x+2y+z+9t &= 38
	\end{align*}
	
	Considerem la matriu ampliada i esglaonem:
	\[
	\begin{amatrix}{4}
	1 & -1 & 2 & 3 & 21 \\
	-1 & 2 & 1 & 5 & 26 \\
	3 & 1 & -2 & 1 & -9\\
	3 & 2 & 1 & 9 & 38
	\end{amatrix}
	\rightsquigarrow
	\begin{amatrix}{4}
	1 & -1 & 2 & 3 & 21 \\
	0 & 1 & 3 & 8 & 47 \\
	0 & 4 & -8 & -8 & -72\\
	0 & 5 & -5 & 0 & -25
	\end{amatrix}
	\]
	\[
	\begin{amatrix}{4}
	1 & 0 & 5 & 11 & 68 \\
	0 & 1 & 3 & 8 & 47 \\
	0 & 0 & -20 & -40 & -260\\
	0 & 0 & -20 & -40 & -260
	\end{amatrix}\rightsquigarrow
	\begin{amatrix}{4}
	1 & 0 & 5 & 11 & 68 \\
	0 & 1 & 3 & 8 & 47 \\
	0 & 0 & -20 & -40 & -260\\
	0 & 0 & 0 & 0 & 0
	\end{amatrix}
	\]
	\[
	\begin{amatrix}{4}
	1 & 0 & 5 & 11 & 68 \\
	0 & 1 & 3 & 8 & 47 \\
	0 & 0 & 1 & 2 & 13\\
	0 & 0 & 0 &  0 & 0
	\end{amatrix}
	\rightsquigarrow
	\begin{amatrix}{4}
	1 & 0 & 0 & 1 & 3 \\
	0 & 1 & 0 & 2 & 8 \\
	0 & 0 & 1 & 2 & 13\\
	0 & 0 & 0 &  0 & 0
	\end{amatrix}
	\]
	Com que el rang de la matriu associada és 3 i el de l'ampliada també, el sistema és compatible. Com que tenim 4 incògnites, té 4-3=1 paràmetre lliures (per tant, sistema compatible indeterminat). Amb l'esglaonament que hem fet, la $t$ és el paràmetre lliure i podem escriure la solució:
	\[
	\begin{pmatrix}
	x \\ y \\ z
	\end{pmatrix} =
	\begin{pmatrix}
	3 \\ 8 \\ 13
	\end{pmatrix}
	-t
	\begin{pmatrix}
	1 \\ 2 \\ 2
	\end{pmatrix} \text{ amb $t\in\K$.}
	\]
\end{exemple}
\begin{exemple}
	Per tal d'aprofitar els càlculs del sistema anterior, tant sols fem una petita modificació al l'últim terme independent:
	\begin{align*}
	x - y + 2z + 3t &= 21 \\
	-x+2y+z+5t&=26\\
	3x+y-2z+t&=-9\\
	3x+2y+z+9t&=39
	\end{align*}
	Considerem la matriu ampliada i esglaonem (són els mateixos passos d'abans, pel que tant sols escrivim la primera i última matriu):
	\[
	\begin{amatrix}{4}
	1 & -1 & 2 & 3 & 21 \\
	-1 & 2 & 1 & 5 & 26 \\
	3 & 1 & -2 & 1 & -9\\
	3 & 2 & 1 & 9 & 38
	\end{amatrix}
	\rightsquigarrow	
	\begin{amatrix}{4}
	1 & 0 & 0 & 1 & 3 \\
	0 & 1 & 0 & 2 & 8 \\
	0 & 0 & 1 & 2 & 13\\
	0 & 0 & 0 &  0 & 1
	\end{amatrix}
	\]
	si som estrictes en el concepte de reduïda, falta utilitzar l'$1$ de l'última fila per a posar zeros a l'última columna. En qualsevol cas, veiem que la matriu ampliada té rang 4, mentre que l'associada té rang 3, pel que el sistema és incompatible.
\end{exemple}

Considerem ara el cas particular en que $B=\0_m$ (un vector format per zeros).
\begin{definicio}
	\begin{enumerate}[(a)]
		\item Diem que el sistema d'equacions lineals $AX=B$ \emph{és homogeni}\index{sistema d'equacions!sistema homogeni} si $B=\0_m$.
		\item Si $AX=B$ és un sistema, parlem de \emph{sistema homogeni associat}\index{sistema d'equacions!sistema homogeni associat} al sistema $AX=\0_m$.
	\end{enumerate}
\end{definicio}
Tenim els resultats següents:
\begin{enumerate}[(a)]
	\item Si el sistema és homogeni, llavors $X=\0_n$ és una solució, per tant el sistema és compatible.
	\item Si $X$ és solució del sistema homogeni i $\lambda\in\K$, llavors $\lambda X$ també és solució del sistema: si ho pensem com a multiplicació de matrius, tenim la igualtat $A(\lambda X)=\lambda (AX)=\lambda \0_m=\0_m$, per tant $\lambda X$ també és solució.
	\item Si $X$ i $Y$ són solucions d'un sistema homogeni, llavors $X+Y$ també és solució: tornem a escriure-ho en forma matricial: $A(X+Y)=AX+AY=\0_m+\0_m=\0_m$.
	\item Si $AX=B$ és un sistema, amb $X$ i $Y$ una solucions, llavors $X-Y$ és una solució del sistema homogeni associat: $A(X-Y)=AX-AY=B-B=\0_m$.
	\item Si $AX=B$ és un sistema i $X$ una solució particular, qualsevol altre solució $Y$ es pot escriure com $Y=X+Z$, amb $Z$ solució del sistema homogeni associat: això es dedueix de l'apartat anterior: si $X$ i $Y$ són solucions, llavors $Z=Y-X$ és solució de l'homogeni. Si $X$ és solució del sistema i $Z$ de l'homogeni associat, llavors $AY=A(X+Z)=AX+AZ=B+\0_m=B$.
\end{enumerate}
\begin{exemple}
	Considerem el sistema d'equacions de l'Exemple \ref{exem:SCI}:
	\begin{align*}
	x + y + 2z + 3t &= 21 \\
	-x+2y+z+5t&=26\\
	3x+y-2z+t&=-9\\
	3x+2y+z+9t&=38
	\end{align*}
	Veiem que el sistema homogeni associat és
	\begin{align*}
	x + y + 2z + 3t &= 0\\
	-x+2y+z+5t&=0\\
	3x+y-2z+t&=0\\
	3x+2y+z+9t&=0
	\end{align*}
	i té per solució
	\[
	\begin{pmatrix}
	x \\ y \\ z
	\end{pmatrix} =
	t \begin{pmatrix}
	1 \\ 2 \\ 2
	\end{pmatrix} \text{ amb $t\in\K$.}
	\]
	Tenint en compte que una solució particular és $(x,y,z)=(3,8,13)$, podem escriure:
	\[
	\begin{pmatrix}
	x \\ y \\ z
	\end{pmatrix} =
	\begin{pmatrix}
	3 \\ 8 \\ 13
	\end{pmatrix}-
	t \begin{pmatrix}
	1 \\ 2 \\ 2
	\end{pmatrix} \text{ amb $t\in\K$.}
	\]
\end{exemple}
 
 \begin{llista-exercicis}
\item[Secció 1.1:] 6, 12, 16.
\item[Secció 1.2:] 10, 18.
\item[Secció 1.3:] 6, 8, 18, 20, 24, 28.
 \end{llista-exercicis}


\input{2_Esp_vec_apl_lin.tex}
\input{3_Diagonalitzacio.tex}
% !TeX encoding = UTF-8
% !TeX spellcheck = ca_ES-valencia
% !TeX root = MatCADAlgLin.tex


Podeu trobar el contingut de la part d'ortogonalitat a \cite[Tema 5]{Bret}, i de la part de formes quadràtiques a (\cite[Tema~8]{Bret},\cite[Tema~4]{NaXa}).
\subsection{Ortogonalitat a \texorpdfstring{$\R^n$}{Rn}}
Considerem l'espai vectorial $\R^n$ i recordem el producte escalar definit com: si $\vec u=\smat{u_1\\u_2\\ \vdots\\u_n}$ i $\vec v=\smat{v_1\\v_2\\ \vdots \\ v_n}$ són vectors d'$\R^n$, llavors:
$$
\vec u \cdot \vec v = \vec u^T \vec v=\sum_{i=1}^n u_iv_i \,.
$$
\begin{definicio}
\begin{itemize}
    \item Diem que dos vectors $\vec u$ i $\vec v$ són \emph{ortogonals} si $\vec u\cdot\vec v=0$.
    \item Definim la \emph{longitud d'un vector}:
    $$
    ||\vec u||=\sqrt{\vec u\cdot \vec u}=\sqrt{\sum_{i=1}^n u_i^2}.
    $$
    \item Diem que un vector $\vec u\in\R^n$ és \emph{unitari} si $||\vec u||=1$.
    \item Diem que els vectors $\vec u_1, \dots, \vec u_k$ de $\R^n$ \emph{són ortogonals}\index{ortogonals} si són ortogonals dos a dos, o sigui, si $\vec u_i\cdot\vec u_j=0$ per a tot $i\neq j$.
    \item Diem que els vectors $\vec u_1, \dots, \vec u_k$ de $\R^n$ \emph{són ortonormals}\index{ortonormals} si són unitaris i ortogonals dos a dos, o sigui, si 
    $$\vec u_i\cdot\vec u_j=\left\{ \begin{array}{ll} 1 & \text{si $i=j$} \\ 0 & \text{si $i\neq j$}\end{array}\right.$$
\end{itemize}
\end{definicio}
\begin{observacio}
Podem passar de vectors no nuls $\vec u_1, \dots, \vec u_k$ de $\R^n$ ortogonals a ortonormals dividint cadascun per la seva norma: si $\vec u_1, \dots, \vec u_k$ de $\R^n$ són no nuls, llavors $\frac{\vec u_1}{||\vec u_1||}, \dots, \frac{\vec u_k}{||\vec u_k||}$ són unitaris, i si $\vec u_i\cdot \vec u_j=0$, llavors, $\frac{\vec u_i}{||\vec u_i||}\cdot \frac{\vec u_j}{||\vec u_j||}=\frac{1}{||\vec u_i|| ||\vec u_j||}(\vec u_i\cdot \vec u_j)=0$.
\end{observacio}
\begin{exemple}
La base estàndard d'$\R^n$ formada pels vectors $\vec e_i$ amb un $1$ a la posició $i$ i $0$ a la resta de posicions és una base ortonormal.
\end{exemple}
\begin{lema}
Si els vectors no nuls $\vec u_1, \dots, \vec u_k$ de $\R^n$ són ortogonals, llavors són linealment independents.\\
En particular, si tenim $n$ vectors $\vec u_1, \dots, \vec u_n$ de $\R^n$ ortonormals, formen una base.
\end{lema}
\begin{proof}
Suposem que tenim una combinació lineal dels vectors $\vec u_1, \dots, \vec u_k$ que dóna zero:
$$
0 = \lambda_1 \vec u_1 + \dots + \lambda_n\vec u_n \,.
$$
Llavors, fent el producte escalar per $\vec u_i$ tenim:
\begin{align*}
    0 &  = (\lambda_1 \vec u_1 + \dots + \lambda_n\vec u_n)\cdot \vec u_i= \\
     & = \lambda_1 (\vec u_1 \cdot \vec u_i) + \dots + \lambda_n(\vec u_n\cdot \vec u_i)=\\
     & = \lambda_i ||\vec u_i||^2\quad \text{(tots els altres productes escalars són zero)}
\end{align*}
Per tant, com que $\vec u_i\neq\vec 0$, $\lambda_i=0$ i són linealment independents.

En el cas de tenir $n$ vectors ortonormals, són $n$ vectors ortogonals no nuls, per tant linealment independents i com que la dimensió de $\R^n$ és $n$, són base (Teorema~\ref{teo:baseV}).
\end{proof}


Veurem més endavant com trobar una base ortonormal d'un subespai $V$ de $\R^n$. Suposem però que tenim $[\vec u_1,\ldots,\vec u_k]$ una base ortonormal de $V$, i considerem la \emph{projecció ortogonal}\index{projecció ortogonal} $\proj_V\colon \R^n\rightarrow \R^n$, que envia un vector $\vec v\in \R^n$ al vector $\proj_V(\vec v) = \vec v^\parallel$, on
\[
\vec v^\parallel = (\vec v\cdot \vec u_1)\vec u_1 + \cdots + (\vec v\cdot \vec u_k)\vec u_k \in V.
\]
Observem que $\proj_V\colon \R^n\rightarrow \R^n$ és una aplicació lineal (per les propietats de linealitat del producte escalar). Com que la seva imatge està continguda a $V$ i $\proj_V(\vec u_i) = \vec u_i$, en deduïm que $\Ima(\proj_V)=V$. A més,
\begin{align*}
\Ker(\proj_V) &= \{\vec w\in\R^n ~|~ \vec w^\parallel = 0\}\\
              &= \{\vec w\in\R^n ~|~ \vec w\cdot \vec u_i = 0, \quad i=1,\ldots k\}\\
              &= \{\vec w\in\R^n ~|~ \vec w\cdot \vec v = 0\quad\text{ per a tot }\vec w \in V\}.
\end{align*}
Donarem un nom al nucli de $\proj_V$:

\begin{definicio}
Donat un subespai $V\subseteq \R^n$, el \emph{complement ortogonal}\index{complement ortogonal} de $V$, que escriurem $V^\perp$, és el conjunt
\[
V^\perp = \{\vec w\in \R^n ~|~ \vec w\cdot \vec v = 0\text{ per a tot } \vec v\in V\}.
\]
\end{definicio}

\begin{proposicio}
Sigui $V\subseteq \R^n$ un subespai d'$\R^n$. Aleshores:
\begin{enumerate}[\rm (a)]
    \item $V^\perp$ també és un subespai d'$\R^n$,
    \item $V \cap V^\perp = \{\vec 0\}$,
    \item $\dim(V)+\dim(V^\perp) = n$,
    \item $(V^\perp)^\perp = V$.
\end{enumerate}
\end{proposicio}
\begin{proof}
La primera afirmació es comprova de manera directa, fent servir les propietats del producte escalar.

Per veure la segona, sigui $\vec w\in V\cap V^\perp$. Per tant, $\vec w\cdot\vec w =\| \vec w\|= 0$. Però l'únic vector de longitud $0$ és el vector nul.

Seguidament, observem que $V=\Ima(\proj_V)$ i que $V^\perp = \ker(\proj_V)$. Per tant, per la fórmula del nucli--imatge (Teorema~\ref{teo:ker+ima}) tenim
\[
\dim(\ker\proj_V) + \dim(\Ima\proj_V) = \dim(\R^n)=n,
\]
com volíem veure.

Finalment, per veure que $(V^\perp)^\perp = V$, denotem $W=V^\perp$. Aleshores, si $\vec v\in V$, i $\vec w\in W$, es té que $\vec v\cdot \vec w = 0$, i per tant $\vec v\in W^\perp$. Concloem doncs que $V\subseteq (V^\perp)^\perp$. Però ara observem que $\dim((V^\perp)^\perp) = n - \dim(V^\perp) = n- (n-\dim(V)) = \dim(V)$. Veiem que $(V^\perp)^\perp$ és un subespai que conté $V$ i que té la mateixa dimensió que $V$ i, per tant, ha de coincidir amb $V$.
\end{proof}
\begin{corollari}\label{cor:V+Vper}
Si $V\subseteq\R^n$ és un subespai d'$\R^n$, aleshores $\R^n=V\oplus V^\perp$.
\end{corollari}
\begin{proof}
Per l'apartat (b) de la proposició anterior, la intersecció és buida. Per l'apartat (c), la suma de de les dimensions és $n$. Per tant, si prenem una base $\calb$ de $V$ i una base $\calc$ de $V^\perp$, el conjunt $\calb\cup\calc$ serà linealment independent i tindrà $n$ elements. Per tant, serà base de $\R^n$, que és el que volíem veure.
\end{proof}
D'aquest corol·lari es dedueix que, donat $V$ un subespai d'$\R^n$, tot vector $\vec w\in\R^n$ es pot escriure de forma única com:
\[
\vec w = \vec w^\parallel + \vec w^\perp
\]
amb $\vec w^\parallel \in V$ i $\vec w^\perp \in V^\perp$. A més, tenim que $\vec w^\perp=\proj_{V^\perp}(\vec w)$.

\subsection{El teorema de Pitàgores}
El conegut teorema de Pitàgores es pot formular a $\R^n$. Fixem-nos que el teorema té dues implicacions.
\begin{teorema}
Si $\vec v$ i $\vec w$ són dos vectors d'$\R^n$, aleshores:
\[
\| \vec v + \vec w \|^2 = \|\vec v\|^2 + \|\vec  w\|^2 \iff \vec v \perp \vec w.
\]
\end{teorema}
\begin{proof}
Calculem:
\[
\| \vec v + \vec w\|^2 = (\vec v+\vec w)\cdot (\vec v+\vec w) = \vec v\cdot \vec v + \vec v\cdot \vec w + \vec w\cdot \vec v + \vec w\cdot \vec w=\|\vec v\|^2 + \|\vec w\|^2 +2(\vec v\cdot \vec w). 
\]
Per tant, la igualtat es compleix si i només si $\vec v\cdot \vec w = 0$.
\end{proof}
\begin{corollari}
Sigui $V\subseteq \R^n$ un subespai. Aleshores
\[
\|\proj_V(\vec w)\| \leq \| \vec w\| \text{ per a tot $\vec w\in\R^n$,}
\]
i la desigualtat és una igualtat si i només si $\vec w\in V$.
\end{corollari}
\begin{proof}
Apliquem el teorema anterior a $\vec w^\parallel$ i $\vec w^\perp$ (notem $\vec w = \vec w^\parallel + \vec w^\perp$).
\[
\|\vec w\|^2 = \|\vec w^\parallel\|^2 + \|\vec w^\perp\|^2,
\]
i per tant
\[
\|\proj_V(\vec w)\|^2 \leq \|\vec w\|^2,
\]
amb igualtat si i només si $\|\vec w^\perp\|^2 = 0$, és a dir si i només si $\vec w^\perp = 0$, que és equivalent a $\vec w\in V$.
\end{proof}

Una aplicació d'aquests resultats és una desigualtat molt famosa, coneguda com la \emph{desigualtat de Cauchy--Schwarz}\index{desigualtat de Cauchy--Schwarz}:
\begin{teorema}
Si $\vec v$ i $\vec w$ són vectors de $\R^n$, aleshores
\[
|\vec v\cdot \vec w|\leq \|\vec v\|\|\vec w\|,
\]
amb igualtat si i només si $\vec v$ i $\vec w$ són paral·lels.
\end{teorema}
\begin{proof}
Si $\vec v=\vec 0$, aleshores la desigualtat es compleix trivialment. En cas contrari, considerem el subespai $V=\langle \vec v\rangle$, i apliquem el resultat anterior. Ja havíem vist la fórmula
\[
\proj_V(\vec w) = \frac{\vec w\cdot \vec v}{\|\vec v\|^2} \vec v,
\]
i per tant
\[
\|\proj_V(\vec w)\| = \frac{|\vec w\cdot \vec v|}{\|\vec v\|^2} \|\vec v\|,
\]
i obtenim
\[
\frac{|\vec w\cdot \vec v|}{\|\vec v\|} \leq \|\vec w\|,
\]
que és la desigualtat que busquem un cop passem $\|\vec v\|$ a l'altra banda (i utilitzem que $\vec w\cdot \vec v=\vec v\cdot \vec w$). La igualtat es dóna si i només si $\vec w\in V$, que és equivalent a $\vec v$ essent paral·lel a $\vec w$.
\end{proof}

Observem que si $\vec v$ i $\vec w$ no són zero, aleshores tenim les desigualtats
\[
-1 \leq \frac{\vec v\cdot \vec w}{\|\vec v\|\|\vec w\|}\leq 1.
\]
\begin{definicio}
Donats dos vectors no-nuls $\vec v$ i $\vec w$ d'$\R^n$, \emph{l'angle entre $\vec v$ i $\vec w$}\index{angle entre dos vectors} és
\[
\theta = \arccos\left(\frac{\vec v\cdot \vec w}{\|\vec v\|\|\vec w\|}\right).
\]
\end{definicio}

\subsection{Mètode de Gram-Schmidt}
Aquest mètode permet calcular una base ortonormal d'un subespai de $\R^n$ a partir de projeccions ortogonals. Fixem les notacions següents:
\begin{itemize}
    \item $V\subset \R^n$ un subespai de dimensió $k$ fixat,
    \item $\calb =[\vec v_1, \dots , \vec v_k]$ una base de $V$,
    \item $V_i=\langle \vec v_1, \dots , \vec v_i\rangle \subset V$ el subespai (de dimensió $i$) generat pels $i$ primers vectors de $\calb$. Tenim que $V_1\subset V_2\subset \cdots \subset V_k=V$.
    \item Si $\vec v_i\in\calb$, denotarem per $\vec v_i^\parallel$ la projecció ortogonal de $\vec v_i$ a $V_{i-1}$ i per $\vec v_i^\perp$ el vector  $\vec v_i - \vec v_i^\parallel$, que és ortogonal a $V_{i-1}$.
\end{itemize}
El mètode de Gram-Schmidt\index{Gram-Schmidt} construeix de forma iterativa una base ortonormal a cada $V_i$:
\begin{itemize}
    \item Considerem primer $V_1=\langle \vec v_1 \rangle$, i una base és $\calb_1=[\vec v_1]$. En aquest cas, definim 
    \[\vec u_1=\frac{1}{||\vec v_1||} \vec v_1\]
    i tenim que $\calc_1=[\vec u_1]$ és una base ortonormal de $V_1$.
    \item Considerem ara $V_2=\langle \vec v_1, \vec v_2 \rangle=\langle \vec u_1, \vec v_2 \rangle$. Pel Corol·lari \ref{cor:V+Vper}, $\vec v_2$ es pot escriure de forma única com:
    \[
    \vec v_2 = \vec v_2^\parallel + \vec v_2^\perp
    \]
    amb $\vec v_2^\parallel\in V_1$ i $\vec v_2\perp \in V_1^\perp$. A més, sabem com calcular-los:
    \[
    \vec v_2^\parallel = \proj_{V_1}(\vec v_2)=(\vec v_2 \cdot \vec u_1)\vec u_1
    \]
    i per tant
    \[
    \vec v_2^\perp = \vec v_2 - \vec v_2^\parallel = \vec v_2 - (\vec v_2 \cdot \vec u_1)\vec u_1.
    \]
    Definim $\vec u_2=\frac{1}{||\vec v_2^\perp||}\vec v_2^\perp$, i tenim que $V_2=\langle \vec u_1, \vec u_2\rangle$, amb $\calc_2=[\vec u_1,\vec u_2]$ una base ortonormal.
    \item Suposem ara que ja tenim $\calc_{i-1}=[\vec u_1, \dots, \vec u_{i-1}]$ una base ortonormal de $V_{i-1}$. Considerem $V_i=\langle\vec v_1, \dots, \vec v_{i-1},\vec v_i\rangle=\langle\vec u_1, \dots, \vec u_{i-1},\vec v_i\rangle$, i escrivim:
    \[
    \vec v_i = \vec v_i^\parallel + \vec v_i^\perp
    \]
    amb $\vec v_i^\parallel\in V_{i-1}$ i $\vec v_i\perp \in V_{i-1}^\perp$. Com que els vectors $\vec u_i$ són ortonormals, el càlcul és:
    \[
    \vec v_i^\parallel = \proj_{V_{i-1}}(\vec v_i)=(\vec v_i \cdot \vec u_1)\vec u_1+ \cdots + (\vec v_i \cdot \vec u_{i-1})\vec u_{i-1}
    \]
    i per tant
    \[
    \vec v_i^\perp = \vec v_i - \vec v_i^\parallel = \vec v_i - (\vec v_i \cdot \vec u_1)\vec u_1- \cdots -(\vec v_i \cdot \vec u_{i-1})\vec u_{i-1}.
    \]
    Definim $\vec u_i=\frac{1}{||\vec v_i^\perp||}\vec v_i^\perp$, i tenim que $V_i=\langle \vec u_1, \dots ,\vec u_i\rangle$, amb $\calc_i=[\vec u_1,\dots,\vec u_i]$ una base ortonormal.
\end{itemize}
\begin{exemple}\label{exemple:GramSchmidt}
Considerem $V=\langle \smat{1\\-1\\-1\\1}, \smat{0\\1\\1\\0}\rangle \subset \R^4$ i volem calcular una base ortonormal.

Comencem amb la base $\calb=[\smat{1\\-1\\-1\\1}, \smat{0\\1\\1\\0}]$ i, seguint el procediment, $V_1=\langle \smat{1\\-1\\-1\\1}\rangle$ i tant sols hem de fer-lo unitari:$||\smat{1\\-1\\-1\\1}||=2$ i per tant $V_1=\langle \smat{1\\-1\\-1\\1}\rangle=\langle \smat{1/2\\-1/2\\-1/2\\1/2}\rangle$.

Calculem ara el segon vector:
\[
\vec v_2^\perp = \smat{0\\1\\1\\0} - (\smat{0\\1\\1\\0}\cdot\smat{1/2\\-1/2\\-1/2\\1/2})\smat{1/2\\-1/2\\-1/2\\1/2}= \smat{0\\1\\1\\0}+\smat{1/2\\-1/2\\-1/2\\1/2}=\smat{1/2\\1/2\\1/2\\1/2}.
\]
Llavors $\vec u_2=\frac{1}{||\vec v_2^\perp||} \vec v_2^\perp=\smat{1/2\\1/2\\1/2\\1/2}$.

Tenim, doncs, que $\calc=[\smat{1/2\\-1/2\\-1/2\\1/2},\smat{1/2\\1/2\\1/2\\1/2}]$ és una base ortonormal de $V$.

A més, la matriu del canvi de base és:
\[
S_{\calb,\calc}=\begin{pmatrix} 2 & -1 \\ 0 & 1\end{pmatrix}
\]
\end{exemple}

L'algorisme de Gram-Schmidt ens porta a la factorització $QR$ d'una matriu $A$ (amb les seves columnes linealment independents).
\begin{teorema}
Donada una matriu $A\in M_{m\times n}(\R)$ amb les seves columnes linealment independents, existeixen una matriu $Q\in M_{m\times n}(\R)$ i $R\in M_{n\times n}$ tals que:
\begin{itemize}
    \item $A=QR$,
    \item $Q^T Q=\1_n$ (diem que $Q$ és una matriu ortogonal\index{matriu!ortogonal}),
    \item $R$ és una matriu triangular superior amb els coeficients de la diagonal positius.
\end{itemize}
A més, aquesta factorització és única.
\end{teorema}
\begin{proof}
La demostració és constructiva: anirem calculant els coeficients de les columnes $Q$ i $R$ aplicant el mètode de Gram-Schmidt. Fixem abans les notacions següents per a les columnes d'$A$, de $Q$ i els coeficients de $R$:
\[
A=\begin{pmatrix} | & | & & | \\ \vec v_1 & \vec v_2 & \cdots & \vec v_n \\ | & | & & | \end{pmatrix} ,
Q=\begin{pmatrix} | & | & & | \\ \vec u_1 & \vec u_2 & \cdots & \vec u_n \\ | & | & & | \end{pmatrix} 
\text{ i }
R=\begin{pmatrix} r_{11} & r_{12} & \cdots & r_{1n} \\ 0 & r_{22} & \cdots & r_{2n} \\ \vdots & \vdots & \ddots & \vdots \\ 0 & 0 & \cdots & r_{nn}\end{pmatrix}.
\]
I les matrius formades per les primeres $j$ columnes d'$A$, $Q$ i la submatriu quadrada $j\times j$ d'$R$:
\[
A_j=\begin{pmatrix} | & | & & | \\ \vec v_1 & \vec v_2 & \cdots & \vec v_j\\ | & | & & | \end{pmatrix} ,
Q_j=\begin{pmatrix} | & | & & | \\ \vec u_1 & \vec u_2 & \cdots & \vec u_j \\ | & | & & | \end{pmatrix} 
\text{ i }
R_j=\begin{pmatrix} r_{11} & r_{12} & \cdots & r_{1j} \\ 0 & r_{22} & \cdots & r_{2j} \\ \vdots & \vdots & \ddots & \vdots \\ 0 & 0 & \cdots & r_{jj}\end{pmatrix}.
\]
La construcció comença amb $j=1$, $j=2$ i veurem com es fa el cas $j$ a partir del $j-1$:
\begin{itemize}
    \item Cas $j=1$: llavors $A_1=\vec v_1$: calculem el coeficient $R_1=(r_{11})=(||\vec v_1||)$ i la primera columna de $Q$ com $Q_1=\vec u_1=\frac{1}{r_{11}}\vec v_1$. Veiem que, de moment, es compleix  $A_1=Q_1R_1$. A més, $Q_1^TQ_1=||\vec u_1||=1$ i $R_1$ és triangular superior i amb la diagonal positiva (és un mòdul).
    \item Cas $j=2$: considerem $r_{12}=\vec u_1\cdot \vec v_2$, el vector $\vec v_2^\perp=\vec v_2-r_{12}u_1$, $r_{22}=||\vec v_2^\perp||$ i $\vec u_2=\frac{1}{r_{22}} v_2^\perp$. La primera columna de $Q_2R_2$ és la mateixa que la de $Q_1R_1=A_1$. La segona és
    \[
    r_{12}u_1+r_{22}\vec u_2=(u_1\cdot v_2)\vec u_1+||v_2-(\vec u_1\cdot \vec v_2)\vec u_1||\frac{1}{||\vec v_2-(\vec u_1\cdot \vec v_2)\vec u_1||}(\vec v_2-(\vec u_1\cdot \vec v_2)\vec u_1)=\vec v_2
    \]
    i per tant tenim la segona columna de $A$, llavors $A_2=Q_2R_2$. També veiem que $Q_2^TQ_2=\1_2$ ($u_2$ és perpendicular a $\vec u_1$ i tots dos són unitaris) i $R_2$ és triangular superior i la diagonal positiva (són mòduls de vectors).
    \item Suposem que tenim les $j-1$ primeres columnes de $Q$, i les de $R$ tal que $A_{j-1}=Q_{j-1}R_{j-1}$, $Q_{j-1}^TQ_{j-1}=\1_{j-1}$ i $R_{j-1}$ triangular superior i amb la diagonal positiva. Considerem $r_{ij}=\vec u_i\cdot \vec v_j$ per a $1\leq i <j$, $\vec v_j^\perp=\vec v_j-r_{1j}\vec u_1-\cdots-r_{(j-1)j}\vec u_{j-1}$, $r_{jj}=||\vec v_j^\perp||$ i $\vec u_j=\frac{1}{r_{jj}} \vec v_j^\perp$. Si fem $Q_jR_j$, les primeres $j-1$ columnes són les de $Q_{j-1}R_{j-1}$, per tant són $A_{j-1}$. Si calculem la columna $j$-èssima de $Q_jR_j$:
    \begin{align*}
     r_{1j}\vec u_1+\cdots+r_{jj}\vec u_j & =(\vec u_1\cdot \vec v_j)\vec u_1+\cdots+(\vec u_{j-1}\cdot \vec v_j)\vec u_{j-1}+||\vec v_j^\perp||\frac{1}{||\vec v_j^\perp||} \vec v_j^\perp = \\
     & =(\vec u_1\cdot \vec v_j)\vec u_1+\cdots+(\vec u_{j-1}\cdot \vec v_j)\vec u_{j-1}+ \vec v_j^\perp=\vec v_j .
    \end{align*}
    I per tant $A_j=Q_jR_j$. A més, la primera caixa $(j-1)\times(j-1)$ superior esquerra de $Q_j^TQ_j$ és una $\1_{j-1}$. L'última columna i fila són és $\vec u_i\cdot \vec u_j$, i per tant, com que $\vec u_j$ és perpendicular als altres $\vec u_i$ i unitari, tenim que $Q_j^TQ_j=\1_j$. Finalment, $R_j$ és triangular superior per construcció, i la diagonal formada per mòduls, per tant positius.
\end{itemize}
Falta veure la unicitat, també per inducció sobre $j$:
\begin{itemize}
    \item Cas $j=1$: com que $R$ és triangular superior, $v_1=r_{11}\vec u_1$ amb $||\vec u_1||=1$, per tant $\vec u_1=\frac{\pm1}{||\vec v_1||}v_1$, però com que $r_{11}$ és positiva positiva, $\vec u_1=\frac{1}{||\vec v_1||}\vec v_1$ i $r_{11}=||\vec v_1||$.
    \item Suposem que les $j-1$ primeres columnes de $Q$ i de $R$ ja estan fixades, volem veure que tant sols hi ha una elecció per la $j$-èssima: el vector $\vec u_j$ és un vector perpendicular a l'espai $V_{j-1}=\langle \vec v_1, \dots, \vec v_{j-1}\rangle$ i contingut a $V_j=\langle \vec v_1, \dots, \vec v_j\rangle$. Considerem $V_{j-1}^\perp$ el complement ortogonal de $V_{j-1}$ a $V_j$, que és de dimensió $1$ ($\dim(V_{j-1})+\dim(V_{j-1}^\perp)=\dim(V_j)$), i conté $\vec v_j^\perp$, per tant es té $V_{j-1}^\perp=\langle v_j^\perp \rangle$. Llavors, com que ha de ser unitari, ja $j$-èssima columna de $Q$ ha de ser $\vec u_j=\frac{\pm1}{||\vec v_j^\perp||}\vec v_j^\perp$, però com que $r_{jj}$ és positiva positiva, $\vec u_j=\frac{1}{||\vec v_j^\perp||}\vec v_j^\perp$ i $r_{jj}=||\vec v_j^\perp||$, i tenim la unicitat de la $\vec u_j$ i el coeficient $r_{jj}$. La resta de coeficients de la columna $j$ d'$R$ estan determinats per ser $R$ triangular superior (per tant, per sota de $r_{jj}$ són zero) i el fet de que els de sobre són les coordenades de $\vec v_j-\vec v_j^\perp$ en la base $[\vec u_1,\dots,\vec u_{j-1}]$ de $V_{j-1}$ (i les coordenades en una base són úniques pel Teorema~\ref{teo:base_coord_uniq}).
\end{itemize}
\end{proof}
\begin{exemple}
Podem aprofitar els càlculs fets a l'Exemple \ref{exemple:GramSchmidt} per trobar la factorització $QR$ següent:
\[
\begin{pmatrix}
1 & 0 \\ -1 & 1 \\ -1 & 1 \\ 1 & 0
\end{pmatrix}=
\begin{pmatrix}
1/2 & 1/2 \\ -1/2/ & 1/2 \\ -1/2 & 1/2 \\ 1/2 & 1/2
\end{pmatrix}
\begin{pmatrix}
2 & -1 \\ 0 & 1
\end{pmatrix}.
\]
\end{exemple}

\subsection{Aplicacions i matrius ortogonals}
%Conserva normes \iif conserva producte escalar \iif Q^TQ=\id
\begin{definicio}
Diem que una aplicació lineal $f \colon \R^n\to \R^m$ és \emph{ortogonal}\index{aplicació!ortogonal} si conserva la longitud dels vectors, o sigui, si
\[
||f(\vec v)||=||\vec v|| \text{ per a tot $\vec v\in\R^n$.}
\]
Si $f=f_A$ amb $A\in M_{m\times n}(\R)$, direm que $A$ és una \emph{matriu ortogonal}\index{matriu!ortogonal}.
\end{definicio}

\begin{exemple}
Les rotacions definides a la Secció \ref{subsubsec:rotacio} són ortogonals: es pot interpretar geomètricament o bé fent els càlculs: un vector $\smat{x\\y}$ té norma
\[
||\begin{pmatrix}x \\ y \end{pmatrix}||^2=\begin{pmatrix} x & y \end{pmatrix}\begin{pmatrix} x \\ y \end{pmatrix} = x^2+y^2
\]
i va a parar a:
\[
\begin{pmatrix*}[r]
\cos\theta&-\sin\theta\\
\sin\theta&\cos\theta
\end{pmatrix*}
\begin{pmatrix}x \\ y \end{pmatrix} ,
\]
que té norma
\begin{align*}
||f(\smat{x\\y})||^2 & =\left( \begin{pmatrix*}[r]
\cos\theta&-\sin\theta\\
\sin\theta&\cos\theta
\end{pmatrix*}
\begin{pmatrix}x \\ y \end{pmatrix}  \right)^T
\begin{pmatrix*}[r]
\cos\theta&-\sin\theta\\
\sin\theta&\cos\theta
\end{pmatrix*}
\begin{pmatrix}x \\ y \end{pmatrix} = \\
& = \begin{pmatrix}x &  y \end{pmatrix}
\begin{pmatrix*}[r]
\cos\theta&\sin\theta\\
-\sin\theta&\cos\theta
\end{pmatrix*}
\begin{pmatrix*}[r]
\cos\theta&-\sin\theta\\
\sin\theta&\cos\theta
\end{pmatrix*}
\begin{pmatrix}x \\ y \end{pmatrix} = \\
& = \begin{pmatrix}x &  y \end{pmatrix}
\begin{pmatrix*}[r]
1&0\\
0&1
\end{pmatrix*}
\begin{pmatrix}x \\ y \end{pmatrix} = x^2+y^2\\
\end{align*} 
i per tant la rotació és ortogonal.
\end{exemple}
\begin{exercici}
Demostreu ques reflexions definies a la Secció \ref{subsubsec:reflexio} són ortogonals.
\end{exercici}
\begin{exemple}
Les projeccions a un subespai $V\subset \R^n$ ($V\neq \R^n)$ no són ortogonals: considerem $\vec 0 \neq \vec u \in V^\perp$, llavors $\proj_V(\vec u)=\vec 0$ i per tant no es conserva la longitud de $\vec u$.
\end{exemple}

El lema següent ens permet calcular el producte escalar en funció dels mòduls:
\begin{lema}\label{lema:prod_esc_modul}
Si $\vec u, \vec v$ són vectors d'$\R^n$, llavors:
\[
\vec u \cdot \vec v = \frac{1}{2}\left(||\vec u + \vec v||^2 - ||\vec u||^2 - ||\vec v||^2\right).
\]
\end{lema}
\begin{proof}
Considerem $\vec u, \vec v \in \R^n$. Tenim que:
\begin{align*}
||\vec u + \vec v||^2 & =(\vec u+\vec v)\cdot(\vec u+\vec v) = \vec u \cdot (\vec u+ \vec v) + \vec v \cdot (\vec u+ \vec v) = \\
 & = \vec u \cdot \vec u  + \vec u \cdot \vec v + \vec v \cdot \vec u+\vec v \cdot \vec v = ||\vec u||^2+2 (\vec u \cdot \vec v) + ||\vec v||^2
\end{align*}
I per tant:
\[
\vec u \cdot \vec v = \frac{1}{2}\left(||\vec u + \vec v||^2 - ||\vec u||^2 - ||\vec v||^2\right).
\]
\end{proof}
I d'aquí deduïm:
%\begin{teorema}[Pitàgores]\index{Teorema de Pitàgores}
%Els vectors $\vec u$ i $\vec v$ d'$\R^n$ són ortogonals si i només si
%\[
%||\vec u+\vec v||^2 = ||\vec u||^2+||\vec v||^2 .
%\]
%\end{teorema}
%\begin{proof}
%Es dedueix de la igualtat del Lema \ref{lema:prod_esc_modul}.
%\end{proof}
\begin{teorema}
Una aplicació lineal $f\colon \R^n \to \R^m$ és ortogonal si i només si $f(\vec u)\cdot f(\vec v)=\vec u \cdot \vec v$ per a tot $\vec u$ i $\vec v$ de $\R^n$.
\end{teorema}
\begin{proof}
Si l'aplicació és ortogonal, es compleix que, per a tot $\vec u,\vec v \in \R^n$, $||f(\vec u)||=||\vec u||$ i $||f(\vec v)||=||\vec v||$. Llavors:
\begin{align*}
    f(\vec u)\cdot f\vec v) & = \frac{1}{2}\left(||f(\vec u) +f(\vec v)||^2 - ||f(\vec u)||^2 - ||f(\vec v)||^2\right) = \\
    & = \frac{1}{2}\left(||f(\vec u+\vec v)||^2 - ||f(\vec u)||^2 - ||f(\vec v)||^2\right) = \\
     & = \frac{1}{2}\left(||\vec u + \vec v||^2 - ||\vec u||^2 - ||\vec v||^2\right) = \vec u \cdot \vec v  ,
\end{align*}
on hem utilitzat el Lema \ref{lema:prod_esc_modul} dues vegades i que $f$ és lineal.

Si $f(\vec u)\cdot f(\vec v)=\vec u \cdot \vec v$ per a tot $\vec u$ i $\vec v$ de $\R^n$, com que $||\vec u||^2=\vec u \cdot \vec u$, tenim que, agafant $\vec u=\vec v$, $||f(\vec u)||^2=||\vec u||^2$ i per tant $f$ és ortogonal.
\end{proof}
\begin{corollari}
Si $f = f_A \colon \R^n \to \R^m$ és una aplicació lineal i $\vec e_1, \dots, \vec e_n$ és la base estàndard d'$\R^n$, llavors:
\begin{enumerate}[\rm (a)]
    \item $f$ és ortogonal si i només si $f(\vec e_1), \dots, f(\vec e_n)$ són vectors ortonormals. En particular, si $f$ és ortogonal, $f(\vec e_1), \dots, f(\vec e_n)$ són linealment independents i si $n=m$, són una base ortonormal.
    \item $f_A$ és una aplicació lineal ortogonal si i només si $A$ és una matriu ortogonal.
\end{enumerate}
\end{corollari}
\begin{proof}
Vegem primer (a). Si $f$ és ortogonal, del fet que $f(\vec e_i)\cdot f(\vec e_j)=\vec e_i\cdot \vec e_j$, es dedueix que $f(\vec e_i)$ són unitaris i ortogonals dos a dos, per tant, ortonormals.

Suposem ara que $f(\vec e_i)$ són unitaris i ortogonals dos a dos i unitaris i $\vec u \in \R^n$. Tenim que existeixen nombres reals $\lambda_i$ tals que $\vec u= \lambda_1\vec e_1+ \cdots + \lambda_n\vec e_n$. Llavors $||u||^2=\lambda_1^2+ \cdots + \lambda_n^2$, i $||f(\vec u)||^2=||\lambda_1 f(\vec e_1)+\cdots + \lambda_n f(\vec e_n)||^2=\lambda_1^2+ \cdots + \lambda_n^2$ (pel Teorema de Pitàgores, utilitzant que són ortogonals i unitaris). 

Vegem ara que els $f(\vec e_1)$ són linealment independents: més en general, tenim que si $\vec u_1, \dots, \vec u_k$ són vectors ortonormals, llavors són linealment independents: del teorema de Pitàgores es dedueix que:
\[
||\lambda_1\vec u_1 + \cdots + \lambda_k \vec u_k||^2=\lambda_1^2 + \cdots + \lambda_k^2
\]
I per tant si tenim una combinació $\vec 0 = \lambda_1\vec u_1 + \cdots + \lambda_k \vec u_k$, tots els $\lambda$ han de ser zero. Llavors tenim que, en el nostre cas, $f(\vec e_1), \dots, f(\vec e_n)$ són linealment independents.

Si $n=m$, $f(\vec e_1)$, \ldots, $f(\vec e_m)$ són $m$ vectors de $\R^m$ linealment independents, i per tant, base.

Per a veure (b), si $f_A$ és ortogonal i fem el producte $A^TA$, tenim que el coeficient $(i,j)$ és $f(\vec e_i)\cdot f(\vec e_j)$, per tant, com que són ortonormals, tenim $A^TA=\1_n$.

Les columnes d'$A$ són els vectors $f(\vec e_j)$, per tant, $A^TA=\1_n$ si i només si  $f(\vec e_1), \dots, f(\vec e_n)$ són ortogonals i unitaris (per tant $f$ és ortogonal).
\end{proof}
\begin{teorema}
Si $A,B \in M_n(\R)$ són matrius ortogonals, llavors:
\begin{enumerate}[\rm (a)]
    \item $AB$ també és ortogonal i
    \item $A^{-1}$ és ortogonal (i $A^{-1}=A^T)$.
\end{enumerate}
\end{teorema}
\begin{proof}
$AB$ serà ortogonal si $f_{AB}$ conserva la longitud dels vectors, però $f_{AB}=f_A \circ f_B$ i, per hipòtesi, tant $f_B$ com $f_A$ conserven la longitud, per tant, $f_{AB}$ també.

Per a veure (b): si $\vec v = f_{A^{-1}}(\vec w))$, llavors $\vec w=f_A(\vec v)$ i, per hipòtesi $||\vec w||=||f_A(\vec v)||=||\vec v||$, per tant $A^{-1}$ és ortogonal. 
\end{proof}

\subsection{Matriu d'una projecció ortogonal en una base ortonormal}
\begin{proposicio}
Si $V\subset \R^n$ un subespai vectorial, $[\vec v_1, \dots, \vec v_k]$ és una base ortonormal de $V$ i $Q$ és la matriu que té per columnes els vectors $\vec v_j$, llavors, la matriu de la projecció ortogonal a $V$ és:
\[
[\proj_V]=Q Q^T.
\]
\end{proposicio}
 \begin{proof}
 Com que $[\vec v_1, \dots, \vec v_k]$ és una base ortonormal de $V$, la projecció ortogonal a $V$ del vector $\vec u$ és:
 %\[
 %\vec u = \begin{pmatrix}
 %\lambda_1 \\ \vdots \\ \lambda_n
 %\end{pmatrix}
 %\]
 %és
 \[
 \proj_V(\vec u)=(\vec u\cdot \vec v_1)\vec v_1 + \cdots + (\vec u\cdot \vec v_k)\vec v_k = Q \begin{pmatrix}
 \vec u\cdot \vec v_1 \\ \vdots \\ \vec u\cdot \vec v_k \end{pmatrix}= Q \begin{pmatrix}
 \vec v_1^T \vec u \\ \vdots \\ \vec v_k^T \vec u
 \end{pmatrix} = Q
 \begin{pmatrix}
 \vec v_1^T \\
 \vdots \\
 \vec v_k^T
 \end{pmatrix} \vec u
 =(QQ^T)\vec u
 \]
 I per tant la matriu associada a la projecció és $QQ^T$.
 \end{proof}
\begin{exemple}
Considerem el vector $\smat{x\\y}\neq\smat{0\\0}$ de $\R^2$ i calculem la matriu de la projecció a $V=\langle \smat{x\\y}\rangle$. Una base ortonormal és $\vec v_1=\frac{1}{\sqrt{x^2+y^2}}\smat{x\\y}$ i per tan la matriu és (compareu-ho amb la Proposició \ref{prop:projR2}):
\[
[\proj_V]=\frac{1}{\sqrt{x^2+y^2}} \begin{pmatrix} x \\ y \end{pmatrix} \frac{1}{\sqrt{x^2+y^2}} \begin{pmatrix} x & y \end{pmatrix} = \frac{1}{x^2+y^2} \begin{pmatrix} x^2 & xy \\ xy & y^2 \end{pmatrix} .
\]
\end{exemple}
\begin{exemple}
Volem calcular la matriu de la projecció ortogonal de $\R^4$ a $V=\langle \smat{1\\-1\\-1\\1}, \smat{0\\1\\1\\0}\rangle \subset \R^4$. A l'Exemple \ref{exemple:GramSchmidt} ja hem calcular una base ortonormal, i era  $\calc=[\smat{1/2\\-1/2\\-1/2\\1/2},\smat{1/2\\1/2\\1/2\\1/2}]$, per tant la matriu corresponent a la projecció ortogonal és:
\[
\begin{pmatrix}
1/2 & 1/2 \\ -1/2 & 1/2 \\ -1/2 & 1/2 \\ 1/2 & 1/2
\end{pmatrix}
\begin{pmatrix}
1/2 & -1/2 & -1/2 & 1/2 \\ 1/2 & 1/2 & 1/2 & 1/2 
\end{pmatrix}
\begin{pmatrix}
1/2 & 0 & 0 & 1/2 \\
0 & 1/2 & 1/2 & 0 \\
0 & 1/2 & 1/2 & 0 \\
1/2 & 0 & 0 & 1/2
\end{pmatrix}.
\]
\end{exemple}

\subsection{Mínims quadrats}
Aquesta secció suposarem que tenim un conjunt $E$ i un subconjunt $V\subset E$ i que, a cada element d'$E$, volem assignar-li l'element de $V$ que millor l'aproxima. Per això, suposem que $E$ és un $\R$-espai vectorial i que $V$ és un subespai. Llavors:
\begin{proposicio}\label{prop:proj_com_aprox}
Si $E$ és un $\R$-espai vectorial i $V \subset E$ és un subespai, llavors, per a tot $\vec u \in E$ i $\vec v\in V$ es compleix:
\[
||\vec u - \proj_V(u)|| \leq ||\vec u - \vec v||.
\]
\end{proposicio}
\begin{proof}
Tenim que $\vec u - \vec v = (\vec u - \proj_V(u)) + (\proj_V(u) - \vec v)$ amb el primer vector de $\vec V^\perp$ i el segon de $V$. Aplicant el Teorema de Pitàgores tenim:
\[
||\vec u - \vec v||^2= ||\vec u - \proj_V(\vec u)||^2 + ||\proj_V(\vec u) - \vec v||^2 \geq ||\vec u - \proj_V(\vec u)||^2 ,
\]
i tenim la desigualtat que volem.
\end{proof}
Per tant, considerant la mètrica de la distància, la millor aproximació en aquest cas és la projecció ortogonal.

\subsubsection{Recta de regressió}\index{recta de regressió}
Suposem que tenim un núvol de punts  $(x_1,y_1), \dots , (x_k,y_k)$ de $\R^2$ que, en principi, no estan alineats. Suposem, a més, que hi ha una dependència del tipus $y=f(x)$ (per tant, al núvol de punts seria un problema que hi hagués dos punts amb la mateixa $x$ i diferents $y$). Suposem que els punts estan \emph{aproximadament} alineats i que el que volem trobar és la recta $y=a_0+a_1x$ que aproxima millor aquests punts. 

Una altra manera de pensar-ho és que volem resoldre el sistema d'equacions amb incògnites $a_0$ i $a_1$:
\begin{align*}
a_ 0 + x_1 a_1 & = y_1 \\ & \vdots  \\ a_0 + x_k a_k & = y_k 
\end{align*}
O, en forma matricial:
\[
\begin{pmatrix} 1 & x_1 \\ \vdots & \vdots \\ 1 & x_k \end{pmatrix}
\begin{pmatrix} a_0 \\ a_1 \end{pmatrix} =
\begin{pmatrix} y_1 \\ \vdots \\ y_k \end{pmatrix}
\]
Com que molt probablement els punts no estan alineats, com a sistema d'equacions, és un sistema incompatible. Dit d'una altra manera, el vector $\vec y=\smat{y_1 \\ \vdots \\ y_k}$ no pertany al subespai $V=\langle \smat{1 \\ \vdots \\ 1} , \smat{x_1 \\ \vdots \\ x_k} \rangle$. Si apliquem la Proposició \ref{prop:proj_com_aprox}, el vector de $V$ que serà més proper a $\vec y$ serà $\proj_V(\vec y)$:
\[
\proj_V(\begin{pmatrix} y_1 \\ \vdots \\ y_k \end{pmatrix})= a_0 \begin{pmatrix} 1 \\ \vdots \\ 1 \end{pmatrix} + a_1 \begin{pmatrix} x_1 \\ \vdots \\ x_k \end{pmatrix}= \begin{pmatrix} 1 & x_1 \\ \vdots & \vdots \\ 1 & x_k \end{pmatrix}
\begin{pmatrix} a_0 \\ a_1 \end{pmatrix} .
\]
Podríem calcular $a_0$ i $a_1$ fent primer Gram-Schmidt per a trobar una base ortonormal de $V$ i fent els productes escalars corresponents, o bé fent les consideracions següents: el vector
\[
\begin{pmatrix} y_1 \\ \vdots \\ y_k \end{pmatrix} -
\begin{pmatrix} 1 & x_1 \\ \vdots & \vdots \\ 1 & x_k \end{pmatrix}
\begin{pmatrix} a_0 \\ a_1 \end{pmatrix}
\]
és perpendicular a $V$, per tant ha de complir que:
\[
\begin{pmatrix}
1 & \dots & 1 & \\ x_1 & \dots & x_n
\end{pmatrix}
\left(
\begin{pmatrix} y_1 \\ \vdots \\ y_k \end{pmatrix} -
\begin{pmatrix} 1 & x_1 \\ \vdots & \vdots \\ 1 & x_k \end{pmatrix}
\begin{pmatrix} a_0 \\ a_1 \end{pmatrix}\right) = \begin{pmatrix} 0 \\ 0 \end{pmatrix} 
\]
I per tant:
\[
\begin{pmatrix}
1 & \dots & 1 & \\ x_1 & \dots & x_n
\end{pmatrix}
\begin{pmatrix} 1 & x_1 \\ \vdots & \vdots \\ 1 & x_k \end{pmatrix}
\begin{pmatrix} a_0 \\ a_1 \end{pmatrix} = 
\begin{pmatrix}
1 & \dots & 1 & \\ x_1 & \dots & x_k
\end{pmatrix}
\begin{pmatrix} y_1 \\ \vdots \\ y_k \end{pmatrix}
\]
I queda el sistema compatible determinat:
\[
\begin{pmatrix}
k & \sum x_i \\ \sum x_i & \sum x_i^2 
\end{pmatrix}
\begin{pmatrix} a_0 \\ a_1 \end{pmatrix} = 
\begin{pmatrix} \sum y_i \\ \sum x_iy_i \end{pmatrix}
\]
Que es pot escriure: si $\overline x = \frac{1}{k} \sum x_i$, $\overline y=\frac{1}{k} \sum y_i$, $\overline {x^2}= \frac{1}{k} \sum x_i^2$ i $\overline {xy}=  \frac{1}{k} \sum x_iy_i$:
\[
\begin{pmatrix}
1 & \overline x \\ \overline x & \overline{x^2} 
\end{pmatrix}
\begin{pmatrix} a_0 \\ a_1 \end{pmatrix} = 
\begin{pmatrix}  \overline y \\ \overline{xy} \end{pmatrix}
\]
I la solució es pot escriure com:
\begin{equation}\label{eq:recta_reg}
a_1=\frac{\overline{xy} - \overline x \, \overline y}{\overline{x^2} - (\overline x)^2}
\text{ i } 
a_0= \overline y - a_1 \overline x .
\end{equation}
\begin{exemple}
Considerem les notes següents d'un grup de $21$ alumnes corresponent a una avaluació parcial i la nota final que van treure:
\[
\begin{array}{|c|c||c|c||c|c|}
\hline\text{Parcial} & \text{Final} & \text{Parcial} & \text{Final} & \text{Parcial} & \text{Final}  \\ \hline
8.15 & 8.995 & 6.25 & 7.845 & 2.2 & 4.58 \\
0.3 & 1.98 & 5.85 & 7.075 & 4.6 &4.95 \\
4.3&5.01 & 3.55 & 5.715 & 4.7 & 7.08\\
1.8&2.88 & 1.7 & 2.82 & 2.3 & 3.86\\
5.4&5.08 & 4.6 & 7.13 & 2.7 &5.33 \\
0.2 &2.78 & 7.7 & 8.8 & 7.95 &8.595\\
5.4 & 7.77 & 3.15 &1.245 & 3 &4.95\\ \hline
\end{array}
\]
I volem aproximar la nota final a partir de la del parcial amb una recta:
\[
\text{final}= a_0 + a_1 \text{parcial}
\]
Per tant, podem aplicar la Fórmula \eqref{eq:recta_reg} als valors (aproximem a 3 decimals):
\[
\overline{x}=\overline{\text{parcial}}= 4.085, \quad
\overline{y}=\overline{\text{final}}= 5.451, \quad
\overline{x^2}=21,839 \text{ i }
\overline{xy}= 26,816.
\]
I queda (amb 3 decimals):
\[
\text{final}= 1.843 + 0.883 \text{parcial}.
\]
La Figura \ref{fig:recta_reg} té una representació gràfica d'aquestes dades i de la recta de regressió. A més, l'equació de la recta es pot utilitzar per aproximar una nota final a partir d'una parcial: per exemple, un alumne que hagi tret un $4$ al parcial, la predicció diria que trauria un $5,376$ a la nota final.
\begin{figure}[ht]
\center{\includegraphics[width=10cm]{regre.png}}
\caption{Recta de regressió\label{fig:recta_reg}}
\end{figure}
\end{exemple}
\subsubsection{Cas general}
En el cas general, suposem que tenim un sistema d'equacions lineals:
\[
A x = b
\]
amb $A\in M_{m\times n}(\R)$ amb $m\geq n$ (més equacions que incògnites) i tal que $\Rang(A)=n$. És molt probable que aquest sistema no tingui solució, i per aquests casos ens interessa tenir la millor aproximació:
\begin{lema}
Amb les hipòtesis anteriors, el sistema
\[
A^T A x = A^T b
\]
té solució única. Aquesta solució s'anomena \emph{solució per mínim quadrats del sistema $Ax=b$}\index{mínims quadrats}.
\end{lema}
\begin{proof}
Observem que $A^TA \in M_{n\times n}(\R)$, pel que tant sols cal veure que $\Rang(A^T A)=n$ (i llavors serà un sistema compatible determinat), o el que és el mateix, que l'aplicació lineal $f_{A^TA}$ és injectiva (llavors el sistema homogeni que té per matriu associada $A^TA$ serà compatible determinat i per tant $\Rang(A^T A)=n$). Però si $\vec u \in \R^n$ compleix $A^TA\vec u=\vec 0$, tenim $A^T(A\vec u)=\vec 0$, per tant $A\vec u$ és solució del sistema homogeni que té per matriu associada $A^T$, que és de rang $n$, per tant $A\vec u=\vec 0$. Aplicant ara que $\Rang(A)=n$, tenim que $\vec u=\vec 0$ i per tant $\Rang(A^TA)=n$.
\end{proof}
\begin{lema}
Continuem amb les hipòtesis anteriors.
De tots els vectors $\vec v \in \R^n$, la solució per mínims quadrats del sistema $Ax=b$ és la que és més propera a $v$ en el sentit següent: si $x$ és la solució del sistema $A^TAx=A^Tb$ i $v\in \R^n$, llavors 
\[
||Ax - b||^2 \leq ||Av- b||^2.
\]
\end{lema}
\begin{proof}
Es dedueix del fet que la solució per mínims quadrats sigui la projecció ortogonal de $b$ a l'espai generat per les columnes d'$A$, que, com que el rang d'$A$ és $n$, queda caracteritzada per complir:
\[
A^T(Ax-b)=\vec 0 .
\]
\end{proof}
D'aquí també es dedueix que la projecció del vector $b$ a $V$, el subespai generat per les columnes d'$A$ és, en la base de $V$ formada per les columnes d'$A$ és:
\[
x=(A^TA)^{-1}A^Tb
\]
Per tant, el vector d'$\R^n$ és:
\[
Ax=A(A^TA)^{-1}A^Tb
\]
I per tant:
\begin{proposicio}
Si $V\subset \R^n$ és un subespai vectorial i $[\vec v_1, \dots , \vec v_k]$ és una base de $V$, la projecció de $\R^n$ a $V$ ve donada per la matriu:
\[
[\proj_V]=A(A^TA)^{-1}A^T
\]
on $A$ és la matriu que té per columnes els vectors $\vec v_j$.
\end{proposicio}
\begin{exemple}
Suposem que tenim un grup de 24 estudiants als que s'ha fet una avaluació parcial i un lliurament a mig semestre amb les notes següents. La tercera columna diu quina ha sigut la nota final de l'assignatura (amb més avaluacions entremig):
\[
\begin{array}{|c|c|c||c|c|c|}
\hline \text{Lliurament} & \text{Parcial} & \text{Final} & 
\text{Lliurament} & \text{Parcial} & \text{Final}\\ \hline
9.25 & 9.75 & 8.25 & 
9.5 & 5.2 & 7.27 \\
9.5 & 2.4 & 3.66 &
9 & 7.1 & 8.32 \\
8 & 1.6 & 5.65 &
8 & 4 & 7.61 \\
8 & 7.1 & 6.71 &
8.5 & 3.4 & 6.77 \\
8.5 & 5.7 & 7.77 &
8 & 6 & 7.74 \\
6 & 2.9 & 6.5 &
10 & 5.9 & 8.37 \\
9 & 9 & 8.13 &
10 & 2.6 & 8.13 \\
8 & 7.6 & 7.15 &
7.5 & 1.6 & 5.21 \\
9 & 6.2 & 6.8 &
10 & 9 & 8.52 \\
10 & 10 & 9.07 &
9 & 8 & 7.76 \\
9 & 6.5 & 7.43 &
9.75 & 10 & 8.71 \\
8.5 & 4 & 6.39 &
9.5 & 7.4 & 8.57 \\ \hline
\end{array}
\]
Volem aproximar la nota final a partir de les dues que se saben a mig semestre mitjançant una formula:
\[
\text{final} = a * \text{Lliurament} + b * \text{Parcial} + c
\]
I per tant, fem mínim quadrats per trobar $a,b$ i $c$.
En aquest cas, el sistema que té moltes més equacions que incògnites seria:
\[
\begin{pmatrix}
9.25 & 9.75 & 1 \\
9.5 & 2.4 & 1 \\
8 & 1.6 & 1 \\
\vdots & \vdots & \vdots \\
9.5 & 7.4 & 1
\end{pmatrix}
\begin{pmatrix} a \\ b \\ c \end{pmatrix}=
\begin{pmatrix} 8.25 \\ 3.66 \\ 5.65 \\ \vdots \\ 8.57 \end{pmatrix}
\]
I per tant hem de resoldre el sistema compatible determinat:
\[
\begin{pmatrix}
9.25 & 9.5 & 8 & \cdots & 9.5 \\
9.75 & 2.4 & 1.6 & \cdots & 7.4 \\
1 & 1 & 1 & \cdots & 1
\end{pmatrix}
\begin{pmatrix}
9.25 & 9.75 & 1 \\
9.5 & 2.4 & 1 \\
8 & 1.6 & 1 \\
\vdots & \vdots & \vdots \\
9.5 & 7.4 & 1
\end{pmatrix}
\begin{pmatrix} a \\ b \\ c \end{pmatrix}=
\begin{pmatrix}
9.25 & 9.5 & 8 & \cdots & 9.5 \\
9.75 & 2.4 & 1.6 & \cdots & 7.4 \\
1 & 1 & 1 & \cdots & 1
\end{pmatrix}
\begin{pmatrix} 8.25 \\ 3.66 \\ 5.65 \\ \vdots \\ 8.57 \end{pmatrix}
\]
Quedant:
\[
\left(\begin{array}{ccc}
1885.375 & 1287.5375 & 211.5 \\
1287.5375 & 1015.0425 & 142.95 \\
211.5 & 142.95 & 24
\end{array}\right)
\begin{pmatrix} a \\ b \\ c \end{pmatrix}=
\begin{pmatrix}
1568.205 \\ 1108.1755 \\ 176.49
\end{pmatrix}
\]
Que té per solució (amb 3 decimals):
\[
\begin{pmatrix} a \\ b \\ c \end{pmatrix}=
\begin{pmatrix} 0.191 \\
0.316 \\
3.790 \end{pmatrix}
\]
Per ant l'aproximació és:
\[
\text{final} = 0.191 * \text{Lliurament} + 0.316 * \text{Parcial} + 3.79\, .
\]
Per exemple, d'un alumne que tregui un $4$ al lliurament i un altre $4$ al parcial, aquest model diu que de l'assignatura acabés amb un $5.817$
\end{exemple}


\subsection{Formes bilineals i productes escalars}
% Més abstracte
\begin{definicio}
Donat un espai vectorial $E$ sobre un cos $\K$, una \emph{forma bilineal}\index{forma bilineal} és una aplicació
\[
\phi \colon E \times E \to \K
\]
tal que:
\begin{itemize}
\item $\phi(\vec u_1+\vec u_2,v)=\phi(\vec u_1,\vec v)+\phi(\vec u_2,\vec v)$ per a tots $\vec u_1$, $\vec u_2$ i $\vec v$ d'$E$,
\item $\phi(\lambda \vec u,\vec v)=\lambda \phi(\vec u,\vec v)$ per a tots $\vec u$ i $\vec v$ d'$E$ i $\lambda \in \K$,
\item $\phi(\vec u,\vec v_1,\vec v_2)=\phi(\vec u,\vec v_1)+\phi(\vec u,\vec v_2)$ per a tots $\vec u$, $\vec v_1$ i $\vec v_2$ d'$E$ i
\item $\phi(\vec u,\lambda \vec v)=\lambda \phi(\vec u,\vec v)$ per a tots $\vec u$ i $\vec v$ d'$E$ i $\lambda \in \K$.
\end{itemize}
A més diem que una aplicació bilineal $\phi$ és:
\begin{itemize}
    \item \emph{simètrica}\index{forma bilineal!simètrica} si $\phi(\vec u,\vec v)=\phi(\vec v,\vec u)$ per a tots $\vec u$ i $\vec v$ d'$E$,
    \item \emph{degenerada}\index{forma bilineal!degenerada} si existeix $\vec u\neq \vec 0$ tal que $\phi(\vec u,\vec v)=0$ per a tot $\vec v$ d'$E$, i
    \item \emph{definida positiva}\index{forma bilineal!definida positiva} si $\phi(\vec u,\vec u)>0$ per a tot $\vec u\neq \vec 0$.
\end{itemize}
\end{definicio}
\begin{definicio}\label{def:mat_apl_bil}
Si $E$ és un $\K$-espai vectorial de dimensió finita i $\calb=[\vec v_1, \dots, \vec v_n]$ és una base d'$E$, i $\phi$ és una aplicació bilineal, \emph{la matriu de l'aplicació bilineal en la base $\calb$}\index{aplicació bilineal!matriu}\index{matriu!aplicació bilineal} és:
\[
[\phi]_\calb= \begin{pmatrix}
\phi(\vec v_1,\vec v_1) & \phi(\vec v_1,\vec v_2) & \cdots & \phi(\vec v_1,\vec v_n)\\
\phi(\vec v_2,\vec v_1) & \phi(\vec v_2,\vec v_2) & \cdots & \phi(\vec v_2,\vec v_n)\\
\vdots & \vdots & \ddots & \vdots \\
\phi(\vec v_n,\vec v_1) & \phi(\vec v_n,\vec v_2) & \cdots & \phi(\vec v_n,\vec v_n)
\end{pmatrix}.
\]
\end{definicio}
\begin{lema}
Si $\phi$ és una aplicació bilineal definit sobre un $\K$-espai vectorial $E$ amb base  $\calb=[\vec v_1, \dots, \vec v_n]$ i $[\phi]_\calb$ és la matriu de $\phi$ en la base $\calb$, llavors: si $\vec u$ i $\vec w$ són vectors d'$E$ amb coordenades en la base $\calb$:
\[
[\vec u]_\calb = \begin{pmatrix}
\lambda_1 \\ \vdots \\ \lambda_n
\end{pmatrix}
\text{ i }
[\vec w]_\calb = \begin{pmatrix}
\mu_1 \\ \vdots \\ \mu_n
\end{pmatrix}.
\]
llavors:
\[
\phi(\vec u,\vec w)=\sum_{i=1}^n \sum_{j=1}^n \lambda_i\mu_j\phi(\vec v_i,\vec v_j) =[\vec u]_\calb^T \, [\phi]_\calb \, [\vec w]_\calb .
\]
\end{lema}
\begin{proof}
La primera igualtat és per bilinealitat, i és igual a l'última expressió, que està escrita en forma matricial.
\end{proof}
\begin{lema}\label{lema:canvi_base_forma_bil}
Si tenim $\calb=[\vec v_1, \dots, \vec v_n]$ i $\calc=[\vec u_1,\dots, \vec u_n]$ bases de $\K^n$, $[\Id_n]_{\calc,\calb}$ la matriu del canvi de base (la que té per columnes les coordenades dels vectors $\vec u_j$  en la base $\calb$), i $\phi$ una aplicació bilineal, hi ha la relació següent entre les matrius de l'aplicació bilineal en cada base:
\[
[\phi]_\calc=[\Id_n]_{\calc,\calb}^T \, [\phi]_\calb \, [\Id_n]_{\calc,\calb} 
\]
\end{lema}
\begin{proof}
Les matrius de $[\phi]_\calb$ i $[\phi]_\calc$ han de complir que, per a tot $\vec u$ i $\vec v$ de $\K^n$ es compleixi:
\begin{equation}\label{eq:canvi_base_apl_bil}
[\vec u]_\calc^T \, [\phi]_\calc \, [\vec v]_\calc = \phi(\vec u, \vec v)=  [\vec u]_\calb^T\, [\phi]_\calb\, [\vec v]_\calb .
\end{equation}
Però, per les propietats de la matriu $[\Id]_{\calc,\calb}$ tenim:
\[
[\vec u]_\calb=[\Id]_{\calc,\calb} [\vec u]_\calc ,
\]
i si ho substituïm a l'Equació \eqref{eq:canvi_base_apl_bil}, tenim, que per a tot $\vec u$ i $\vec v$ de $\K^n$:
\[
[\vec u]_\calc^T \, [\phi]_\calc \, [\vec v]_\calc = ([\Id]_{\calc,\calb} [\vec u]_\calc)^T \, [\phi]_\calb \, ([\Id]_{\calc,\calb} [\vec v]_\calc) = [\vec u]_\calc^T \, ([\Id]_{\calc,\calb}^T [\phi]_\calb [\Id]_{\calc,\calb}) \, [\vec v]_\calc ,
\]
pel que tenim la igualtat de matrius que volem.
\end{proof}
\begin{exemple}
Considerem la forma bilineal simètrica $\phi\colon \R^2 \times \R^2 \to \R$ donada per la matriu (en la base estàndard $\calb=[\vec e_1,\vec e_2]$)
\[
[\phi]=\begin{pmatrix} 0 & 1/2 \\ 1/2 & 0 \end{pmatrix}
\]
I volem calcular la matriu de $\phi$ en la base $\calc=[\smat{1\\1},\smat{1\\-1}]$. Llavors hem de calcular:
\[
[\Id]_{\calc,\calb}=\begin{pmatrix} 1 & 1 \\ 1 & -1 \end{pmatrix}
\]
I per tant la matriu
\[
[\phi]_\calc = \begin{pmatrix} 1 & 1 \\ 1 & -1 \end{pmatrix}^T
\begin{pmatrix} 0 & 1/2 \\ 1/2 & 0 \end{pmatrix}
\begin{pmatrix} 1 & 1 \\ 1 & -1 \end{pmatrix}=
\begin{pmatrix} 1 & 0 \\ 0 & -1 \end{pmatrix}.
\]
Per tant, les matrius $\smat{0 & 1/2 \\ 1/2 & 0}$ i $\smat{1 & 0 \\ 0 & -1}$ representen la mateixa forma bilineal en bases diferents.
\end{exemple}

Mirem ara un cas particular de forma bilineal:

\begin{definicio}
Un \emph{producte escalar}\index{producte escalar} sobre un $\R$-espai vectorial $E$ és una forma bilineal simètrica definida positiva. En general, escriurem el resultat amb un punt: $\vec u \cdot \vec v$.
\end{definicio}
Si $\calb$ és una base finita d'un espai vectorial $E$ i $\cdot$ és un producte escalar, la matriu $[\cdot]_\calb$ serà simètrica.
\begin{exemple}
Si considerem $\R^n$ i $\calb=[\vec e_1, \dots, \vec e_n]$ la base estàndard, el producte escalar definit com $\vec u \cdot \vec v=\vec u^T \vec v$ és un producte escalar amb matriu la matriu identitat $\1_n$.
\end{exemple}

\begin{exemple}
Si $E=C^\infty(\R)$ definim el producte escalar:
\[
f\cdot g = \int_{-1}^1 f(x)g(x)\mathrm{d}x .
\]
Com que a $C^\infty(\R)$ no hi ha una base finita, no té sentit parlar de la matriu del producte escalar.
\end{exemple}

\begin{exemple}\label{exemple:complexos}
Considerem els nombres complexos $\C$\index{nombres complexos}. Definits com $\R$ espai vectorial, tenen dimensió $2$ i una base és $[1,i]$, i d'aquí deduïm l'estructura de la suma. A més, però, es poden multiplicar,  i la multiplicació ve caracteritzada pel fet $i^2=-1$ i que sigui commutativa i distributiva amb la suma: o sigui,
\[
\C = \{ a + bi \mid a \in\R, b \in\R \}
\]
amb les operacions:
\[
(a+bi)+(c+di)=(a+c)+(b+d)i \text{ i } (a+bi)(c+di)=(ac-bd)+(ad+bc) i .
\]
Veiem que $\R\subset \C$, considerant el números de la forma $a+0i$.

Si $z=a+bi$, amb $a$ i $b$ reals, és un nombre complex, definim el \emph{conjugat de $z$}\index{conjugat} i escrivim $\overline z$ com:
\[
\overline z = a-bi .
\]
Veiem que:
\begin{itemize}
    \item $z \overline z=a^2+b^2\in \R$, 
    \item $z=\overline{z} \iff z\in\R$,
    \item $\overline{z_1 + z_2}=\overline{z_1}+\overline{z_2}$ i
    \item $\overline{z_1 z_2}=\overline{z_1}\, \overline{z_2}$.
\end{itemize}

Considerem ara $\C^n$ com a $\C$-espai vectorial de dimensió $n$ (suma coordenada a coordenada i multiplicació per un escalar a totes les coordenades). Definim la forma bilineal següent:
\[
\text{si }
\vec v=\begin{pmatrix}
v_1 \\ \vdots \\ v_n
\end{pmatrix}
\text{ i }
\vec w=\begin{pmatrix}
w_1 \\ \vdots \\ w_n
\end{pmatrix}
\text{ llavors }
\vec v\cdot \vec w= \sum_{i=1}^n v_i\overline{w}_i=\vec v^T \overline{\vec w} \text{ on }
\overline{\vec w}=\begin{pmatrix}
\overline w_1 \\ \vdots \\ \overline w_n
\end{pmatrix}.
\]
Que compleix, a més de les propietats de ser forma bilineal:
\begin{itemize}
    \item $\vec v\cdot \vec w$=$\overline{\vec w\cdot \vec v}$ per a tot $\vec v,\vec w\in \C^n$ (diem que és hermítica\index{forma!hermítica}) i
    \item $\vec v\cdot \vec v=||\vec v||^2\in \R$ per a tot $\vec v\in\C^n$.
\end{itemize}
\end{exemple}

\subsection{Tota matriu simètrica sobre \texorpdfstring{$\mathbb{R}$}{R} diagonalitza}
El resultat que volem demostrar a aquest apartat és:
\begin{teorema}[Teorema espectral]
Una matriu $A\in M_n(\R)$ diagonalitza en una base ortonormal si, i només si, $A$ és simètrica.
\end{teorema}
Per demostrar el teorema espectral necessitem utilitzar el \emph{Teorema fonamental de l'àlgebra}, que no demostrem aquí perquè la seva demostració surt dels objectius del curs:
\begin{teorema}[Teorema fonamental de l'àlgebra]
Si $p(x)\in\C[x]$ ($p(x)$ és un polinomi amb coeficients a $\C$) de grau $\geq 1$, existeix $z\in C$ tal que $p(z)=0$.
\end{teorema}
\begin{proof}[Demostració del teorema espectral]
Una implicació és directa: si $A$ diagonalitza en una base ortonormal vol dir que podem escriure:
\[
A=Q D Q^T
\]
amb $D$ una matriu diagonal i $Q$ una matriu ortogonal ($Q^{-1}=Q^T$). Per tant:
\[
A^T=(QDQ^T)^T=(Q^T)^T D^T Q^T=Q D^T Q^T=A
\]
i $A$ és simètrica.

Per demostrar l'altra implicació, ho farem per inducció sobre $n$:
\begin{itemize}
    \item Per a $n=1$, tota matriu és simètrica i diagonal.
    \item Suposem cert per a $n-1$, i volem demostrar-ho per $n$: considerem $A\in M_n(\R)$ i $p_A(x)$ el polinomi característic. Pensem $p_A(x)\in\C[x]$ i, pel teorema fonamental de l'àlgebra, existeix $\lambda \in \C$ tal que $p_A(\lambda)=0$. També tindrem un vector propi $\vec v\in \C^n$ tal que $A \vec v=\lambda \vec v$.
    
    Considerem ara la forma bilineal hermítica definida a l'Exemple \ref{exemple:complexos}:
    \[
    A\vec v \cdot \vec v = \lambda \vec v \cdot \vec v= \lambda ||\vec v||^2 \text{ amb $0\neq ||v||^2\in R$}.
    \]
    Però també:
    \[
    A\vec v \cdot\vec  v = (A \vec v)^T \overline{\vec v} = \vec v^T A^T \overline{\vec v} = \vec v^T \overline{(\overline{A^T} \vec v)} = \vec v^T A \overline{\vec v} = \vec v^T (\overline{\lambda \vec v})=\overline{\lambda} ||\vec v||^2 , 
    \]
    on hem utilitzat que $A=A^T$ ($A$ és simètrica) i $\overline A=A$ ($A$ té coeficients reals).
    
    Per tant, tenim que $\lambda=\overline{\lambda}$ i obtenim $\lambda \in \R$.
    
    Com que $p_A(\lambda)=0$, tenim que existeix un vector propi real $\vec v\neq \vec 0$ tal que $A\vec v=\lambda \vec v$. Podem considerar que $||\vec v||=1$ (si cal, substituïm $\vec v$ per $\frac{1}{||\vec v||}\vec v$).
    
    Ampliem $\vec v$ amb $\vec v_2$, \dots, $\vec v_n$ de tal manera que $[\vec v,\vec v_2, \dots ,\vec v_n]$ sigui una base ortonormal (primer ampliem fins a base, i després apliquem Gram-Schmidt). Si considerem $Q$ la matriu que té per columnes els vectors $\vec v$, $\vec v_2$, \ldots , $\vec v_n$, tenim:
    \[
    QAQ^T=\begin{pmatrix} \lambda & b_{12} & \cdots & b_{1n} \\
    0 & b_{22} & \cdots & b_{2n} \\
    \vdots & \vdots & \ddots & \vdots \\
    0 & b_{n2} & \cdots  & b_{nn}
    \end{pmatrix}
    \]
    on hem utilitzat que $Q^{-1}=Q^T$. Però com que $A=A^T$, tenim $(QAQ^T)^T=QAQ^T$, i per tant $b_{12}=b_{13}=\cdots=b_{1n}=0$.
    
    Ara restringim l'aplicació $f_A$ a l'espai $\langle v_2, \dots , v_n\rangle$, que en la base $[v_2, \dots , v_n]$ té per matriu
    \[
    B=\begin{pmatrix} b_{22} & \cdots & b_{2n} \\
    \vdots & \ddots & \vdots \\
    b_{n2} & \cdots  & b_{nn}
    \end{pmatrix}
    \]
    que és simètrica i $B\in M_{n-1}(\R)$. Per hipòtesis d'inducció, $B$ diagonaliza en una base ortonormal, pel que existeix $P\in M_{n-1}(\R)$ matriu ortogonal i $D$ matriu diagonal tals que $B=P^TDP$. Amb això tenim:
    \[
    QAQ^T=\left(\begin{array}{c|c}
        \lambda & 0 \\ \hline
        0 & P^T D P
    \end{array}\right)=
    \left(\begin{array}{c|c}
        1 & 0 \\ \hline
        0 & P^T
    \end{array}\right)
    \left(\begin{array}{c|c}
        \lambda & 0 \\ \hline
        0 & D 
    \end{array}\right)
    \left(\begin{array}{c|c}
        1 & 0 \\ \hline
        0 &  P
    \end{array}\right)
    \]
    Per tant, considerem
    \[
    Q'=\left(\begin{array}{c|c}
        1 & 0 \\ \hline
        0 &  P
    \end{array}\right) Q \text{ i }
    D'=\left(\begin{array}{c|c}
        \lambda & 0 \\ \hline
        0 & D 
    \end{array}\right)
    \]
    i tenim que $Q'$ és una matriu ortogonal i $A=Q'^{-1} D' Q'$, amb $D'$ una matriu diagonal, pel que $A$ diagonalitza a una base ortonormal.
    \end{itemize}
\end{proof}

\subsection{Descomposició en valors singulars}
A aquest apartat resoldrem la pregunta següent amb les eines que hem vist: si $f=f_A\colon \R^n \to \R^m$ és una aplicació lineal, existeix una base ortonormal $\calb=[\vec v_1, \dots, \vec v_n]$ d'$\R^n$ tal que $f(\vec v_1), \dots, f(\vec v_n)$ siguin ortogonals?

Veurem que sí:
\begin{teorema}\label{teo:val_sing}
$f=f_A\colon \R^n \to \R^m$ és una aplicació lineal, llavors existeix una base ortonormal $\calb=[\vec v_1, \dots, \vec v_n]$ d'$\R^n$ tal que $f(\vec v_1), \dots, f(\vec v_n)$ tal que són ortogonals.
\end{teorema}
\begin{proof}
Considerem $f=f_A$, i per tant $A\in M_{m\times n}(\R)$ la matriu tal que $f(\vec v)=A\vec v$. Llavors, la matriu $A^TA\in M_n(\R)$ és simètrica, i per tant diagonalitza en una base ortonormal, per tant, existeix una base ortonormal $\calb[\vec v_1, \dots, \vec v_n]$ de $\R^n$ i escalars $\lambda_1, \dots, \lambda_n\in \R$ tals que
\[
A^T A \vec v_i=\lambda_i \vec v_i \text{ per a tot $i=1, \dots, n$.}
\]
Volem veure que $f(\vec v_1), \dots, f(\vec v_n)$ són ortogonals, o sigui, $f(\vec v_i)\cdot f(\vec v_j)=0$ si $i\neq j$:
\begin{align*}
f(\vec v_i)\cdot f(\vec v_j) & =(A\vec v_i)\cdot(A\vec v_j)=(A\vec v_i)^T(A\vec v_j)=(\vec v_i^T A^T)(A \vec v_j)= \\
 &  = \vec v_i^T (A^T A \vec v_j)=\vec v_i^T (\lambda_j \vec v_j)=\lambda_j (\vec v_i \cdot \vec v_j)=0 .
\end{align*}
\end{proof}
\begin{observacio}
El Teorema~\ref{teo:val_sing} no diu que $\calb$ sigui única, i de fet, en general, no ho és: considereu l'aplicació identitat de $\R^n \to \R^n$, que envia qualsevol base ortonormal a vectors ortogonals.
\end{observacio}

 La interpretació geomètrica a $\R^2$ és la següent: la imatge d'una circumferència de radi $1$ centrada a l'origen és una el·lipsi, els vectors propis d'$A^TA$ corresponen a les direccions principals de l'el·lipsi, i les longituds dels semieixos principals són les arrels quadrades dels valors propis d'$A^TA$. Posem nom a aquests valors:
 \begin{definicio}
 Donada una matriu $A\in M_{m\times n}(\R)$, definim els \emph{valors singulars d'$A$}\index{valors singulars} com les arrels quadrades dels valors propis de la matriu $A^TA$: $\sigma_1, \dots, \sigma_m$ positius tals que $\sigma_i^2$ és valor propi d'$A^TA$.
 \end{definicio}
 \begin{observacio}
 $\sigma_i$ és el mòdul $||f(\vec v_i)||$ a l'enunciat del Teorema \ref{teo:val_sing}.
 \end{observacio}
 \begin{proposicio}
 Si $f=f_A\colon \R^n \to \R^m$ és una aplicació lineal i $\Rang(A)=r$, llavors hi ha $r$ valors singulars diferent de zero i els $n-r$ restants valen zero.
 \end{proposicio}
\begin{proof}
Tenim que en la base $\calb$ del Teorema \ref{teo:val_sing}, $f_A$ té per matriu $[f_A]\calb$, una matriu amb $\sigma_i$ a la diagonal i zeros a la resta. Llavors, com que el rang és la dimensió de la imatge, el rang de $f_A$  és igual al de $[f_A]_\calb$, que és el nombre de $\sigma_i$ no nuls, i la resta han de ser zero.
\end{proof}

Podem aprofitar aquests raonaments per demostrar:
\begin{teorema}\label{teo:SVD}
Si $A\in M_{m\times n}(\R)$, llavors existeixen $P \in M_{m}(\R)$ ortogonal, $D\in M_{m\times n}(\R)$ amb els únics elements no nuls a la diagonal i positius, i $Q\in M_n(\R)$ ortogonal tals que:
\[
A=PDQ.
\]
A més, podem ordenar les entrades diagonals de $D$ de més gran a més petita. Anomenem a aquesta descomposició com una \emph{descomposició en valors singulars d'$A$}\index{descomposició en valors singulars}.
\end{teorema}
\begin{proof}
Considerem l'aplicació lineal $f_A$ i veiem que necessitem bases ortonormals $\calb$ d'$\R^n$ i $\calc$ d'$\R^m$ tals que $D=[f_A]_{\calb,\calc}$ sigui diagonal.

El Teorema \ref{teo:val_sing} ens dóna una base $\calb$ d'$\R^n$ que és ortonormal. Ordenem aquesta base de tal manera que $||f(\vec v_i)||\geq ||f(\vec v_{i+1})||$. Sigui $Q^T=Q^{-1}$ la matriu que té per columnes aquesta base.

Els vectors $f(\vec v_1), \dots, f(\vec v_n)$ són ortogonals. Suposem que els $k$ primers són no nuls, llavors, considerant:
$\vec w_i=\frac{1}{||f(\vec v_i)||}$ per a $1\leq i \leq k$, tindrem vectors ortonormals d'$\R^m$, per tant linealment independents, i que podem ampliar amb $\vec w_{k+1}, \dots, \vec w_m$ fins a tenir una base ortonormal $\calc$ d'$\R^m$. Sigui $P$ la matriu que té per columnes aquests vectors.

Com que $f(\vec v_i)=||f(\vec v_i)|| \vec w_i$ si $i\leq k$ i $f(\vec v_i)=\vec 0$ si $i>k$, tenim que $[f]_{\calb,\calc}$ és diagonal amb els valors singulars d'$A$ a la diagonal ordenats de més gran a més petit.
\end{proof}

\begin{exemple}
Considerem l'aplicació lineal $f_A\colon \R^2 \to \R^2$ amb:
\[
A=\begin{pmatrix}
3 & -9/5 \\ -1 & 13/5
\end{pmatrix}
\]
i volem fer-ne una descomposició en valors singulars.

Calculem primer $A^TA$:
\[
A^T A = \begin{pmatrix}
10 & -8 \\ -8 & 10
\end{pmatrix}
\]
I la diagonalitzem. El polinomi característic és:
\[
p_{A^TA}(x)=x^2 - 20 x + 36 = (x-18)(x-2)
\]
Per tant, els valors propis d'$A^TA$ són $\lambda_1=18$ i $\lambda_2=2$ i els corresponents vectors propis:
\[
\Ker(A^TA-18\1_2)=\langle \begin{pmatrix} 1 \\ -1 \end{pmatrix} \rangle \text{ i }
\Ker(A^TA-2\1_2)=\langle \begin{pmatrix} 1 \\ 1 \end{pmatrix} \rangle
\]
I com que els hem de considerar unitaris, obtenim:
\[
Q^T=\begin{pmatrix}
1/\sqrt{2} & 1/\sqrt{2} \\ -1/\sqrt{2} & 1/\sqrt{2}
\end{pmatrix}
\text{ i per tant }
Q=\begin{pmatrix}
1/\sqrt{2} & -1/\sqrt{2} \\ 1/\sqrt{2} & 1/\sqrt{2}
\end{pmatrix}.
\]

La matriu diagonal són els valors singulars, que són les arrels quadrades dels valors propis d'$A^TA$, per tant són $\sigma_1=\sqrt{18}=3\sqrt{2}$ i $\sigma_2=\sqrt{2}$ i tenim:
\[
D=\begin{pmatrix}
3\sqrt{2} & 0 \\ 0 & \sqrt{2}
\end{pmatrix}.
\]

I la matriu $P$ té per columnes les imatges de $\smat{1\\-1}$ i $\smat{1\\1}$ per $f_A$ dividides per les seves normes, per tant:
\[
\begin{pmatrix}
3 & -9/5 \\ -1 & 13/5
\end{pmatrix} 
\begin{pmatrix}
1 \\ -1 
\end{pmatrix}=
\begin{pmatrix}
24/5 \\ -18/5
\end{pmatrix}=
6 \begin{pmatrix}
4/5 \\ -3/5
\end{pmatrix}
\]
i
\[
\begin{pmatrix}
3 & -9/5 \\ -1 & 13/5
\end{pmatrix} 
\begin{pmatrix}
1 \\ 1 
\end{pmatrix}=
\begin{pmatrix}
6/5 \\ 8/5
\end{pmatrix}=
2 \begin{pmatrix}
3/5 \\ 4/5
\end{pmatrix}
\]
Per tant:
\[
P=\begin{pmatrix}
4/5 & 3/5 \\ -3/5 & 4/5
\end{pmatrix}
\]
I una descomposició d'$A$ en valors singulars és:
\[
\begin{pmatrix}
3 & -9/5 \\ -1 & 13/5
\end{pmatrix} 
=\begin{pmatrix}
4/5 & 3/5 \\ -3/5 & 4/5
\end{pmatrix} 
\begin{pmatrix}
3\sqrt{2} & 0 \\ 0 & \sqrt{2}
\end{pmatrix}
\begin{pmatrix}
1/\sqrt{2} & -1/\sqrt{2} \\ 1/\sqrt{2} & 1/\sqrt{2}
\end{pmatrix}.
\]
A la Figura \ref{fig:circ_ellipse} podem veure com es modifica la circumferència unitat quan apliquem la matriu $A$. Veiem que a la direcció $\smat{4\\-3}$ hi ha el primer eix principal amb semilongitud $3\sqrt{2}$ i a la direcció $\smat{3\\4}$ l'altre eix, amb semilongitud $\sqrt{2}$.
\begin{figure}[ht]
\center{
\begin{tabular}{ccc}
\begin{minipage}{5cm}{\includegraphics[width=\textwidth]{cercle.png}}\end{minipage}  & $\Rightarrow$ & \begin{minipage}{5cm}{\includegraphics[width=\textwidth]{ellipse.png}}\end{minipage}
\end{tabular}}
\caption{Deformació de la circumferència unitat per l'aplicació lineal $f_A$.\label{fig:circ_ellipse}}
\end{figure}
\end{exemple}

\subsection{Classificació de formes bilineals simètriques sobre \texorpdfstring{$\mathbb{R}^n$}{Rn}}
A aquesta secció l'objectiu és classificar les formes bilineals. Definim abans què vol dir que siguin equivalents:
\begin{definicio}
Diem que dues formes bilineals $\phi_1,\phi_2\colon \K^n\times\K^n\to \K$ amb matrius corresponents $[\phi_1]$ i $\phi_2]$ \emph{són equivalents} si existeix una base $\calc$ de $\K^n$ tal que
\[
[\phi_1]=[\phi_2]_\calc .
\]
Dit d'una altra manera, si existeix una matriu invertible $\cals\in M_n(\K)$ (la matriu del canvi de base, que té per columnes els vectors de $\calc$) tal que
\[
[\phi_1]=\cals^T [\phi_2] \cals.
\]
Escriurem $q_1\sim q_2$.
\end{definicio}
\begin{lema}
Ser equivalents com a formes bilineals de $\K^n\times \K^n$ a $\K$ és una relació d'equivalència. O sigui:
\begin{itemize}
    \item $\phi \sim \phi$ per a tot $\phi$ forma bilineal (reflexiva),
    \item $\phi_1\sim \phi_2 \Leftrightarrow \phi_2 \sim \phi_1$ per a totes $\phi_1, \phi_2$ formes bilineals (simètrica) i
    \item $\phi_1\sim \phi_2$ i $\phi_2\sim \phi_3$ implica $\phi_1\sim \phi_3$ per a totes $\phi_1,\phi_2,\phi_3$ formes quadràtiques (transitiva).
\end{itemize}
\end{lema}
\begin{proof}
Cal veure en cada cas quina és la matriu $\cals$ del canvi de base:
\begin{itemize}
    \item Per veure que $\phi\sim \phi$, considerem $\cals=\1_n$,
    \item Si $\phi_1\sim \phi_2$ tenim que existeix $\cals$ una matriu de canvi de base (i per tant invertible) tal que:
    \[
    [\phi_1]=\cals^T [\phi_2] \cals.
    \]
    Però aquesta igualtat també es pot escriure com:
    \[
    [\phi_2]=(\cals^{-1})^T [\phi_1] \cals^{-1}
    \]
    i per tant $\phi_2\sim \phi_1$.
    \item Per hipòtesis tenim que existeixen $\cals_1$ i $\cals_2$ tals que:
    \[
    [\phi_1]=\cals_1^T [\phi_2] \cals_1
    \text{ i }
    [\phi_2]=\cals_2^T [\phi_3] \cals_2.
    \]
    Llavors:
    \[
    [\phi_1]=\cals_1^T \cals_2^T [\phi_3] \cals_2 \cals_1=
    (\cals_2 \cals_1)^T [\phi_3] (\cals_2 \cals_1)
    \]
    i tenim $\phi_1\sim \phi_3$.
\end{itemize}
\end{proof}

A partir d'ara considerem que $\K=\R$ i l'objectiu és classificar les formes bilineals $\phi\colon \R^n\times \R^n \to \R$:
\begin{teorema}\label{teo:class_form:bilR}
Tota forma bilineal simètrica $\phi\colon \R^n\times \R^n\to \R$ és equivalent a una forma bilineal que té per matriu una matriu diagonal amb coeficients (a la diagonal): $1$ a les $r$ primeres files, $-1$ a les $s$ següents i $0$ a les $t$ últimes ($n=r+s+t$). O sigui una matriu de la forma:
\[
\begin{pmatrix}
1      & \cdots & 0 & 0 & \cdots & 0 & 0 & \cdots & 0 \\
       & \ddots &   &   & \cdots &   &   & \cdots &  \\
0      & \cdots & 1 & 0 & \cdots & 0 & 0 & \cdots & 0 \\
0      & \cdots & 0 & -1 & \cdots & 0 & 0 & \cdots & 0 \\
       &        &   &    & \ddots &   &   & \cdots &   \\
0      & \cdots & 0 &  0 & \cdots & -1 & 0 & \cdots & 0 \\
0      & \cdots & 0 &  0 & \cdots & 0 & 0 & \cdots & 0 \\
       &        &   &    & \cdots &   &   & \ddots &   \\
0      & \cdots & 0 &  0 & \cdots & 0 & 0 & \cdots & 0 
\end{pmatrix}
\]
A més, si definim la parella $(r,s)$ com la \emph{signatura de $\phi$} ($\sign(\phi)$)\index{forma bilineal! signatura}, dues formes bilineals bilineals $\phi_1,\phi_2\colon \R^n\times \R^n\to \R$ són equivalents si i només si $\sign(\phi_1)=\sign(\phi_2)$.
\end{teorema}
\begin{proof}
Considerem $[\phi]$ la matriu simètrica $n\times n$ sobre $\R$ de la forma bilineal $\phi$. Pel Teorema espectral, existeix una matriu ortogonal $Q$  i una matriu diagonal $D$ amb valors $\lambda_1, \dots, \lambda_n$ tal que:
\[
D=Q^T [\phi] Q \,.
\]
Ara fem els canvis següents:
\begin{itemize}
    \item Podem reordenar els elements de la diagonal de $D$ reordenant les columnes de $Q$: si $Q'$ és la matriu que resulta d'intercanviar les columnes $j$ i $k$ de $Q$, llavors, $Q'$ continua sent ortogonal i el producte:
    \[
    (Q')^T[\phi]Q'
    \]
    també és diagonal, però intercanviant les posicions $i$ i $j$, o sigui, els valors $\lambda_i$ i $\lambda_j$.\\
    Aprofitem aquesta propietat per reordenar la diagonal de $D$ i posar primer els $\lambda_i>o$, llavors els $\lambda_i<0$ i finalment els $\lambda_i=0$. Per tant, podem suposar que tenim $Q$ una matriu ortogonal tal que
    \[
    D=Q^T [\phi] Q \,.
    \]
    amb $D$ una matriu diagonal tal que a la diagonal té els $r$ primers coeficients positius, els $s$ següents negatius i els $t$ últims zero.
    \item Considerem $A$ la matriu diagonal següent, definida a partir dels coeficients de $D$: el coeficient $a_{ii}$ es defineix com:
    \[
    a_{ii}=\begin{cases}
    1/\sqrt{\lambda_i} & \text{si $1\leq i \leq r$,} \\
    1/\sqrt{-\lambda_i} & \text{si $r < i \leq r+s$,} \\
    1 & \text{si $r+s < i \leq n$.}
    \end{cases}
    \]
    Tenim que $A^TDA$ és una matriu diagonal amb $r$ uns a les primeres files, $s$ menys uns a les següents i zero a les últimes.\\
    Per tant:
    \[
    [\phi] \sim A^TQ^T [\phi] QA = (QA)^T [\phi] QA
    \]
    i $(QA)^T [\phi] QA$ és diagonal i com diu l'enunciat del teorema.
\end{itemize}
De moment hem vist que, com que ser equivalent a té la propietat transitiva, si $\phi_1$ i $\phi_2$ tenen la mateixa signatura, les dues són equivalents a una mateixa forma bilineal i per tant $\phi_1\sim \phi_2$.\\
Cal veure el recíproc, o el que és equivalent, que dues matrius diagonals equivalents $D$ i $D'$ amb $(r,s,t)$ i $(r',s',t')$ coeficients $1$, $-1$ i $0$ respectivament, llavors $(r,s,t)=(r',s',t')$. \\
Com que $D$ i $D'$ són equivalents, existeix $A$ una matriu invertible que té per columnes $v_i$ tal que:
\[
D'=A^T D A.
\]
Si fem el producte de la dreta, les $r$ primeres files són $||v_1||^2$ ($D$ coincideix amb la identitat a les $r$ primeres files i columnes), \ldots, $||v_r||^2$, per tant són números positius, per tant $r'\geq r$ i argumentant canviant els papers de $D$ i $D'$ (i $A$ per $A^{-1}$), tenim $r\geq r'$, pel que $r=r'$.\\
$r+s$ és el rang de $D$. Com que $D'$ és el producte de $D$ per matrius invertibles, $r+s$ també és el rand de $D'$, que és $r'+s'$, per tant, $s=s'$.\\
Finalment, $t=n-(r+s)=n-(r'+s')=t'$.
\end{proof}
\begin{exemple}\label{exemple:class_form_bil}
Considerem la forma bilineal $\phi\colon \R^4\times\R^4\to\R$ donada per la matriu:
\[
[\phi]=\left(\begin{array}{rrrr}
1 & 5 & -1 & 3 \\
5 & 1 & 3 & -1 \\
-1 & 3 & 1 & 5 \\
3 & -1 & 5 & 1
\end{array}\right).
\]
Una manera de classificar-la és de calcular el polinomi característic i estudiar el signe dels valors on s'anul·la:
\[
p_{[\phi]}(x)=\det(A-x\1_4)=x^{4} - 4 x^{3} - 64 x^{2} + 256 x .
\]
I $p_{[\phi]}(x)$ s'anul·la als valors $\{-8,0,4,8\}$, per tant té signatura $(2,1)$ i és equivalent a la forma bilineal que té per matriu:
\[
\begin{pmatrix} 1 & 0 & 0 & 0 \\ 0 & 1 & 0 & 0 \\ 0 & 0 & -1 & 0 \\ 0 & 0 & 0 & 0 \end{pmatrix} .
\]
\end{exemple}

Acabem aquesta secció aprofitant aquesta classificació per fer la corresponent de les formes quadràtiques.

\begin{definicio}
Una \emph{forma quadràtica}\index{forma quadràtica} sobre un cos $\K$ és una aplicació $q\colon \K^n\to \K$ que es pot expressar com:
\[
q(x_1, \dots , x_n)=\sum_{1\leq i , j \leq n} \lambda_{ij}x_i x_j
\]
amb $\lambda_{ij}\in \K$.

També es pot escriure com:
\[
q(x_1, \dots , x_n)= \vec x^T A \vec x ,
\]
on $\vec x=\smat{x_1 \\ \vdots \\ x_n}$ i $A$ és la matriu (simètrica) que té per coeficients
\[
a_{ij}=\begin{cases} \lambda_{ii} & \text{si $i=j$,} \\ \frac{\lambda_{ij}+\lambda_{ji}}{2} & \text{si $i\neq j$.}\end{cases}
\]
\end{definicio}

\begin{observacio}
Amb el que hem vist, tenim una bijecció entre les matrius simètriques (que es poden pensar com formes bilineals en una base donada) i les formes quadràtiques. Escriurem $[q]$ per denotar la matriu simètrica corresponent a la forma quadràtica $q$.
\end{observacio}
\begin{exemple}
La forma quadràtica $q\colon \R^2 \to \R$ definida per $q(x_1,x_2)=ax_1^2+bx_1x_2+cx_2^2$ correspon a la matriu simètrica
\[
\begin{pmatrix}
a & b/2 \\ b/2 & c
\end{pmatrix} .
\]
Podem recuperar la forma quadràtica fent:
\[
q(x_1,x_2)=\begin{pmatrix} x_1 & x_2 \end{pmatrix}\begin{pmatrix}
a & b/2 \\ b/2 & c
\end{pmatrix} \begin{pmatrix} x_1 \\ x_2 \end{pmatrix}=ax_1^2 + bx_1x_2+cx_2^2 .
\]
\end{exemple}
\begin{exemple}
Considerem les formes quadràtiques $q_1$ i $q_2$ de $\R^2$ a $\R$ definides com $q_1(x_1,x_2)=x_1x_2$ i $q_2(y_1,y_2)=y_1^2-y_2^2$. Observem que si fem el canvi de base:
\[
\begin{pmatrix} y_1 \\ y_2  \end{pmatrix} =
\begin{pmatrix} 1/2 & 1/2 \\ 1/2 & -1/2 \end{pmatrix}
\begin{pmatrix} x_1 \\ x_2  \end{pmatrix}
\]
es poden escriure:
\[
y_1=\frac12x_1+\frac12x_2 \text{ i } y_2=\frac12x_1-\frac12x_2 
\]
i tenim
\[
q_2(y_1,y_2)=(\frac12x_1+\frac12x_2)^2 - (\frac12x_1-\frac12x_2)^2=x_1x_2 .
\]
Per tant, tenen la mateixa forma. Aquesta igualtat també es pot veure en forma matricial observant que si $\cals$ és la matriu del canvi de base, tenim $[q_1]=\cals^T [q_2] \cals$, on $[q_1]$ i $[q_2]$ són les matrius simètriques corresponents a les formes bilineals $q_1$ i $q_2$ respectivament:
\[
\begin{pmatrix} 0 & 1/2 \\ 1/2 & 0 \end{pmatrix}=
\begin{pmatrix} 1/2 & 1/2 \\ 1/2 & -1/2 \end{pmatrix}^T
\begin{pmatrix} 1 & 0 \\ 0 & -1 \end{pmatrix}
\begin{pmatrix} 1/2 & 1/2 \\ 1/2 & -1/2 \end{pmatrix}
\]
\end{exemple}
\begin{definicio}
Diem que dues formes quadràtiques $q_1,q_2\colon \K^n\to \K$ \emph{són equivalents}, amb matrius  si existeix un canvi de base $\cals$ a $\K^n$ tal que 
\[
[q_1]=\cals^T [q_2] \cals.
\]
Escriurem $q_1\sim q_2$.
\end{definicio}

Com que els canvis de base afecten de la mateixa manera que afectaven a les formes bilineals (veure Lema \ref{lema:canvi_base_forma_bil}), obtenim que ``\emph{ser equivalent a}'' també es una relació d'equivalència a les formes quadràtiques i l'anàleg al Teorema \ref{teo:class_form:bilR}:
\begin{teorema}\label{teo:class_formQuadR}
Tota forma quadràtica $q\colon \R^n\to \R$ és equivalent a una forma bilineal del tipus:
\[
q_1(x_1,\dots, x_n)=x_1^2+ \cdots + x_r^2 - x_{r+1}^2-\cdots -x_{r+s}^2 .
\]
A més, si definim la parella $(r,s)$ com la \emph{signatura de $q$} ($\sign(q)$)\index{forma quadràtica! signatura}, dues formes quadràtiques $q_1,q_2\colon \R^n\to \R$ són equivalents si i només si $\sign(q_1)=\sign(q_2)$.
\end{teorema}
\begin{exemple}
Classifiquem la forma quadràtica $q\colon \R^4 \to \R$:
\[
q(x_1,x_2,x_3,x_4)=x_1^2+10 x_1x_2 -2 x_1x_3+6x_1x_4 + x_2^2 -6x_2x_3-2x_2x_4+x_3^3+10x_3x_4-x_4^2.
\]
Que té per matriu:
\[
\left(\begin{array}{rrrr}
1 & 5 & -1 & 3 \\
5 & 1 & 3 & -1 \\
-1 & 3 & 1 & 5 \\
3 & -1 & 5 & 1
\end{array}\right)
\]
Per tant, podem aprofitar els càlculs de l'Exemple \ref{exemple:class_form_bil}, i tenim que és equivalent a:
\[
q(y_1,y_2,y_3,y_4)=y_1^2+y_2^2-y_3^2.
\]
\end{exemple}

\begin{llista-exercicis}
\item[Secció 5.1:] 12, 16, 18.
\item[Secció 5.2:] 14, 34, 38.
\item[Secció 5.3:] 5-11, 13-20, 32.
\item[Secció 5.4:] 2, 8, 10.
\item[Secció 5.5:] 4, 10, 16, 22.
\item[Secció 8.1:] 6, 12, 16, 22.
\item[Secció 8.2:] 4, 10, 18, 22.
\item[Secció 8.3:] 6, 16, 18, 20.
\end{llista-exercicis}
% !TeX encoding = UTF-8
% !TeX spellcheck = ca_ES-valencia
% !TeX root = MatCADAlgLin.tex
Podem definir els nombres complexos com
\[
\C =\{a+bi ~|~ a,b\in\R\}.
\]
Com a espai vectorial sobre $\R$, podem identificar $\C$ amb $\R^2$, i per tant tenim definida una suma i una producte per reals:
\[
(a+bi) + (c+di) = (a+c) + (b+d)i, \quad \lambda(a+bi)=\lambda a + \lambda b i.
\]
Tenim però una operació addicional, el producte de nombres complexos:
\[
(a+bi)(c+di) = (ac-bd) + (ad+bc)i,
\]
on la fórmula es recorda fàcilment si tenim en compte que $i^2=-1$, i fent servir la propietat distributiva.

Definim la \emph{part real} i la \emph{part imaginària} \index{part real}\index{part imaginària} d'un nombre complex com:
\[
\Re(a+bi) = a,\quad \Im(a+bi) = b.
\]
També el \emph{conjugat}\index{conjugat} d'un nombre complex es defineix com:
\[
\overline{(a+bi)} = a-bi,
\]
i per tant tenim les fórmules
\[
\Re(z) = \frac{z+\bar z}{2},\quad \Im(z)=\frac{z-\bar z}{2i},\quad \forall z\in\C.
\]
La \emph{norma}\index{norma} d'un element $z=a+bi$ és
\[
N(z) = z\bar z = (a+bi)(a-bi) = a^2+b^2\geq 0.
\]
Observem que $z\bar z \geq 0$, i que $z\bar z = 0$ si i només si $z=0$. Això ens permet calcular fàcilment una fórmula per l'invers d'$a+bi$:
\[
(a+bi)^{-1} = \frac{1}{a+bi} = \frac{a-bi}{(a+bi)(a-bi)} = \frac{a}{a^2+b^2}+\frac{-b}{a^2+b^2}i.
\]
Pensat com un element de $\R^2$, $N(z)=|z|^2$, on $|z|=\sqrt{a^2+b^2}$ és el mòdul de $z$ pensat com un element de $\R^2$. Les coordenades polars del punt $(a,b)\in\R^2$ es calculen fent servir que:
\[
a = r\cos(\theta),\quad b = r\sin(\theta),
\]
i per tant:
\[
r = \sqrt{z\bar z}=\sqrt{a^2+b^2},\quad \theta = \arg(z) = \arctan(b/a) (+\pi),
\]
on haurem de sumar $\pi$ a $\arctan(b/a)\in[-\pi/2,\pi/2]$ si el punt $(a,b)$ pertany al II o III quadrant.

Podem escriure (aquí ho podem pensar com una notació, encara que té una justificació algebraica) que
\[
a+bi = re^{i\theta},
\]
i així podem recordar les fórmules
\[
|zw|=|z||w|,\quad \arg(zw)=\arg(z)+\arg(w)\pmod{2\pi}.
\]
El següent resultat és el motiu pel què ens interessa treballar a $\C$:

\begin{teorema}[Teorema fonamental de l'àlgebra]
Tot polinomi $f(x)\in \C[x]$ de grau $n\geq 0$ es pot escriure com
\[
f(x) = c (x-\lambda_1)\cdots (x-\lambda_n),
\]
on $c$ i $\lambda_1,\ldots,\lambda_n$ són nombres complexos (possiblement repetits).
\end{teorema}


\printindex
\begin{thebibliography}{20}
\bibitem{Bret}
	Otto Bretscher,
	\textit{Linear Algebra with Applications},
	1997, Prentice-Hall International Inc.
\bibitem{NaXa}
	Enric Nart, Xavier Xarles,
	\textit{Apunts d'àlgebra lineal}, 2016, Col.lecci\'o Materials, 237, UAB.
\end{thebibliography}
\end{document}